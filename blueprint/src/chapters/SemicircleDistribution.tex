\chapter{SemicircleDistribution}

\section{Semicircle Probability Density Function}



%---Richard's Part for Semicircle Distribution (Start)---% 
%---------%
%---------%
\begin{definition}\label{def:semicirclePDFReal}
  \mathlibok 
  \lean{semicirclePDFReal}
    The function $f : \mathbb{R} \times \mathbb{R}_{\geq 0} \times \mathbb{R} \rightarrow \mathbb{R}$ defined by
   \[
    f(\mu,v,x) 
    = \frac{1}{2πv} \sqrt{(4v - (x - μ)^2)_+}
   \]
   is called the probability density function (p.d.f.) of the semicircle distribution.
\end{definition}
%---------%
%---------%
\begin{lemma}\label{lem:semicirclePDFReal_def}
  \mathlibok 
  \lean{semicirclePDFReal_def}
  \uses{def:semicirclePDFReal}
    Given a mean $\mu \in \mathbb{R}$ and a variance $v \in \mathbb{R}_{\geq 0}$, the p.d.f. $f : \mathbb{R} \rightarrow \mathbb{R}$ 
    of the semicircle distribution with mean $\mu$ and variance $v$ is given by
  \[
    f(x) = 
    \frac{1}{2πv} \sqrt{(4v - (x - μ)^2)_+}.
  \]
\end{lemma}
%---------%
%---------%
\begin{lemma}\label{lem:semicirclePDFReal_zero_var}
    \mathlibok
    \lean{semicirclePDF_zero_var}
    \uses{def:semicirclePDFReal}
    If the variance $v$ is given to be zero, then the p.d.f. of the semicircle distribution is the zero functional.
%    In other words, for any mean $m \in \mathbb{R}$ and $x \in \mathbb{R}$, we have
%    \[
%    f(m,0,x) = 0.
%    \]
\end{lemma}
\begin{proof}
    By Definition \ref{def:semicirclePDFReal}, the square root of a nonpositive number is defined to be zero.
    Hence, the p.d.f. with a zero variance must be the zero functional.
\end{proof}
%---------%
%---------%
\begin{lemma}\label{lem:semicirclePDFReal_nonneg}
    \mathlibok
    \lean{semicirclePDFReal_nonneg}
    \uses{def:semicirclePDFReal}
    The p.d.f. of the semicircle distribution is always nonnegative. 
\end{lemma}
\begin{proof}
   By Definition \ref{def:semicirclePDFReal}, the square root of a nonpositive number is defined to be zero. 
   Furthermore, the variance is always assumed to be nonnegative.
   Therefore, since the fractional term and the square root term are always nonnegative,
   we conclude the p.d.f. is always nonnegative. 
\end{proof}
%---------%
%---------%
\begin{lemma}\label{lem:measurable_semicirclePDFReal}
    \lean{measurable_semicirclePDFReal}
    \uses{}
    \notready
    Given a mean $\mu \in \mathbb{R}$ and a variance $v \in \mathbb{R}_{\geq 0}$, the p.d.f. $f : \mathbb{R} \rightarrow \mathbb{R}$ 
    of the semicircle distribution with mean $\mu$ and variance $v$ is measurable.
\end{lemma}
\begin{proof}
% sorry
\end{proof}
%---------%
%---------%
\begin{lemma}\label{lem:stronglyMeasurable_semicirclePDFReal}
    \mathlibok
    \lean{stronglyMeasurable_semicirclePDFReal}
    \uses{def:semicirclePDFReal,lem:measurable_semicirclePDFReal}
    Given a mean $\mu \in \mathbb{R}$ and a variance $v \in \mathbb{R}_{\geq 0}$, the p.d.f. $f : \mathbb{R} \rightarrow \mathbb{R}$ 
    of the semicircle distribution with mean $\mu$ and variance $v$ is strongly measurable.
\end{lemma}
\begin{proof}
    By Lemma \ref{lem:measurable_semicirclePDFReal}, we know the p.d.f. $f$ with fixed mean $\mu$ and variance $v$ is measurable.
    Since $\mathbb{R}$ is equipped with a second countable topology, the fact that $f$ with fixed mean $\mu$ and variance $v$ implies $f$ is strongly measurable.  
\end{proof}
%---------%
%---------%
\begin{lemma}\label{lem:integrable_semicirclePDFReal}
    \lean{integrable_semicirclePDFReal}
    \uses{def:semicirclePDFReal}
    \notready
    Given a mean $\mu \in \mathbb{R}$ and a variance $v \in \mathbb{R}_{\geq 0}$, the p.d.f. $f : \mathbb{R} \rightarrow \mathbb{R}$ 
    of the semicircle distribution with mean $\mu$ and variance $v$ is integrable.
\end{lemma}
\begin{proof}
% sorry
\end{proof}
%---------%
%---------%
\begin{lemma}\label{lem:lintegral_semicirclePDFReal_eq_one}
    \lean{lintegral_semicirclePDFReal_eq_one}
    \uses{def:semicirclePDFReal}
    \notready
    Given a mean $\mu \in \mathbb{R}$ and a nonzero variance $v \in \mathbb{R}_{> 0}$, the lower integral of the p.d.f. $f : \mathbb{R} \rightarrow \mathbb{R}$ 
    of the semicircle distribution with mean $\mu$ and variance $v$ equals $1$.
\end{lemma}
\begin{proof}
% sorry
\end{proof}
%---------%
%---------%
\begin{lemma}\label{lem:integral_semicirclePDFReal_eq_one}
    \lean{integral_semicirclePDFReal_eq_one}
    \uses{def:semicirclePDFReal}
    \notready
    Given a mean $\mu \in \mathbb{R}$ and a nonzero variance $v \in \mathbb{R}_{> 0}$, the integral of the p.d.f. $f : \mathbb{R} \rightarrow \mathbb{R}$ 
    of the semicircle distribution with mean $\mu$ and variance $v$ equals $1$.
\end{lemma}
\begin{proof}
% sorry
\end{proof}
%---------%
%---------%
\begin{lemma}\label{lem:semicirclePDFReal_sub}
    \mathlibok
    \lean{semicirclePDFReal_sub}
    \uses{def:semicirclePDFReal}
    For any p.d.f. $f : \mathbb{R} \rightarrow \mathbb{R}$ 
    of the semicircle distribution, the following relation is satisfied:
    \[
    f(\mu,v,x-y) = f(\mu+y,v,x)
    \]
    for any $u \in \mathbb{R}$, $v \in \mathbb{R}_{\geq 0}$, and $x,y \in \mathbb{R}$. 
\end{lemma}
\begin{proof}
   Expanding Definition \ref{def:semicirclePDFReal} gives
   \[
   f(\mu,v,x-y) 
   = \frac{1}{2πv} \sqrt{\bigl( 4v - ( (x - y) - μ)^2 \bigl)_+} 
   = \frac{1}{2πv} \sqrt{\bigl( 4v - (x - (μ + y))^2 \bigl)_+}
   = f(\mu+y,v,x).
   \]
\end{proof}
%---------%
%---------%
\begin{lemma}\label{lem:semicirclePDFReal_add}
    \mathlibok
    \lean{semicirclePDFReal_add}
    \uses{def:semicirclePDFReal,lem:semicirclePDFReal_sub}
    For any p.d.f. $f : \mathbb{R} \rightarrow \mathbb{R}$ 
    of the semicircle distribution, the following relation is satisfied:
    \[
    f(\mu,v,x+y) = f(\mu-y,v,x)
    \]
    for any $u \in \mathbb{R}$, $v \in \mathbb{R}_{\geq 0}$, and $x,y \in \mathbb{R}$. 
\end{lemma}
\begin{proof}
   Expanding Definition \ref{def:semicirclePDFReal} gives
   \[
   f(\mu,v,x+y) 
   = \frac{1}{2πv} \sqrt{\bigl( 4v - ( (x + y) - μ)^2 \bigl)_+} 
   = \frac{1}{2πv} \sqrt{\bigl( 4v - (x - (μ - y))^2 \bigl)_+}
   = f(\mu-y,v,x).
   \]
\end{proof}
%---------%
%---------%
\begin{lemma}\label{lem:semicirclePDFReal_inv_mul}
    \lean{semicirclePDFReal_inv_mul}
    \uses{def:semicirclePDFReal}
    \notready
    For any p.d.f. $f : \mathbb{R} \rightarrow \mathbb{R}$ 
    of the semicircle distribution, the following relation is satisfied:
    \[
    f(\mu,v,c^{-1} x) = |c| \cdot f(c \mu,c^2 v,x)
    \]
    for any $u \in \mathbb{R}$, $v \in \mathbb{R}_{\geq 0}$, $x \in \mathbb{R}$, and nonzero $c \in \mathbb{R}$. 
\end{lemma}
\begin{proof}
% sorry
\end{proof}
%---------%
%---------%
\begin{lemma}\label{lem:semicirclePDFReal_mul}
    \mathlibok
    \lean{semicirclePDFReal_mul}
    \uses{def:semicirclePDFReal,lem:semicirclePDFReal_inv_mul}
    For any p.d.f. $f : \mathbb{R} \rightarrow \mathbb{R}$ 
    of the semicircle distribution, the following relation is satisfied:
    \[
    f(\mu,v,cx) = |c^{-1}| \cdot f(c^{-1} \mu,c^{-2} v,x)
    \]
    for any $u \in \mathbb{R}$, $v \in \mathbb{R}_{\geq 0}$, $x \in \mathbb{R}$, and nonzero $c \in \mathbb{R}$. 
\end{lemma}
\begin{proof}
   Expanding Definition \ref{def:semicirclePDFReal} gives
   \[
   f(\mu,v,c x)
   = \frac{1}{2πv} \sqrt{\bigl( 4v - ( cx - μ)^2 \bigl)_+} 
   = \frac{1}{2πv} \sqrt{\bigl( 4v -  c^2(x -  c^{-1}μ)^2 \bigl)_+} 
   = |c^{-1}| \frac{1}{2π(c^{-2}v)} \sqrt{\bigl( 4(c^{-2}v)  - (x - c^{-1} \mu)^2 \bigl)_+}
   = |c^{-1}| \cdot f(c^{-1} \mu,c^{-2} v,x).
   \]
\end{proof}
%---------%
%---------%
%---Richard's Part for Semicircle Distribution (End)---% 


\section{To Extended Nonnegative Reals}


%---Paul's Part for Semicircle Distribution (Start)---% 


\begin{definition}
  \label{def:semicirclePDF}
  \leanok
  \lean{ProbabilityTheory.semicirclePDF}
  \uses{def:semicirclePDFReal}
  Let $f : \mathbb{R} \times \mathbb{R}_{\geq 0} \times \mathbb{R} \to \mathbb{R}$ denote the real-valued semicircle density defined in Definition \ref{def:semicirclePDFReal}. Define the function $h: \mathbb{R} \to \mathbb{R} \cup \{\infty\} $ as follows:
  $$
  h(x) := \begin{cases}
x & \text{if } x \ge 0, \\
0 & \text{otherwise}.
\end{cases}
  $$ 
  Then we define the function $ g : \mathbb{R} \times \mathbb{R}_{\geq 0} \times \mathbb{R} \to [0,\infty] \subseteq \overline{\mathbb{R}}_{\ge 0}$ by:
    $$
    \bar{sc} (\mu,v,x) := h(sc(\mu,v,x)),
    $$
\end{definition}

\begin{lemma}
  \leanok
  \label{lem:semicirclePDF_def}
  \lean{ProbabilityTheory.semicirclePDF_def}
  \uses{def:semicirclePDF}
  For all $\mu \in \mathbb{R} , v \in \mathbb{R}_{\geq 0}$, the extended pdf. $ \bar{sc} : \mathbb{R} \to [0,\infty]$  satisfies:
  $$
    \bar{sc}(\mu,v) = \left( x \mapsto h(sc(\mu,v,x)) \right).
  $$
\end{lemma}

\begin{lemma}
  \leanok
  \label{lem:semicirclePDF_zero_var}
  \lean{ProbabilityTheory.semicirclePDF_zero_var}
  \uses{def:semicirclePDF,lem:semicirclePDFreal_zero_var}
  If the variance $v$ is zero, then the extended pdf. is identically zero:
 $$ 
    \forall x \in \mathbb{R}, \quad \bar{sc}(\mu,0,x) = 0.
 $$ 
\end{lemma}
\begin{proof}
    This follows immediately from the definition of $\bar{sc}$ as $h(sc(\mu,0,x))$, and the fact that $sc(\mu,0,x) = 0$ from Lemma \ref{lem:semicirclePDFReal_zero_var}.
\end{proof}

\begin{lemma}
    \leanok
    \label{lem:toReal_semicirclePDF}
    \lean{ProbabilityTheory.toReal_semicirclePDF}
    \uses{lem:semicirclePDFReal_nonneg}
    Let $ \mu \in \mathbb{R} ,  v \in \mathbb{R}_{\ge 0}$, and $x \in \mathbb{R} $.
Then the real value recovered from the extended semicircle PDF satisfies:
\[
    \bar{sc}(\mu, v, x)^{\operatorname{toReal}} = sc(\mu, v, x).
\]
\end{lemma}
\begin{proof}
Since $sc(\mu, v, x) \ge 0$, we have $h(sc(\mu, v, x)) = sc(\mu, v, x)$, and thus
\[
    \bar{sc}(\mu, v, x) = h(sc(\mu, v, x)) = sc(\mu, v, x).
\]
Therefore,
\[
    \bar{sc}(\mu, v, x)^{\operatorname{toReal}} = sc(\mu, v, x),
\]
as desired.
\end{proof}

\begin{lemma}
  \leanok
  \label{lem:semicirclePDF_nonneg}
  \lean{ProbabilityTheory.semicirclePDF_nonneg}
  \uses{def:semicirclePDF}
  If $v > 0$, then for all $\mu, x \in \mathbb{R}$, the extended pdf is nonnegative:
  \[
      0 \le \bar{sc}(\mu,v,x).
  \]
\end{lemma}
\begin{proof}
    This is immediate from the definition of $\bar{sc}$ as $h(sc(\mu,v,x))$ and the nonnegativity of $sc$ (Lemma~\ref{lem:semicirclePDFReal_nonneg}).
\end{proof}

\begin{lemma}
  \leanok
  \label{lem:semicirclePDF_finite}
  \lean{ProbabilityTheory.semicirclePDF_lt_top}
  \uses{def:semicirclePDF}
  For all $\mu, x \in \mathbb{R}$, and $v \in \mathbb{R}_{\ge 0}$, we have:
    $$ 
    \bar{sc}(\mu,v,x) < \infty.
    $$ 
\end{lemma}
\begin{proof}
Since $sc(\mu,v,x) \in \mathbb{R}_{\ge 0}$, we have $\bar{sc}(\mu,v,x) = h(sc(\mu,v,x)) < \infty$.
\end{proof}

\begin{lemma}
  \leanok
  \label{lem:semicirclePDF_ne_top}
  \lean{ProbabilityTheory.semicirclePDF_ne_top}
  \uses{def:semicirclePDF}
  For all \( \mu, x \in \mathbb{R} \), and \( v \in \mathbb{R}_{\ge 0} \), the extended pdf is finite:
  \[
    \bar{sc}(\mu,v,x) \ne \infty.
  \]
\end{lemma}

\begin{lemma}
  \leanok
  \label{lem:support_semicirclePDF}
  \lean{ProbabilityTheory.support_semicirclePDF}
  \uses{def:semicirclePDF,def:semicirclePDFReal}
  Let \( \mu \in \mathbb{R} \) and \( v \in \mathbb{R}_{> 0} \). Then the support of the extended pdf is
  \[
      \operatorname{supp}(\bar{sc}(\mu,v)) = \left\{ x \in \mathbb{R} : sc(\mu,v,x) \ne 0 \right\}
    = \left(\mu - 2\sqrt{v}, \mu + 2\sqrt{v})\right.
  \]
\end{lemma}
\begin{proof}
    sorry
\end{proof}

\begin{lemma}
  \leanok
  \label{lem:measurable_semicirclePDF}
  \lean{ProbabilityTheory.measurable_semicirclePDF}
  \uses{lem:measurable_semicirclePDFReal}
  The function \( x \mapsto \bar{sc}(\mu,v,x) \) is measurable for all \( \mu \in \mathbb{R} \), \( v \in \mathbb{R}_{\ge 0} \).
\end{lemma}
\begin{proof}
  Since $h$ is measurable, and $h$ is a measurable map \( \mathbb{R}_{\ge 0} \to \overline{\mathbb{R}}_{\ge 0} \), their composition is measurable.
\end{proof}

\begin{lemma}
  \leanok
  \label{lem:lintegral_semicirclePDF_eq_one}
  \lean{ProbabilityTheory.lintegral_semicirclePDF_eq_one}
  \uses{def:semicirclePDF,lem:lintegral_semicirclePDFReal_eq_one}
  If $v > 0$, then the total integral of $\bar{sc}$ with respect to Lebesgue measure is 1:
  \[
      \int_{\mathbb{R}} \bar{sc}(\mu,v,x) \, dx = 1.
  \]
\end{lemma}
\begin{proof}
  This follows from the equality:
  \[
  \int_{\mathbb{R}} h(sc(\mu,v,x)) \, dx = h( \left( \int_{\mathbb{R}} sc(\mu,v,x) \, dx \right) = h(1) = 1
  \]
  using Lemma \ref{lem:lintegral_semicirclePDFReal_eq_one}.
\end{proof}

%---Paul's Part for Semicircle Distribution (End)---% 


\section{Semicircle Distribution}


%---Kiran's Part for Semicircle Distribution (End)---% 


\begin{definition}
    \label{def:semicircle_distribution}
    \uses{def:semicirclePDF}
    \leanok
    The semicircle distribution with mean $\mu$ and variance $v$ is the Dirac delta at μ if v = 0; otherwise, it's the Lebesgue measure weighted by the semicircle PDF.
\end{definition}

\begin{lemma}
    \label{lemma:semicircleReal_of_var_ne_zero}
    \uses{def:semicircle_distribution}
    \leanok
    If $v \neq 0$, then the definition the semicircle distribution is defined as the Lebesgue measure weighted by the semicircle probability density function.
\end{lemma}

\begin{proof}
    Follows directly from definition of semicircle distribution.
\end{proof}

\begin{lemma}
    \label{lemma:semicircleReal_zero_var}
    \uses{def:semicircle_distribution}
    \leanok
    If the variance is 0, then the semicircle distribution is exactly the Dirac measure at $\mu$.
\end{lemma}

\begin{proof}
    Follows directly from definition of semicircle distribution.
\end{proof}

\begin{lemma} %instance: unsure if this should be a lemma or something else
    \label{lemma:instIsProbabilityMeasuresemicircleReal}
    \uses{lemma:semicircleReal_of_var_ne_zero}
    \leanok
    The measure semicircleReal is a probability measure, no matter the values of $\mu \in \mathbb{R}$ and $v \in \mathbb{R}_{\ge 0}$.
\end{lemma}

%proof for instance? unsure

\begin{lemma}
    \label{lemma:noAtoms_semicircleReal}
    \uses{lemma:semicircleReal_of_var_ne_zero, lemma:instIsProbabilityMeasuresemicircleReal} %idk if this is right
    \leanok
    If the variance v is nonzero, then the semicircle distribution has no atoms.
\end{lemma}

\begin{lemma}
    \label{lemma:semicircleReal_apply}
    \uses{lemma:semicircleReal_of_var_ne_zero}
    \leanok
    For a semicircle measure with mean $\mu$ and nonzero variance v, the measure of any measurable set $s$ equals the Lebesgue integral over $s$ of the semicircle probability density function at x.
\end{lemma}

\begin{lemma}
    \label{lemma:semicircleReal_apply_eq_integral}
    \uses{lemma:semicircleReal_apply, lemma:integrable_semicirclePDFReal, lemma:semicirclePDFReal_nonneg}
    \leanok
    For any real mean $\mu$, and any nonnegative variance $v$ that is not zero, and any measurable set $s$ of real numbers, the semicircle distribution measure of the set $s$ equals the extended nonnegative real number version (ENNReal.ofReal) of the integral of the semicircle probability density function over s.
\end{lemma}

\begin{lemma}
    \label{lemma:semicircleReal_absolutelyContinuous}
    \uses{lemma:semicircleReal_of_var_ne_zero}
    \leanok
    For a semicircle distribution with mean $\mu$ and nonzero variance $v$, the measure semicircleReal $μ$ $v$ is absolutely continuous with respect to the Lebesgue measure.
\end{lemma}

\begin{lemma}
    \label{lemma:rnDeriv_semicircleReal}
    \uses{lemma:semicircleReal_zero_var, lemma:semicirclePDF_zero_var, lemma:semicircleReal_of_var_ne_zero, lemma:measurable_semicirclePDF}
    \leanok
    The Radon–Nikodym derivative of the semicircle measure semicircleReal $\mu$ $v$ with respect to the Lebesgue measure is almost everywhere equal to the semicircle probability density function semicirclePDF $(\mu$, $v)$.
\end{lemma}

\begin{lemma}
    \label{lemma:integral_semicircleReal_eq_integral_smul}
    \uses{def:semicircleReal, lemma:measurable_semicirclePDF, lemma:semicirclePDF_lt_top}
    \leanok

    Let $ f : \mathbb{R} \to E $ be a function where $ E $ is a normed vector space over $\mathbb{R}$. For the semicircle distribution with mean $\mu \in \mathbb{R}$ and variance $ v > 0 $, we have:

    \[
    \int f(x) \, d(\mathrm{semicircleReal}\ \mu\, v)(x) = \int \mathrm{semicirclePDFReal}(\mu, v, x) \cdot f(x) \, dx.
    \]
    
    % In other words, integrating $ f $ with respect to the semicircle measure is equivalent to integrating the product of $ f $ and the semicircle probability density function with respect to Lebesgue measure.
    
\end{lemma}

%---Kiran's Part for Semicircle Distribution (End)---% 


\section{Transformations}


%---Haoyan's Part for Semicircle Distribution (Start)---%


\begin{lemma}\label{lem:semicircleReal_map_add_const}
  \notready
  \lean{semicircleReal_map_add_const}
  The map of a semicircle distribution by addition of a constant is semicircular. That is,
  given a constant $y \in \mathbb{R}$, $ \mathrm{SC}(\mu, v) \circ (X \mapsto X + y )^{-1} = \mathrm{SC}(\mu + y, v)$.
  \begin{proof}
    %sorry
  \end{proof}
\end{lemma}


\begin{lemma}\label{lem:semicircleReal_map_const_add}
  \mathlibok
  \lean{semicircleReal_map_const_add}
  The map of a semicircle distribution by addition of a constant is semicircular. That is,
  given a constant $y \in \mathbb{R}$, $ \mathrm{SC}(\mu, v) \circ (X \mapsto y + X )^{-1} = \mathrm{SC}(y + \mu, v)$.
  \begin{proof}
    Obvious from commutativity between $X + y$ and $y + X$.
  \end{proof}
\end{lemma}


\begin{lemma}\label{lem:semicircleReal_map_const_mul}
  \notready
  \lean{semicircleReal_map_const_mul}
    The map of a semicircle distribution by multiplication by a constant is semicircular. That is,
  given a constant $c \in \mathbb{R}$, $ \mathrm{SC}(\mu, v) \circ (X \mapsto cX )^{-1} = \mathrm{SC}(c\mu , c^2v)$.
  \begin{proof}
     %sorry
  \end{proof}
\end{lemma}



\begin{lemma}\label{lem:semicircleReal_map_mul_const}
  \mathlibok
  \lean{semicircleReal_map_mul_const}
  \uses{lem:semicircleReal_map_const_mul}
   The map of a semicircle distribution by multiplication by a constant is semicircular. That is,
  given a constant $c \in \mathbb{R}$, $ \mathrm{SC}(\mu, v) \circ (X \mapsto Xc )^{-1} = \mathrm{SC}(\mu c , c^2v)$.
  \begin{proof}
    Use commutativity between $Xc$ and $cX$.
  \end{proof}
\end{lemma}



\begin{lemma}\label{lem:semicircleReal_map_neg}
  \mathlibok
  \lean{semicircleReal_map_neg}
  \uses{lem:semicircleReal_map_const_mul}
  Given a constant $c \in \mathbb{R}$, $ \mathrm{SC}(\mu, v) \circ (X \mapsto -X )^{-1} = \mathrm{SC}(-\mu  , v)$
  \begin{proof}
     Special case of the multiplication by constant map with constant being $-1$.
  \end{proof}
\end{lemma}

\begin{lemma}\label{lem:semicircleReal_map_sub_const}
  \mathlibok
  \lean{semicircleReal_map_sub_const}
  \uses{lem:semicircleReal_map_add_const}
   The map of a semicircle distribution by multiplication by a constant is semicircular. That is,
  given a constant $y \in \mathbb{R}$, $ \mathrm{SC}(\mu, v) \circ (X  \mapsto X - y )^{-1} = \mathrm{SC}(\mu - y  , v)$
  \begin{proof}
   Use the map by addition of constant and substitute constant for its $-1$ multiple.
  \end{proof}
\end{lemma}


\begin{lemma}\label{lem:semicircleReal_map_const_sub}
  \mathlibok
  \lean{semicircleReal_map_const_sub}
  \uses{lem:semicircleReal_map_neg, lem:semicircleReal_map_const_add}
  The map of a semicircle distribution by multiplication by a constant is semicircular. That is,
  given a constant $y \in \mathbb{R}$, $ \mathrm{SC}(\mu, v) \circ (X \mapsto y-X  )^{-1} = \mathrm{SC}(y-\mu  , v)$
  \begin{proof}

  \end{proof}
\end{lemma}


\begin{lemma}\label{lem:semicircleReal_add_const}
  \mathlibok
  \lean{semicircleReal_add_const}
  \uses{lem:semicircleReal_map_add_const}
  Given a real random variable $X \sim \mathrm{SC}(\mu, v)$
  then for a constant $y \in \mathbb{R}$, $X + y \sim \mathrm{SC}(\mu + y, v)$
  \begin{proof}
  \end{proof}
\end{lemma}


\begin{lemma}\label{lem:semicircleReal_const_add}
  \mathlibok
  \lean{semicircleReal_const_add}
  \uses{lem:semicircleReal_add_const}
  Given a real random variable $X \sim \mathrm{SC}(\mu, v)$
  then for a constant $y \in \mathbb{R}$, $X + y \sim \mathrm{SC}(y + \mu, v)$
  \begin{proof}

  \end{proof}
\end{lemma}


\begin{lemma}\label{lem:semicircleReal_const_mul}
  \mathlibok
  \lean{semicircleReal_const_mul}
  \uses{lem:semicircleReal_map_const_mul}
  Given a real random variable $X \sim \mathrm{SC}(\mu, v)$,
  then for a constant $c \in \mathbb{R}$, $cX \sim \mathrm{SC}(c\mu , c^2v)$
  \begin{proof}

  \end{proof}
\end{lemma}


\begin{lemma}\label{lem:semicircleReal_mul_const}
  \mathlibok
  \lean{semicircleReal_mul_const}
  \uses{lem:semicircleReal_const_mul}
   Given a real random variable $X \sim \mathrm{SC}(\mu, v)$,
  then for a constant $c \in \mathbb{R}$, $Xc \sim \mathrm{SC}(\mu c , c^2v)$
  \begin{proof}

  \end{proof}
\end{lemma}

%end Transformation




%moments

\begin{lemma}\label{lem:integral_id_semicircleReal}
  \lean{integral_id_semicircleReal}
  \notready
  $$\mathbb{E}[X] = \int x d \sigma = \mu$$
\end{lemma}

\begin{lemma}\label{lem:variance_fun_id_semicircleReal}
  \lean{variance_fun_id_semicircleReal}
  \notready
  $Var(X) = v$
\end{lemma}


\begin{lemma}\label{lem:variance_id_semicircleReal}
  \lean{variance_id_semicircleReal}
  \notready
  The variance of a real semicircle distribution with parameter $(\mu, v)$ is
  its variance parameter $v$
\end{lemma}


\begin{lemma}\label{lem:memLp_id_semicircleReal}
  \lean{memLp_id_semicircleReal}
  \notready
  All the moments of a real semicircle distribution are finite. That is, the identity is in $L_p$ for
  all finite $p$
\end{lemma}


\begin{lemma}\label{lem:centralMoment_two_mul_semicircleReal}
  \lean{centralMoment_fun_two_mul_semicircleReal}
  \notready
   $\mathbb{E}[(X  - \mu)^{2n}] = v_n C_n $
\end{lemma}

\begin{lemma}\label{lem:centralMoment_fun_two_mul_semicircleReal}
  \lean{centralMoment_fun_two_mul_semicircleReal}
  \notready
   $\mathbb{E}[(X  - \mu)^{2n}] = v_n C_n $
\end{lemma}

\begin{lemma}\label{lem:centralMoment_odd_semicircleReal}
  \lean{centralMoment_odd_semicircleReal}
  \notready
  $\mathbb{E}[(X  - \mu)^{2n + 1}] = 0 $
\end{lemma}

\begin{lemma}\label{lem:centralMoment_fun_odd_semicircleReal}
  \lean{centralMoment_fun_odd_semicircleReal}
  \notready
   $\mathbb{E}[(X  - \mu)^{2n + 1}] = 0 $
\end{lemma}


%--- Haoyan's Part for Semicircle Distribution (End)---%

