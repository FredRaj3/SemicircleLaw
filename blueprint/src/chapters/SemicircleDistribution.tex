\chapter{SemicircleDistribution}

\section{SemicircleDistribution}

\begin{definition}
    \label{def:semicircle_distribution}
    \uses{def:semicirclePDF}
    \leanok
    The semicircle distribution with mean $\mu$ and variance $v$ is the Dirac delta at μ if v = 0; otherwise, it's the Lebesgue measure weighted by the semicircle PDF.
\end{definition}

\begin{lemma}
    \label{lemma:semicircleReal_of_var_ne_zero}
    \uses{def:semicircle_distribution}
    \leanok
    If $v \neq 0$, then the definition the semicircle distribution is defined as the Lebesgue measure weighted by the semicircle probability density function.
\end{lemma}

\begin{proof}
    Follows directly from definition of semicircle distribution.
\end{proof}

\begin{lemma}
    \label{lemma:semicircleReal_zero_var}
    \uses{def:semicircle_distribution}
    \leanok
    If the variance is 0, then the semicircle distribution is exactly the Dirac measure at $\mu$.
\end{lemma}

\begin{proof}
    Follows directly from definition of semicircle distribution.
\end{proof}

\begin{lemma} %instance: unsure if this should be a lemma or something else
    \label{lemma:instIsProbabilityMeasuresemicircleReal}
    \uses{lemma:semicircleReal_of_var_ne_zero}
    \leanok
    The measure semicircleReal is a probability measure, no matter the values of $\mu \in \mathbb{R}$ and $v \in \mathbb{R}_{\ge 0}$.
\end{lemma}

%proof for instance? unsure

\begin{lemma}
    \label{lemma:noAtoms_semicircleReal}
    \uses{lemma:semicircleReal_of_var_ne_zero, lemma:instIsProbabilityMeasuresemicircleReal} %idk if this is right
    \leanok
    If the variance v is nonzero, then the semicircle distribution has no atoms.
\end{lemma}

\begin{lemma}
    \label{lemma:semicircleReal_apply}
    \uses{lemma:semicircleReal_of_var_ne_zero}
    \leanok
    For a semicircle measure with mean $\mu$ and nonzero variance v, the measure of any measurable set $s$ equals the Lebesgue integral over $s$ of the semicircle probability density function at x.
\end{lemma}

\begin{lemma}
    \label{lemma:semicircleReal_apply_eq_integral}
    \uses{lemma:semicircleReal_apply, lemma:integrable_semicirclePDFReal, lemma:semicirclePDFReal_nonneg}
    \leanok
    For any real mean $\mu$, and any nonnegative variance $v$ that is not zero, and any measurable set $s$ of real numbers, the semicircle distribution measure of the set $s$ equals the extended nonnegative real number version (ENNReal.ofReal) of the integral of the semicircle probability density function over s.
\end{lemma}

\begin{lemma}
    \label{lemma:semicircleReal_absolutelyContinuous}
    \uses{lemma:semicircleReal_of_var_ne_zero}
    \leanok
    For a semicircle distribution with mean $\mu$ and nonzero variance $v$, the measure semicircleReal $μ$ $v$ is absolutely continuous with respect to the Lebesgue measure.
\end{lemma}

\begin{lemma}
    \label{lemma:rnDeriv_semicircleReal}
    \uses{lemma:semicircleReal_zero_var, lemma:semicirclePDF_zero_var, lemma:semicircleReal_of_var_ne_zero, lemma:measurable_semicirclePDF}
    \leanok
    The Radon–Nikodym derivative of the semicircle measure semicircleReal $\mu$ $v$ with respect to the Lebesgue measure is almost everywhere equal to the semicircle probability density function semicirclePDF $\mu$ $v$.
\end{lemma}

\begin{lemma}
    \label{lemma:integral_semicircleReal_eq_integral_smul}
    \uses{def:semicircleReal, lemma:measurable_semicirclePDF, lemma:semicirclePDF_lt_top}
    \leanok

    Let $ f : \mathbb{R} \to E $ be a function where $ E $ is a normed vector space over $\mathbb{R}$. For the semicircle distribution with mean $\mu \in \mathbb{R}$ and variance $ v > 0 $, we have:

    \[
    \int f(x) \, d(\mathrm{semicircleReal}\ \mu\, v)(x) = \int \mathrm{semicirclePDFReal}(\mu, v, x) \cdot f(x) \, dx.
    \]
    
    % In other words, integrating $ f $ with respect to the semicircle measure is equivalent to integrating the product of $ f $ and the semicircle probability density function with respect to Lebesgue measure.
    
\end{lemma}

