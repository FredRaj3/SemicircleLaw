\chapter{Weak Convergence}

%---Richard's Part for Section 4.3 (Start)---% 
%---------%
%---------%
\begin{lemma}[R-3-1 : lem:weak\_convergence]
  \notready
  \label{lem:weak_convergence}
  Let $k \in \mathbb{N}$ and $\epsilon > 0$. Then for any $b > 4$,
  \[
  \underset{n \rightarrow \infty}{\lim \sup} \mathbb{P} \biggl( \int_{|x| > b} |x|^k \mu_{\mathbf{X}_n}(dx) > \epsilon \biggl) = 0.
  \]
\end{lemma}
%---------%
%---------%
\begin{lemma}[R-3-2 : lem:R-3-2]
  \notready
  \label{lem:R-3-2}
  \uses{}
  \[
  \mathbb{P} \biggl( \int_{|x| > b} |x|^k \mu_{\mathbf{X}_n}(dx) > \epsilon \biggl) \leq \frac{1}{\epsilon} \mathbb{E} \biggl( \int_{|x| > b} |x|^k \mu_{\mathbf{X}_n}(dx) \biggl).
  \]
\end{lemma}
\begin{proof}
  This follows from Markov's inequality.
\end{proof}
%---------%
%---------%
\begin{definition}[R-3-3 : def:R-3-3]
  \notready
  \label{def:R-3-3}
  \uses{} % Definition of the ESD.
  We define $\nu$ to be the random measure $\nu(dx) = |x|^k \mu_{\mathbf{X}_n}(dx)$.
\end{definition}
%---------%
%---------%
\begin{lemma}[R-3-4 : lem:R-3-4]
  \notready
  \label{lem:R-3-4}
  \uses{def:R-3-3} % Definition of $\nu$.
  \[
  \int_{|x| > b} |x|^k \mu_{\mathbf{X}_n}(dx) = \nu\{x : |x|^k > b^k\}.
  \]
\end{lemma}
\begin{proof}
  \[
  \int_{|x| > b} |x|^k \mu_{\mathbf{X}_n}(dx) = \nu\{x : |x| > b\} = \nu\{x : |x|^k > b^k\}.
  \]
\end{proof}
%---------%
%---------%
\begin{lemma}[R-3-5 : lem:R-3-5]
  \notready
  \label{lem:R-3-5}
  \uses{lem:R-3-4,def:R-3-3} % Previous result & definition of $\nu$.
  \[
  \int_{|x| > b} |x|^k \mu_{\mathbf{X}_n}(dx) \leq \frac{1}{b^k} \int |x|^{2k} \mu_{\mathbf{X}_n}(dx).
  \]
\end{lemma}
\begin{proof}
  First, applying Markov's inequality on Lemma \ref{lem:R-3-4} gives
  \[
  \int_{|x| > b} |x|^k \mu_{\mathbf{X}_n}(dx) \leq \frac{1}{b^k} \int |x|^k \nu(dx).
  \] 
  Next, `substituting' $\nu(dx)$ with $|x|^k \mu_{\mathbf{X}_n}(dx)$ gives
  \[
  \int_{|x| > b} |x|^k \mu_{\mathbf{X}_n}(dx) \leq \frac{1}{b^k} \int |x|^{2k} \mu_{\mathbf{X}_n}(dx).
  \] 
\end{proof}
%---------%
%---------%
\begin{lemma}[R-3-6 : lem:R-3-6]
  \notready
  \label{lem:R-3-6}
  \uses{lem:R-3-5,lem:R-3-2} % Previous result & first equation.
  \[
  \mathbb{P} \biggl( \int_{|x| > b} |x|^k \mu_{\mathbf{X}_n}(dx) > \epsilon \biggl) \leq \frac{1}{\epsilon b^k} \mathbb{E} \biggl( \int |x|^{2k} \mu_{\mathbf{X}_n}(dx) \biggl).
  \]
\end{lemma}
\begin{proof}
  Substituting the term inside the expectation of Lemma \ref{lem:R-3-2} with the expression acquired from Lemma \ref{lem:R-3-5} gives
  \[
  \mathbb{P} \biggl( \int_{|x| > b} |x|^k \mu_{\mathbf{X}_n}(dx) > \epsilon \biggl) \leq \frac{1}{\epsilon} \mathbb{E} \biggl( \int_{|x| > b} |x|^k \mu_{\mathbf{X}_n}(dx) \biggl) \leq \frac{1}{\epsilon b^k} \mathbb{E} \biggl( \int |x|^{2k} \mu_{\mathbf{X}_n}(dx) \biggl).
  \]
\end{proof}
%---------%
%---------%
\begin{lemma}[R-3-7 : lem:R-3-7]
  \notready
  \label{lem:R-3-7}
  \uses{lem:R-3-6} % Previous result & the spectral theorem argument used under Theorem 2.3.
  \[
  \mathbb{P} \biggl( \int_{|x| > b} |x|^k \mu_{\mathbf{X}_n}(dx) > \epsilon \biggl) \leq \frac{1}{\epsilon b^k} \cdot \frac{1}{n} \mathbb{E} \mathrm{Tr} (\mathbf{X}_n^k).
  \]
\end{lemma}
\begin{proof}
  Substituting the term inside the expectation of Lemma \ref{lem:R-3-6} with the expression acquired in Lemma \ref{} gives
  \[
  \mathbb{P} \biggl( \int_{|x| > b} |x|^k \mu_{\mathbf{X}_n}(dx) > \epsilon \biggl) \leq \frac{1}{\epsilon b^k} \mathbb{E} \biggl( \int |x|^{2k} \mu_{\mathbf{X}_n}(dx) \biggl) = \frac{1}{\epsilon b^k} \cdot \frac{1}{n} \mathbb{E} \mathrm{Tr} (\mathbf{X}_n^k).
  \]
\end{proof}
%---Richard's Part for Section 4.3 (End)---% 


%---Paul's Part for Section 4.3 (Start)---% 
\begin{lemma}[Catalan Number bound]
    \label{lem:Catalan_bound}
    \notready
    Let $C_{k}$ be the Catalan number. Then, for all $k \in \mathbb{N}$, we have:
    $$
    C_{k} \leq 4^{k}
    $$ 
\end{lemma}

\begin{lemma}[Bound for $\lim \sup_{n\to\infty}\frac{1}{n}\mathbb{E} \operatorname{Tr} (\mathbf{X}_{n}^{k})$]
    \label{lem:bound_for_expectation_trace}
    \uses{lem:Catalan_bound, prop:matrix_moments_convergence}
    \notready
    Let $\{Y_{ij}\}_{1 \leq i \leq j}$ be independent random variables, with $\{Y_{ii}\}_{i\geq 1}$ identically distributed and $\{Y_{ij}\}_{1 \leq i < j}$ identically distributed. Suppose that $r_k = \max\{\bE(|Y_{11}|^k),\bE(|Y_{12}|^k)\} < \infty$ for each $k\in\bN$. Suppose further than $\bE(Y_{ij})=0$ for all $i,j$. If $i>j$, define $Y_{ij} \equiv Y_{ji}$, and let $\mathbf{Y}_n$ be the $n\times n$ matrix with $[\mathbf{Y}_n]_{ij} = Y_{ij}$ for $1\le i,j\le n$. Let $\mathbf{X}_n = n^{-1/2}\mathbf{Y}_n$ be the corresponding Wigner matrix. Then, we have:
    $$
    \lim \sup_{n\to\infty} \frac{1}{n}\mathbb{E}\operatorname{Tr} (\mathbf{X}_{n}^{k}) \leq 4^{k}
    $$ 
\end{lemma}
\begin{proof}
    From proposition \ref{prop:matrix_moments_convergence}, we know that $\lim_{n\to\infty} \frac{1}{n}\mathbb{E}\operatorname{Tr}(\mathbf{X}_{n}^{k}) \leq C_{k}$ (odd and even cases of $k$). We further know that since the limit exists, $\lim \sup_{n\to\infty} \frac{1}{n}\mathbb{E}\operatorname{Tr}(\mathbf{X}_{n}^{k}) = \lim_{n\to\infty} \frac{1}{n}\mathbb{E}\operatorname{Tr}(\mathbf{X}_{n}^{k})$. Using lemma \ref{lem:Catalan_bound}, we have:
     $$
     \lim \sup_{n\to\infty }\mathbb{E}\frac{1}{n}\operatorname{Tr}(\mathbf{X}_{n}^{k}) \leq C_{k} \leq 4^{k}
    $$ 
    as required.
\end{proof}

\begin{lemma}[New bound for $\lim \sup_{n\to\infty} \mathbb{P}(\int_{|x| > b}|x|^{k} \mu \mathbf{x}_{n}(dx))$]
    \label{lem:new_bound_for_lim_sup_p}
    \uses{lem:bound_for_expectation_trace, lem:R-3-7}
    \notready
    let $k \in \mathbb{N}$ and  $\epsilon > 0$. then for any  $b > 4$, 
     $$
    \lim \sup_{n\to\infty} \mathbb{P}(\int_{|x| > b}|x|^{k} \mu \mathbf{x}_{n}(dx)) \leq \frac{1}{\epsilon}(\frac{4}{b})^{k}
    $$ 
\end{lemma}

\begin{lemma}[Increasing Sequence $|x|^{k}$]
    \label{lem:increasing_sequence_x_to_k}
    \notready

    for $x > b > 4 > 1 \in \mathbb{R}$, we have $k \mapsto |x|^{k}$ is increasing with $k$.
\end{lemma}

\begin{lemma}[Increasing Sequence $\mathbb{P}(\int_{|x| > b}|x|^{k}\mu \mathbf{x}_{n}(dx))$]
    \label{lem:increasing_sequence_p_of_int_x_to_k}
    \notready

    for $x > b > 4 > 1 \in \mathbb{R}$, we have $k \mapsto \mathbb{P} (\int_{|x| > b}|x|^{k}\mu \mathbf{x}_{n} (dx))$ is increasing with $k$.
\end{lemma}

\begin{lemma}[Increasing Sequence $\lim \sup_{n\to\infty}\mathbb{P}(\int_{|x| > b}|x|^{k}\mu \mathbf{x}_{n}(dx))$]
    \label{lem:increasing_sequence_lim_sup_p}
    \notready

    for $x > b > 4 > 1 \in \mathbb{R}$, we have $k \mapsto \lim\sup_{n\to\infty}\mathbb{P} (\int_{|x| > b}|x|^{k}\mu \mathbf{x}_{n} (dx))$ is increasing with $k$.
\end{lemma}

\begin{lemma}[Nonnegative Sequence $\lim \sup_{n\to\infty}\mathbb{P}(\int_{|x| > b}|x|^{k}\mu \mathbf{x}_{n}(dx))$]
    \label{lem:nonnegative_sequence_lim_sup_p}
    \notready

    for $x > b > 4 > 1 \in \mathbb{R}$, we have $k \mapsto \lim\sup_{n\to\infty}\mathbb{P} (\int_{|x| > b}|x|^{k}\mu \mathbf{x}_{n} (dx)) \geq 0$ for all $k \in \mathbb{N}$. 
\end{lemma}

\begin{lemma}[Decreasing Sequence $\frac{1}{\epsilon}(\frac{4}{b})^{k}$]
    \label{lem:decreasing_sequence_1_over_epsilon_4_over_b_to_k}
    \notready

    for $x > b > 4 > 1 \in \mathbb{R}$, we have $k \mapsto \frac{1}{\epsilon}(\frac{4}{b})^{k}$ is decreasing to $0$ as $k\to\infty$.
\end{lemma}

\begin{lemma}[Limit of Sequence $\lim \sup_{n\to\infty}\mathbb{P}(\int_{|x| > b}|x|^{k}\mu \mathbf{x}_{n}(dx))$]
    \label{lem:limit_of_sequence_lim_sup_p}
    \uses{lem:nonnegative_sequence_lim_sup_p, lem:decreasing_sequence_1_over_epsilon_4_over_b_to_k, lem:new_bound_for_lim_sup_p}
    \notready

    for $x > b > 4 > 1 \in \mathbb{R}$, the sequence $k \mapsto \lim\sup_{n\to\infty}\mathbb{P} (\int_{|x| > b}|x|^{k}\mu \mathbf{x}_{n} (dx))$ has:
    $$
    \lim_{k\to\infty} \lim\sup_{n\to\infty}\mathbb{P}(\int_{|x| > b}|x|^{k}\mu \mathbf{x}_{n}(dx)) = 0
    $$ 
\end{lemma}

\begin{proof}
    We know that for all $k \in \mathbb{N}$, $\lim\sup_{n\to\infty}\mathbb{P}(\int_{|x| > b}|x|^{k}\mu \mathbf{x}_{n}(dx)) \geq 0$ (lemma \ref{lem:nonnegative_sequence_lim_sup_p}). We also see that the sequence is bounded by $\frac{1}{\epsilon} (\frac{4}{b})^{k}$ (lemma \ref{lem:new_bound_for_lim_sup_p}), which is decreasing to $0$ as $k\to\infty$ (lemma \ref{lem:decreasing_sequence_1_over_epsilon_4_over_b_to_k}). Via squeeze theorem (where we use the zero sequence as another bound), we have:
    $$
    \lim_{k\to\infty} \lim\sup_{n\to\infty}\mathbb{P}(\int_{|x| > b}|x|^{k}\mu \mathbf{x}_{n}(dx)) = 0
    $$ 
    as required.
\end{proof}

\begin{lemma}[All terms of Sequence are Zero conditions]
    \label{lem:all_terms_zero_conditions}
    \notready
    for any sequence $(a_{k})_{k=1}^{\infty} \in \mathbb{R}$, if $a_{k}$ has:
    \begin{enumerate}
        \item for all $k \in \mathbb{N}$,  $a_{k} \geq 0$ (nonnegative sequence)
        \item for all $k \in \mathbb{N}$, we have  $a_{k+1} \geq a_{k}$ (strictly increasing sequence)
        \item $\lim_{k\to\infty} a_{k} = 0$

        then for all $k \in \mathbb{N}$, we have $a_{k} = 0$
    \end{enumerate}
\end{lemma}

\begin{lemma}[Lemma 4.7 from \cite{Kemp2013RMTNotes}]
    \label{lem:convergence_to_zero_of_lim_sup_p}
    \uses{lem:increasing_sequence_lim_sup_p, lem:all_terms_zero_conditions, lem:limit_of_sequence_lim_sup_p, lem:nonnegative_sequence_lim_sup_p}
    \notready
    let $k \in \mathbb{N}$ and  $\epsilon > 0$. Then for any  $b > 4$,
    $$
    \lim \sup_{n\to\infty}\mathbb{P}(\int_{|x| > b}|x|^{k} \mu\mathbf{x}_{n}(dx)) = 0
    $$
\end{lemma}

\begin{proof}
    Consider sequence $(a_{k})_{k =1}^{\infty}$ defined as $a_{k} = \lim \sup_{n\to\infty}\mathbb{P}(\int_{|x| > b}|x|^{k} \mu\mathbf{x}_{n}(dx))$. Then, we use the conditions from lemma \ref{lem:all_terms_zero_conditions}:
    \begin{enumerate}
        \item From lemma \ref{lem:nonnegative_sequence_lim_sup_p}, we know that $a_{k} \geq 0$ for all $k \in \mathbb{N}$, so the sequence is nonnegative.
        \item From lemma \ref{lem:increasing_sequence_lim_sup_p}, we know that $a_{k+1} \geq a_{k}$ for all $k \in \mathbb{N}$, so the sequence is strictly increasing.
        \item From lemma \ref{lem:limit_of_sequence_lim_sup_p}, we know that $\lim_{k\to\infty} a_{k} = 0$, so the sequence converges to zero.
    \end{enumerate}
    Therefore the sequence $a_{k} = \lim \sup_{n\to\infty}\mathbb{P}(\int_{|x| > b}|x|^{k} \mu\mathbf{x}_{n}(dx)) = 0$ for all $k \in \mathbb{N}$, as required.
\end{proof}

\begin{lemma}[Function from Weierstrass Approx Theorem]
    \label{lem:function_weierstrass_approx}
    \notready
    Fix a bounded, continuous function $f \in C_{b}(\mathbb{R})$, fix $\epsilon > 0$, fix $b > 4$. Then, there exists a polynomial  $P_{\epsilon}$ such that:
    $$
    \sup_{|x| \leq b} |f(x) - P_{\epsilon}(x)| < \frac{\epsilon}{6}
    $$ 
\end{lemma}

\begin{lemma}[Estimates from Triangle Inequality]
    \label{lem:estimates_from_triangle_inequality}
    \uses{lem:function_weierstrass_approx}
    \notready
    Let $f \in C_{b}(\mathbb{R})$, fix $\epsilon > 0$, and  $b > 4$, and let  $P_{\epsilon}$ be the polynomial from lemma \ref{lem:function_weierstrass_approx}. Then, we have:
    $$
    \left|\int f d\mu_{\mathbf{x}_{n}} - \int f d\sigma_{1}\right| \leq \left|\int f d\mu_{\mathbf{x}_{n}} - \int P_{\epsilon} d\mu_{\mathbf{x}_{n}}\right| + \left|\int P_{\epsilon} d\mu_{\mathbf{x}_{n}} - \int P_{\epsilon} d\sigma_{1}\right| + \left|\int P_{\epsilon} d\sigma_1 - \int f d\sigma_1\right|
    $$ 
\end{lemma}

%------- Paul's part ends


%------- Kiran's part begins

\begin{lemma}
  \label{lem:f_pigeonhole}
  \uses{lem:estimates_from_triangle_inequality} 
  \notready
  If $\left|\int f\,d\mu_{X_n} - \int f\,d\sigma_1\right|>\epsilon$, then  $\left|\int f\,d\mu_{X_n} - \int P_\epsilon\,d\mu_{X_n}\right|>\epsilon/3, 
    \left|\int P_\epsilon\,d\mu_{X_n} - \int P_\epsilon\,d\sigma_1\right| >\epsilon/3,$ or $\left|\int P_\epsilon\,d\sigma_1 - \int f\,d\sigma_1\right|>\epsilon/3$.
\end{lemma}

\begin{proof}
  \notready
  Pigeonhole.
\end{proof}



\begin{lemma}
  \label{lem:f_probability_inequality}
  \uses{lem:f_pigeonhole, lem:estimates_from_triangle_inequality} 
  \notready
  $\bP\left( \left|\int f\,d\mu_{X_n} - \int f\,d\sigma_1\right|>\epsilon\right) \le \bP\left(\left|\int f\,d\mu_{X_n} - \int P_\epsilon\,d\mu_{X_n}\right|>\epsilon/3\right)
    + \bP\left(\left|\int P_\epsilon\,d\mu_{X_n} - \int P_\epsilon\,d\sigma_1\right| >\epsilon/3\right) 
    + \bP\left(\left|\int P_\epsilon\,d\sigma_1 - \int f\,d\sigma_1\right|>\epsilon/3\right)$
\end{lemma}

\begin{proof}
  \notready
  Triangle equality: use MeasureTheory.lintegral\_edist\_triangle. 
  (Not sure if this is needed)
\end{proof}


\begin{lemma}
  \label{lem:p_epsilon_minus_f_d_sigma_eq_zero}
  \uses{lem:function_weierstrass_approx, lem:estimates_from_triangle_inequality} % sup from wat, def b
  \notready
  $\bP\left(\left|\int P_\epsilon\,d\sigma_1 - \int f\,d\sigma_1\right|>\epsilon/3\right)$ is identically zero.
\end{lemma}

\begin{proof}
  \notready
  By construction, $|P_\epsilon-f|<\epsilon/6$ on $[-b,b]$, which includes the support $[-2,2]$ of $\sigma_1$.
\end{proof}

\begin{lemma}
  \label{lem:f_p_epsilon_inequality}
  \uses{}
  \notready
  \[ \left|\int (f-P_\epsilon)\,d\mu_{X_n}\right| \le \int_{|x|\le b} |f(x)-P_\epsilon(x)|\,\mu_{X_n}(dx) + \int_{|x|>b} |f(x)-P_\epsilon(x)|\,\mu_{X_n}(dx). \]
\end{lemma}

\begin{proof}
  \notready
  Break up the integral.
\end{proof}


\begin{lemma}
  \label{lem:f_p_epsilon_pigeonhole}
  \uses{}
  \notready
  If $\left|\int f\,d\mu_{X_n} - \int P_\epsilon\,d\mu_{X_n}\right|>\epsilon/3$, then $\int |f-P_\epsilon|\mathbb{1}_{|x|\le b}\,d\mu_{X_n}>\epsilon/6$ or $\int |f-P_\epsilon|\mathbb{1}_{|x|> b}\,d\mu_{X_n}>\epsilon/6$.
  
\end{lemma}

\begin{proof}
  \notready
  Pigeonhole.
\end{proof}


\begin{lemma}
  \label{lem:f_p_epsilon_probability_inequality}
  \uses{lem:f_p_epsilon_inequality, lem:f_p_epsilon_pigeonhole}
  \notready
  \[
    \bP\left(\left|\int f\,d\mu_{X_n} - \int P_\epsilon\,d\mu_{X_n}\right|>\epsilon/3\right) \le \bP\left(\int |f-P_\epsilon|\mathbb{1}_{|x|\le b}\,d\mu_{X_n}>\epsilon/6\right)
  + \bP\left(\int |f-P_\epsilon|\mathbb{1}_{|x|> b}\,d\mu_{X_n}>\epsilon/6\right).
  \]
  
\end{lemma}

\begin{proof}
  \notready
  Use previous two lemmas.
\end{proof}

%------- Kiran's part ends


%--------Haoyan's part-----------------------

\begin{lemma}
    \label{lem:first_term_of_first_term_zero}
    \uses{lem:f_p_epsilon_probability_inequality} %also uses |f - P| < \epsilon /6 on [-b, b]  and first term expansion
    \notready
    $  \limsup_{n\to\infty} \left(\mathbb{P}\left(\int |f-P_\epsilon|\mathbb{1}_{|x|\le b}\,d\mu_{\mathrm{X}_n}>\epsilon/6\right) \right) = 0$
\end{lemma}


% \begin{lemma}
%     \label{lem:first_term_estimate}
%     \uses{lem:first_term_of_first_term_zero}
%     \notready
%     $\mathbb{P}\left(\left|\int f\,d\mu_{\mathrm{X}_n} - \int P_\epsilon\,d\mu_{\mathrm{X}_n}\right|>\epsilon/3\right) \le \mathbb{P}\left(\int |f-P_\epsilon|\mathbb{1}_{|x|> b}\,d\mu_{\mathrm{X}_n}>\epsilon/6\right)$
% \end{lemma}

% \begin{lemma}
%     \label{lem:last_term_estimate}
%     \uses{} %\uses |f - P| <.....
%     \notready
%     $ \limsup_{n\to\infty} \left(\mathbb{P}\left(\int |f-P_\e|\1_{|x|> b}\,d\mu_{\mathrm{X}_n}>\e/6\right) \right) = 0$
% \end{lemma}


\begin{lemma}
    \label{lem:reestimate}
    \uses{lem:first_term_of_first_term_zero, lem:p_epsilon_minus_f_d_sigma_eq_zero, lem:f_probability_inequality}     %also requires last term being zero, requires the lem for the three terms....
    \notready
     \begin{align*} \mathbb{P}\left( \left|\int f\,d\mu_{\mathrm{X}_n} - \int f\,d\sigma_1\right|>\epsilon\right) &\le \mathbb{P}\left(\int |f-P_\epsilon|\mathbb{1}_{|x|> b}\,d\mu_{\mathrm{X}_n}>\epsilon/6\right) \\ 
     &+ \mathbb{P}\left(\left|\int P_\epsilon\,d\mu_{\mathrm{X}_n} - \int P_\epsilon\,d\sigma_1\right| >\epsilon/3\right).
     \end{align*}
\end{lemma}


\begin{lemma}
    \label{lem:second_term_estimate}
    \uses{prop:matrix_moments_convergence} %probably needs the prop 4.1 we proved, polynomial as test function thus converges
    \notready
    $ \limsup_{n \to \infty} \left(\mathbb{P}\left(\left|\int P_\epsilon\,d\mu_{\mathrm{X}_n} - \int P_\epsilon\,d\sigma_1\right| >\epsilon/3\right) \right) = 0 $
\end{lemma}



\begin{lemma}
    \label{lem:final_estimate}
    \uses{lem:reestimate, lem:second_term_estimate, lem:f_probability_inequality}  %need the estimate for the third term being zero,also linking it to the original inequality
    \notready
    $\mathbb{P}\left( \left|\int f\,d\mu_{\mathrm{X}_n} - \int f\,d\sigma_1\right|>\epsilon\right) \leq \mathbb{P}\left(\int |f-P_\epsilon|\mathbb{1}_{|x|> b}\,d\mu_{\mathrm{X}_n}>\epsilon/6\right)$
\end{lemma}



\begin{lemma}  
    \label{lem:polynomial_ineq}
    \uses{}
    \notready
    $ \mathbb{P}\left(\int |f-P_\epsilon| \mathbb{1}_{|x|\ge b}\,d\mu_{\mathrm{X}_n} > \epsilon/6\right) \le \mathbb{P}\left(\int c|x|^k\mathbb{1}_{|x|\ge b}\,\mu_{\mathrm{X}_n}(dx) > \epsilon/6\right) $
    \begin{proof}
    Let $k =$deg$P_\epsilon$, and since $f$ is bounded, $|f(x) - P_{\epsilon}(x)| \leq \|f\|_{\infty} + |P_\epsilon(x)|$. Also note it is on interval $|x| > b$, which completes the proof.
    \end{proof}
\end{lemma}


\begin{lemma}
    \label{lem:fP_bound}
    \uses{lem:convergence_to_zero_of_lim_sup_p, lem:polynomial_ineq}   %need the one for lemma 4.7
    \notready
    $\limsup_{n \to \infty}\mathbb{P}\left(\int c|x|^k\mathbb{1}_{|x|\ge b}\,\mu_{\mathrm{X}_n}(dx) > \epsilon/6\right) = 0 $
\end{lemma}


\begin{lemma}
    \label{lem:fP_zero}
    \uses{lem:fP_bound}
    \notready
    $\limsup_{n \to \infty}  \mathbb{P}\left(\int |f-P_\epsilon| \mathbb{1}_{|x|\ge b}\,d\mu_{\mathrm{X}_n} > \epsilon/6\right) = 0$
\end{lemma}



\begin{proposition}
  \label{prop:Wigner_Semicircle_Law } %prop 2.3
  \uses{lem:fP_zero, lem:final_estimate }
  \notready
  Let $\mathrm{X}_n=n^{-1/2}\mathrm{Y}_n$ be a sequence of Wigner matrices, with entries satisfying $\mathbb{E}(Y_{ij})=0$ for all $i,j$ and $\mathbb{E}(Y_{12}^2)=t$.  Then the empirical law of eigenvalues $\mu_{\mathrm{X}_n}$ converges in probability to $\sigma_t$ as $n\to\infty$.  Precisely: for any $f\in C_b(\mathbb{R})$ (continuous bounded functions) and each $\epsilon>0$,
  $$ \lim_{n\to\infty} \mathbb{P}\left(\left|\int f\,d\mu_{\mathrm{X}_n} - \int f\,d\sigma_t\right|>\epsilon\right)=0. $$
\end{proposition}

