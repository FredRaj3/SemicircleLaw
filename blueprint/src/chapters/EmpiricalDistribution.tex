\chapter{Empirical Distributions}

In this section, we let $\mathcal{X}$ be a Polish space. The topology on $\mathcal{X}$ will be generated by a metric $d$,
and we equip $\mathcal{X}$ with its Borel $\sigma$-algebra $\mathcal{B}(\mathcal{X})$. Further $(\Omega, \mathcal{F})$
will be a general measurable space (we do not fix a measure on this space), and $n$ will denote an arbitrary natural number.
We will also sometimes refer to a second Polish space $\mathcal{Y}$ with associated metric $\tilde d$, also equipped with
its Borel $\sigma$-algebra. The space of measures on $\mathcal{X}$ will be denoted $\mathcal{M}(\mathcal{X})$, and the
space of probability measures on $\mathcal{X}$ will be denoted $\mathcal{P}(\mathcal{X})$. We endow $\mathcal{M}(\mathcal{X})$
with the $\sigma$-algebra generated by the projection maps $\{\pi_{A} : \mathcal{M}(\mathcal{X}) \to \bR\}_{A \in \mathcal{B}(\mathcal{X})}$,
and similarly for $\mathcal{P}(\mathcal{X})$.

We will also consider the space of multisets of elements in $\mathcal{X}$. This will be given the disjoint union topology, where
the union is over the $n$th symmetric power $\mathrm{Sym}^n \mathcal{X}$ for $n \in \bN$. The $n$th symmetric power has the quotient
topology inherited from $\mathcal{X}^n$



\begin{definition}
  \label{def:empiricalDistribution}
  \uses{}
  \notready
  Let $S$ be a multiset of elements in $\mathcal{X}$ with finite cardinality. Then the associated empirical distribution is the probability measure on $\mathcal{X}$ defined by
  \[
  {\sf ED}(S) = \frac{1}{|S|}\sum_{x \in S} \delta_{x}.
  \]
\end{definition}

\begin{lemma}
  \label{lem:dirac_prob_measure_measurable_map}
  \uses{}
  \notready
  Let $(\mathcal{X}, d)$ be a Polish space, and equip it with its Borel $\sigma$-algebra. Then the map $\delta : \mathcal{X} \to \mathcal{M}(\mathcal{X})$, $\delta : x \mapsto \delta_x$, is measurable.
\end{lemma}

\begin{proof}
  \uses{}
  \notready
\end{proof}


\begin{lemma}
  \label{lem:sum_of_dirac_measures_measurable}
  \uses{lem:dirac_prob_measure_measurable_map}
  \notready
  Let $x_1, \dots, x_n \in \mathcal{X}$. For $i= 1, \dots, n$, suppose $a_i \geq 0$. Then the map
  \[
  (x_1, \dots, x_n) \mapsto \sum_{i=1}^n a_i \delta_{x_i}
  \]
  is a measurable map from $\mathcal{X}^n$ to $\mathcal{M}(\mathcal{X})$, where $\mathcal{X}^n$ is equipped with the product $\sigma$-algebra.
\end{lemma}

\begin{proof}
  \uses{}
  \notready
\end{proof}

\begin{lemma}
  \label{lem:empiricalDistribution_map_measurable}
  \uses{def:empiricalDistribution, lem:sum_of_dirac_measures_measurable}
  \notready
  The map ${\sf ED}_n :\mathcal{X}^n \to \mathcal{P}(\mathcal{X})$ is measurable, where $\mathcal{X}^n$ is equipped with the product $\sigma$-algebra.
\end{lemma}

\begin{proof}
  \uses{}
  \notready
\end{proof}


\begin{lemma}[Convex combination of probability measures is a probability measure]
  \label{lem:convex_combination_prob_measure_prob_measure}
  \uses{}
  \notready
  Let $\mu_1, \dots, \mu_n$ be probabilty measures over a measurable space $(\Omega, \mathcal{F})$. For $i = 1, \dots, n$, suppose $a_i \in [0,1]$ satisfy $\sum_{i=1}^{n} a_i = 1$. Then the convex combination
  \[
  \sum_{i=1}^n a_i \mu_i
  \]
  is a probability measure on $(\Omega, \mathcal{F})$.
\end{lemma}

\begin{proof}
  \uses{}
  \notready
\end{proof}

\begin{lemma}
  \label{lem:empiricalDistribution_probability_measure}
  \uses{lem:convex_combination_prob_measure_prob_measure}
  \notready
  The empirical distribution associated to $(x_1, \dots, x_n)$ is a probability measure.
\end{lemma}



\begin{lemma}
  \label{lem:random_dist_test_function_measurable}
  \uses{lem:dirac_prob_measure_measurable_map}
  \notready
  Let $\mu : \Omega \to \mathcal{P}(\mathcal{X})$ be measurable. Let $f : \mathcal{X} \to \mathcal{Y}$ be measurable, and define $\varphi : \Omega \to \mathcal{Y}$ by
  \[
  \varphi(\omega) = \int_{\mathcal{X}}f(x) \, d\mu(\omega)(x).
  \]
  Then $\varphi$ is a measurable map from $\Omega \to \mathcal{Y}$.

\end{lemma}

\begin{proof}
  \uses{}
  \notready
\end{proof}


\begin{definition}[Empirical Probability Distribtion]
  \label{def:empiricalProbabilityDistribution}
  \uses{def:empiricalDistribution}
  \notready
  Let $X_1, \dots, X_n : \Omega \to \mathcal{X}$ be measurable functions defined on a common measurable space $(\Omega, \mathcal{F})$. Then the empirical distribution associated to $X_1, \dots, X_n$ is the random probability measure on $\mathcal{X}$ given by
  \[
  {\sf ED}_{X_1, \dots, X_n}(\omega) = \frac{1}{n}\sum_{i=1}^n \delta_{X_i (\omega)}
  \]
\end{definition}

\begin{lemma}
  \label{lem:empiricalProbabilityDistribution_probability_meaure}
  \uses{def:empiricalProbabilityDistribution}
  \notready
  The empirical probability distribution associated to $X_1, \dots, X_n$ is a probability measure.
\end{lemma}

\begin{proof}
  \uses{lem:convex_combination_prob_measure_prob_measure}
  \notready
We need to check that for every $\omega \in Omega$, ${\sf ED}_{X_1, \dots, X_n}(\omega)(\mathcal{X}) = 1$. Note that for each $\omega$, $\delta_{X_i (\omega)}$ is a probability measure. Thus by Lemma~\ref{lem:convex_combination_prob_measure_prob_measure}, so is $\frac{1}{n}\sum_{i=1}^n \delta_{X_i (\omega)}$.
\end{proof}

\begin{lemma}
  \label{lem:empiricalProbabilityDistribution_map_measurable}
  \uses{def:empiricalProbabilityDistribution, lem:sum_of_dirac_measures_measurable, lem:empiricalDistribution_map_measurable}
  \notready
  Let $X_1, \dots, X_n : \Omega \to \mathcal{X}$ be measurable functions. Then the map ${\sf ED}_{X_1, \dots, X_n} : \Omega \to \mathcal{P}(\mathcal{X})$ is measurable.
\end{lemma}

\begin{proof}
  \uses{}
  \notready
\end{proof}


\begin{lemma}
  \label{lem:empiricalProbabilityDistribution_test_function_measurable}
  \uses{lem:empiricalProbabilityDistribution_map_measurable, def:empiricalProbabilityDistribution, lem:random_dist_test_function_measurable}
  \notready
  Let $X_1, \dots, X_n : \Omega \to \mathcal{X}$ be be measurable. Let $f : \mathcal{X} \to \mathcal{Y}$ be measurable, and define $\langle {\sf ED}_{X_1, \dots, X_n}, f \rangle: \Omega \to \mathcal{Y}$ by
  \[
  \langle {\sf ED}_{X_1, \dots, X_n}, f \rangle(\omega) = \int_{\mathcal{X}}f(x) \, d{\sf ED}_{X_1, \dots, X_n}(\omega)(x).
  \]
  Then $\langle {\sf ED}_{X_1, \dots, X_n}, f \rangle$ is a measurable map from $\Omega \to \mathcal{Y}$.
\end{lemma}

\begin{proof}
  \uses{}
  \notready
\end{proof}

\begin{definition}
  \label{def:empiricalCDF}
  \uses{def:empiricalProbabilityDistribution, lem:empiricalProbabilityDistribution_test_function_measurable}
  \notready
  Let $X_1, \dots, X_n$ be independent and indentically distributed according to some probability measure $\mu$. Then the empirical CDF is the random function $\hat F_n : \Omega \to \bR \to [0,1]$ defined by
  \[
  \hat F_n(\omega)(t) = \frac{1}{n}\sum_{i=1}^n \mathbb{1}_{X_i(\omega) \leq t}.
  \]
\end{definition}

\begin{theorem}[Glivenko-Cantelli Theorem]
  \label{thm:empiricalCDF_independent_variables_converges_CDF}
  \uses{def:empiricalCDF}
  \notready
Let $X_1, \dots, X_n$ be independent and indentically distributed according to some probability measure $\mu$. Let $F(t) = \bP (X_1 \leq t)$ be the CDF of the distribution $\mu$. Then
\[
\sup_{t \in \bR} |\hat F_n(t) - F(t)| \to 0
\]
with probability one.
\end{theorem}
