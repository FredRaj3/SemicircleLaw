
%section Transformation
\begin{lemma}\label{lem:semicircleReal_map_add_const}
  \uses{lem:integral_semicircleReal_eq_integral_smul}    %convert the statement into integral and use semicirclereal pdf...
  \notready
  \lean{semicircleReal_map_add_const}
  Given semicircular measure $\sigma(\mu, v)$ with mean $\mu$ and variance $v$, then for any constant $y \in \mathbb{R}$, the pushforward
  of $\sigma(\mu, v)$ under the map $x \mapsto x + y$ is again a semicircular measure $\sigma(\mu + y, v)$. 
  \begin{proof}
    %sorry
  \end{proof}
\end{lemma}


\begin{lemma}\label{lem:semicircleReal_map_const_add}
  \mathlibok
  \lean{semicircleReal_map_const_add}
  Given semicircular measure $\sigma(\mu, v)$ with mean $\mu$ and variance $v$, then for any constant $y \in \mathbb{R}$, the pushforward
  of $\sigma(\mu, v)$ under the map $x \mapsto y + x$ is again a semicircular measure $\sigma(\mu + y, v)$. 
  \begin{proof}
    Obvious from commutativity between $x + y$ and $y + x$.
  \end{proof}
\end{lemma}


\begin{lemma}\label{lem:semicircleReal_map_const_mul}
  \notready
  \lean{semicircleReal_map_const_mul}
  \uses{lem:integral_semicircleReal_eq_integral_smul}
    Given semicircular measure $\sigma(\mu, v)$ with mean $\mu$ and variance $v$, then for any constant $c \in \mathbb{R}$, the pushforward
  of $\sigma(\mu, v)$ under the map $x \mapsto c * x$ is again a semicircular measure $\sigma(c\mu, c^2v)$. 
  \begin{proof}
     %sorry
  \end{proof}
\end{lemma}



\begin{lemma}\label{lem:semicircleReal_map_mul_const}
  \mathlibok
  \lean{semicircleReal_map_mul_const}
  \uses{lem:semicircleReal_map_const_mul}
   Given semicircular measure $\sigma$ with mean $\mu$ and variance $v$, then for any constant $c \in \mathbb{R}$, the pushforward
  of $\sigma(\mu, v)$ under the map $x \mapsto x * c$ is again a semicircular measure $\sigma(c\mu, c^2v)$. 
  \begin{proof}
    Use commutativity between $Xc$ and $cX$.
  \end{proof}
\end{lemma}



\begin{lemma}\label{lem:semicircleReal_map_neg}
  \mathlibok
  \lean{semicircleReal_map_neg}
  \uses{lem:semicircleReal_map_const_mul}
  Given semicircular measure $\sigma(\mu, v)$ with mean $\mu$ and variance $v$, the pushforward
  of $\sigma(\mu, v)$ under the map $x \mapsto -x$ is again a semicircular measure $\sigma(- \mu, v)$. 
  \begin{proof}
     Special case of the multiplication by constant map with constant being $-1$.
  \end{proof}
\end{lemma}



\begin{lemma}\label{lem:semicircleReal_map_sub_const}
  \mathlibok
  \lean{semicircleReal_map_sub_const}
  \uses{lem:semicircleReal_map_add_const, lem:semicircleReal_map_neg}
  Given semicircular measure $\sigma(\mu, v)$ with mean $\mu$ and variance $v$, then for any constant $y \in \mathbb{R}$, the pushforward
  of $\sigma(\mu, v)$ under the map $x \mapsto x - y$ is again a semicircular measure $\sigma( \mu - y, v)$. 
  \begin{proof}
   Use the map by addition of constant and substitute constant for its $-1$ multiple.
  \end{proof}
\end{lemma}


\begin{lemma}\label{lem:semicircleReal_map_const_sub}
  \mathlibok
  \lean{semicircleReal_map_const_sub}
  \uses{lem:semicircleReal_map_neg, lem:semicircleReal_map_const_add}
   Given semicircular measure $\sigma(\mu, v)$ with mean $\mu$ and variance $v$, then for any constant $y \in \mathbb{R}$, the pushforward
  of $\sigma(\mu, v)$ under the map $x \mapsto y - x$ is again a semicircular measure $\sigma(y - \mu, v)$. 
  \begin{proof}

  \end{proof}
\end{lemma}


\begin{lemma}\label{lem:semicircleReal_add_const}
  \mathlibok
  \lean{semicircleReal_add_const}
  \uses{lem:semicircleReal_map_add_const}
  Given a real random variable $X \sim \sigma(\mu, v)$
  then for a constant $y \in \mathbb{R}$, $X + y \sim \sigma(\mu + y, v)$
  \begin{proof}
  \end{proof}
\end{lemma}


\begin{lemma}\label{lem:semicircleReal_const_add}
  \mathlibok
  \lean{semicircleReal_const_add}
  \uses{lem:semicircleReal_add_const}
  Given a real random variable $X \sim \sigma(\mu, v)$
  then for a constant $y \in \mathbb{R}$, $y + X \sim \sigma(\mu + y, v)$
  \begin{proof}

  \end{proof}
\end{lemma}


\begin{lemma}\label{lem:semicircleReal_const_mul}
  \mathlibok
  \lean{semicircleReal_const_mul}
  \uses{lem:semicircleReal_map_const_mul}
  Given a real random variable $X \sim \sigma(\mu, v)$,
  then for a constant $c \in \mathbb{R}$, $cX \sim \sigma(c\mu , c^2v)$
  \begin{proof}

  \end{proof}
\end{lemma}


\begin{lemma}\label{lem:semicircleReal_mul_const}
  \mathlibok
  \lean{semicircleReal_mul_const}
  \uses{lem:semicircleReal_const_mul}
   Given a real random variable $X \sim \sigma(\mu, v)$,
  then for a constant $c \in \mathbb{R}$, $Xc \sim \sigma(c \mu  , c^2v)$
  \begin{proof}

  \end{proof}
\end{lemma}

%end Transformation


%moments

%convert the random variable into standard semicircle distribution which has 0 mean
\begin{lemma}\label{lem:integral_id_semicircleReal}
  \lean{integral_id_semicircleReal}
  \uses{lem:semicircleReal_add_const, lem:semicircleReal_const_add, lem:semicircleReal_const_mul, lem:semicircleReal_mul_const}
  \notready
  If $X \sim \sigma(\mu, v)$, then its expectation $$\mathbb{E}[X] = \int x d \sigma(\mu, v) = \mu$$
  \begin{proof}
   %sorry
  \end{proof}
\end{lemma}

%similar to above
\begin{lemma}\label{lem:variance_fun_id_semicircleReal}
  \lean{variance_fun_id_semicircleReal}
  \uses{lem:semicircleReal_add_const, lem:semicircleReal_const_add, lem:semicircleReal_const_mul, lem:semicircleReal_mul_const}
  \notready
  If $X \sim \sigma(\mu, v)$, then its variance $Var(X) = v$
  \begin{proof}
   %sorry
  \end{proof}
\end{lemma}


\begin{lemma}\label{lem:variance_id_semicircleReal}
  \lean{variance_id_semicircleReal}
  \uses{lem:variance_fun_id_semicircleReal}
  \mathlibok
  The variance of a real semicircle distribution with parameter $(\mu, v)$ is
  its variance parameter $v$
\end{lemma}


\begin{lemma}\label{lem:memLp_id_semicircleReal'}
  \lean{memLp_id_semicircleReal}
  \mathlibok
  \uses{lem:variance_id_semicircleReal}
  All the moments of a real semicircle distribution are finite. That is, the identity is in $L_p$ for
  all finite $p$
\end{lemma}

% probably need to reformulate the recursive definition of Catalan number into the closed form for the integration to be equal.
%or maybe simply use mathlib........
\begin{lemma}\label{lem:centralMoment_two_mul_semicircleReal}
  \lean{centralMoment_fun_two_mul_semicircleReal}
  \uses{def:Catalan_number, lem:integral_id_semicircleReal, lem:variance_fun_id_semicircleReal, lem:variance_id_semicircleReal, lem:semicircleReal_add_const, lem:semicircleReal_const_add,lem:semicircleReal_const_mul, lem:semicircleReal_mul_const}
  \notready
   $\mathbb{E}[(X  - \mu)^{2n}] = v^n C_n $
   \begin{proof}
    %sorry
   \end{proof}
\end{lemma}


\begin{lemma}\label{lem:centralMoment_fun_two_mul_semicircleReal}
  \lean{centralMoment_fun_two_mul_semicircleReal}
  \uses{lem:centralMoment_two_mul_semicircleReal}
  \notready
   $\mathbb{E}[(X  - \mu)^{2n}] = v^n C_n $
   \begin{proof}
    %sorry
   \end{proof}
\end{lemma}


\begin{lemma}\label{lem:centralMoment_odd_semicircleReal}
  \lean{centralMoment_odd_semicircleReal}
  \uses{lem:semicircleReal_add_const, lem:semicircleReal_const_add}
  \notready
  $\mathbb{E}[(X  - \mu)^{2n + 1}] = 0 $
  \begin{proof}
    %sorry
   \end{proof}
\end{lemma}


\begin{lemma}\label{lem:centralMoment_fun_odd_semicircleReal}
  \lean{centralMoment_fun_odd_semicircleReal}
  \uses{lem:centralMoment_odd_semicircleReal}
  \notready
   $\mathbb{E}[(X  - \mu)^{2n + 1}] = 0 $
   \begin{proof}
    %sorry
   \end{proof}
\end{lemma}

%Moments end
