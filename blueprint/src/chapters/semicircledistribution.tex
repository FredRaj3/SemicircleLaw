\chapter{SemicircleDistribution}


%---Richard's Part for Semicircle Distribution (Start)---% 
%---------%
%---------%
\begin{definition}\label{def:semicirclePDFReal}
  \mathlibok 
  \lean{semicirclePDFReal}
    The function $\mathrm{sc} : \mathbb{R} \times \mathbb{R}_{\geq 0} \times \mathbb{R} \rightarrow \mathbb{R}$ defined by
   \[
    \mathrm{sc}(\mu,v,x) 
    = \frac{1}{2πv} \sqrt{(4v - (x - μ)^2)_+}
   \]
   is called the probability density function (pdf) of the semicircle distribution.
\end{definition}
%---------%
%---------%
\begin{lemma}\label{lem:semicirclePDFReal_def}
  \mathlibok 
  \lean{semicirclePDFReal_def}
  \uses{def:semicirclePDFReal}
    Given a mean $\mu \in \mathbb{R}$ and a variance $v \in \mathbb{R}_{\geq 0}$, the pdf $\mathrm{sc} : \mathbb{R} \rightarrow \mathbb{R}$ 
    of the semicircle distribution with mean $\mu$ and variance $v$ is given by
  \[
    \mathrm{sc}(x) = 
    \frac{1}{2πv} \sqrt{(4v - (x - μ)^2)_+}.
  \]
\end{lemma}
%---------%
%---------%
\begin{lemma}\label{lem:semicirclePDFReal_zero_var}
    \mathlibok
    \lean{semicirclePDFReal_zero_var}
    \uses{def:semicirclePDFReal}
    If the variance $v$ is given to be zero, then the pdf of the semicircle distribution is the function that is identically zero.
%    In other words, for any mean $m \in \mathbb{R}$ and $x \in \mathbb{R}$, we have
%    \[
%    f(m,0,x) = 0.
%    \]
\end{lemma}
\begin{proof}
    By Definition \ref{def:semicirclePDFReal}, the square root of a nonpositive number is defined to be zero.
    Hence, the pdf with a zero variance must be the function that is identically zero.
\end{proof}
%---------%
%---------%
\begin{lemma}\label{lem:semicirclePDFReal_nonneg}
    \mathlibok
    \lean{semicirclePDFReal_nonneg}
    \uses{def:semicirclePDFReal}
    The pdf of the semicircle distribution is always nonnegative. 
\end{lemma}
\begin{proof}
   By Definition \ref{def:semicirclePDFReal}, the square root of a nonpositive number is defined to be zero. 
   Furthermore, the variance is always assumed to be nonnegative.
   Therefore, since the fractional term and the square root term are always nonnegative,
   we conclude the pdf is always nonnegative. 
\end{proof}
%---------%
%---------%
\begin{lemma}\label{lem:measurable_semicirclePDFReal}
    \lean{measurable_semicirclePDFReal}
    \uses{}
    \notready
    Given a mean $\mu \in \mathbb{R}$ and a variance $v \in \mathbb{R}_{\geq 0}$, the pdf $\mathrm{sc} : \mathbb{R} \rightarrow \mathbb{R}$ 
    of the semicircle distribution with mean $\mu$ and variance $v$ is measurable.
\end{lemma}
\begin{proof}
% sorry
\end{proof}
%---------%
%---------%
\begin{lemma}\label{lem:stronglyMeasurable_semicirclePDFReal}
    \mathlibok
    \lean{stronglyMeasurable_semicirclePDFReal}
    \uses{def:semicirclePDFReal,lem:measurable_semicirclePDFReal}
    Given a mean $\mu \in \mathbb{R}$ and a variance $v \in \mathbb{R}_{\geq 0}$, the pdf $\mathrm{sc} : \mathbb{R} \rightarrow \mathbb{R}$ 
    of the semicircle distribution with mean $\mu$ and variance $v$ is strongly measurable.
\end{lemma}
\begin{proof}
    By Lemma \ref{lem:measurable_semicirclePDFReal}, we know the pdf $\mathrm{sc}$ with fixed mean $\mu$ and variance $v$ is measurable.
    Since $\mathbb{R}$ is equipped with a second countable topology, the fact that $\mathrm{sc}$ with fixed mean $\mu$ and variance $v$ implies $\mathrm{sc}$ is strongly measurable.  
\end{proof}
%---------%
%---------%
\begin{lemma}\label{lem:integrable_semicirclePDFReal}
    \lean{integrable_semicirclePDFReal}
    \uses{def:semicirclePDFReal}
    \notready
    Given a mean $\mu \in \mathbb{R}$ and a variance $v \in \mathbb{R}_{\geq 0}$, the pdf $\mathrm{sc} : \mathbb{R} \rightarrow \mathbb{R}$ 
    of the semicircle distribution with mean $\mu$ and variance $v$ is integrable.
\end{lemma}
\begin{proof}
% sorry
\end{proof}
%---------%
%---------%
\begin{lemma}\label{lem:lintegral_semicirclePDFReal_eq_one}
    \lean{lintegral_semicirclePDFReal_eq_one}
    \uses{def:semicirclePDFReal}
    \notready
    Given a mean $\mu \in \mathbb{R}$ and a nonzero variance $v \in \mathbb{R}_{> 0}$, the lower Lebesgue integral of the p.d.f. $\mathrm{sc} : \mathbb{R} \rightarrow \mathbb{R}$ 
    of the semicircle distribution with mean $\mu$ and variance $v$ equals $1$.
\end{lemma}
\begin{proof}
% sorry
\end{proof}
%---------%
%---------%
\begin{lemma}\label{lem:integral_semicirclePDFReal_eq_one}
    \lean{integral_semicirclePDFReal_eq_one}
    \uses{def:semicirclePDFReal}
    \notready
    Given a mean $\mu \in \mathbb{R}$ and a nonzero variance $v \in \mathbb{R}_{> 0}$, the integral of the pdf $\mathrm{sc} : \mathbb{R} \rightarrow \mathbb{R}$ 
    of the semicircle distribution with mean $\mu$ and variance $v$ equals $1$.
\end{lemma}
\begin{proof}
% sorry
\end{proof}
%---------%
%---------%
\begin{lemma}\label{lem:semicirclePDFReal_sub}
    \mathlibok
    \lean{semicirclePDFReal_sub}
    \uses{def:semicirclePDFReal}
    For any pdf $\mathrm{sc} : \mathbb{R} \rightarrow \mathbb{R}$ 
    of the semicircle distribution, the following relation is satisfied:
    \[
    \mathrm{sc}(\mu,v,x-y) = \mathrm{sc}(\mu+y,v,x)
    \]
    for any $\mu \in \mathbb{R}$, $v \in \mathbb{R}_{\geq 0}$, and $x,y \in \mathbb{R}$. 
\end{lemma}
\begin{proof}
   Expanding Definition \ref{def:semicirclePDFReal} gives
   \[
   \mathrm{sc}(\mu,v,x-y) 
   = \frac{1}{2πv} \sqrt{\bigl( 4v - ( (x - y) - μ)^2 \bigl)_+} 
   = \frac{1}{2πv} \sqrt{\bigl( 4v - (x - (μ + y))^2 \bigl)_+}
   = \mathrm{sc}(\mu+y,v,x).
   \]
\end{proof}
%---------%
%---------%
\begin{lemma}\label{lem:semicirclePDFReal_add}
    \mathlibok
    \lean{semicirclePDFReal_add}
    \uses{def:semicirclePDFReal,lem:semicirclePDFReal_sub}
    For any pdf $\mathrm{sc} : \mathbb{R} \rightarrow \mathbb{R}$ 
    of the semicircle distribution, the following relation is satisfied:
    \[
    \mathrm{sc}(\mu,v,x+y) = \mathrm{sc}(\mu-y,v,x)
    \]
    for any $\mu \in \mathbb{R}$, $v \in \mathbb{R}_{\geq 0}$, and $x,y \in \mathbb{R}$. 
\end{lemma}
\begin{proof}
   Expanding Definition \ref{def:semicirclePDFReal} gives
   \[
   \mathrm{sc}(\mu,v,x+y) 
   = \frac{1}{2πv} \sqrt{\bigl( 4v - ( (x + y) - μ)^2 \bigl)_+} 
   = \frac{1}{2πv} \sqrt{\bigl( 4v - (x - (μ - y))^2 \bigl)_+}
   = \mathrm{sc}(\mu-y,v,x).
   \]
\end{proof}
%---------%
%---------%
\begin{lemma}\label{lem:semicirclePDFReal_inv_mul}
    \lean{semicirclePDFReal_inv_mul}
    \uses{def:semicirclePDFReal}
    \notready
    For any pdf $\mathrm{sc} : \mathbb{R} \rightarrow \mathbb{R}$ 
    of the semicircle distribution, the following relation is satisfied:
    \[
    \mathrm{sc}(\mu,v,c^{-1} x) = |c| \cdot \mathrm{sc}(c \mu,c^2 v,x)
    \]
    for any $\mu \in \mathbb{R}$, $v \in \mathbb{R}_{\geq 0}$, $x \in \mathbb{R}$, and nonzero $c \in \mathbb{R}$. 
\end{lemma}
\begin{proof}
% sorry
\end{proof}
%---------%
%---------%
\begin{lemma}\label{lem:semicirclePDFReal_mul}
    \mathlibok
    \lean{semicirclePDFReal_mul}
    \uses{def:semicirclePDFReal,lem:semicirclePDFReal_inv_mul}
    For any pdf $\mathrm{sc} : \mathbb{R} \rightarrow \mathbb{R}$ 
    of the semicircle distribution, the following relation is satisfied:
    \[
    \mathrm{sc}(\mu,v,cx) = |c^{-1}| \cdot \mathrm{sc}(c^{-1} \mu,c^{-2} v,x)
    \]
    for any $\mu \in \mathbb{R}$, $v \in \mathbb{R}_{\geq 0}$, $x \in \mathbb{R}$, and nonzero $c \in \mathbb{R}$. 
\end{lemma}
\begin{proof}
   Expanding Definition \ref{def:semicirclePDFReal} gives
   \[
   \mathrm{sc}(\mu,v,c x)
   = \frac{1}{2πv} \sqrt{\bigl( 4v - ( cx - μ)^2 \bigl)_+} 
   = \frac{1}{2πv} \sqrt{\bigl( 4v -  c^2(x -  c^{-1}μ)^2 \bigl)_+} 
   = |c^{-1}| \frac{1}{2π(c^{-2}v)} \sqrt{\bigl( 4(c^{-2}v)  - (x - c^{-1} \mu)^2 \bigl)_+}
   = |c^{-1}| \cdot \mathrm{sc}(c^{-1} \mu,c^{-2} v,x).
   \]
\end{proof}
%---------%
%---------%
%---Richard's Part for Semicircle Distribution (End)---% 



\iffalse
%---------%
%---------%
%\begin{lemma}\label{}
%    \mathlibok
%    \lean{}
%    \uses{}
%    
%\end{lemma}
%\begin{proof}
%
%\end{proof}
%---------%
%---------%
\fi