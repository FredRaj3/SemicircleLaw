\begin{proposition}[Proposition 4.1 in \cite{Kemp2013RMTNotes}]
  \label{prop:matrix_moments_convergence}
  \notready
  \uses{def:}
  Let $\{Y_{ij}\}_{1\le i\le j}$ be independent random variables, with $\{Y_{ii}\}_{i\ge 1}$ identically distributed and $\{Y_{ij}\}_{1\le i<j}$ identically distributed.  Suppose that $r_k = \max\{\bE(|Y_{11}|^k),\bE(|Y_{12}|^k)\} <\infty$ for each $k\in\bN$.  Suppose further than $\bE(Y_{ij})=0$ for all $i,j$ and set $t=\bE(Y_{12}^2)$.  If $i>j$, define $Y_{ij} \equiv Y_{ji}$, and let $\mathbf{Y}_n$ be the $n\times n$ matrix with $[\mathbf{Y}_n]_{ij} = Y_{ij}$ for $1\le i,j\le n$.  Let $\mathbf{X}_n = n^{-1/2}\mathbf{Y}_n$ be the corresponding Wigner matrix.  Then
\[
\lim_{n\to\infty} \frac{1}{n}\bE\Tr(\mathbf{X}_n^k) = \begin{cases}
  t^{k/2}C_{k/2}, & k\text{ even} \\
  0, & k\text{ odd}
\end{cases}.
\]
\end{proposition}

\begin{proof}
\notready
Proof
\end{proof}



\begin{lemma}\label{lem:trace_smul}
  \mathlibok 
  \lean{Matrix.trace_smul}
  Let $R$ be a ring with a monoid action from $\alpha$ (i.e., $\alpha$ acts distributively on $R$).
  For any scalar $r \in \alpha$ and any square matrix $A$ over $R$, the trace of the scalar
  multiple $r \cdot A$ equals the scalar multiple of the trace of $A$, i.e.,
  \[ \text{tr}(r \cdot A) = r \cdot \text{tr}(A). \]
  \end{lemma}

  \begin{proof}\leanok 
  \end{proof}




%---Richard's Part for Proposition 4.1 (Updated) (Start)---% 
%---------% 
%---------% 
\begin{definition}[Length $|w_\mathbf{i}|$ : R-1-1 : def:length\_of\_w\_i]
  \label{def:length_of_w_i}
  \uses{def:graph_path}
  % We use the definition of the walk $w_\mathbf{i}$.
  Given a path $w_\mathbf{i}$ generated by some $k$-index $\mathbf{i}$, we let $|w_\mathbf{i}|$ denote the length of $w_\mathbf{i}$. 
\end{definition}
%---------% 
%---------% 
%\begin{definition}[Edge Counting Function $w_\mathbf{i}(\cdot)$ : R-1-*]
  % Paul's section might already have this definition.
%  \label{def:edge_counting_function_w}
%  \uses{def:graph_multi_index,def:graph_path}
  % We use the definition of the walk $w_\mathbf{i}$ and also the definition of the edge set $E_\mathbf{i}$ of the graph $G_\mathbf{i}$.
%  Given a graph $G_\mathbf{i} = (V_\mathbf{i},E_\mathbf{i})$ and a path $w_\mathbf{i}$ generated by some $k$-index $\mathbf{i}$, we let $w_\mathbf{i}(e)$ denote the number of times the edge $e \in E_\mathbf{i}$ is traversed by the path $w_\mathbf{i}$. 
%\end{definition}
%---------% 
%---------% 
\begin{lemma}[$|w_\mathbf{i}| = k$ : R-1-2 : lem:abs\_w\_i\_eq\_k]
  \label{lem:abs_w_i_eq_k}
  \uses{def:graph_multi_index,def:length_of_w_i,def:graph_edge_count}
  % We use the definition of the graph $G_\mathbf{i}$, "Length of $|w_\mathbf{i}", and "Edge Counting Function $w_\mathbf{i}(\cdot)$".
  For any $k$-index $\mathbf{i}$, the connected graph $G_\mathbf{i} = (V_\mathbf{i},E_\mathbf{i})$ has at most $k$ vertices. Furthermore
  \[
  |w_\mathbf{i}| \equiv \sum_{e \in E_\mathbf{i}} w_\mathbf{i}(e) = k.
  \]
\end{lemma}
\begin{proof}
  Foremost, since the number of vertices of the graph $G_\mathbf{i}$ 
  are the number of distinct elements of the $k$-index $\mathbf{i}$, it clearly follows that $\#V_\mathbf{i} \leq k$.
  On the other hand, recall that each $w_i(e)$ denotes the number of times the edge $e \in E_\mathbf{i}$ is traversed by the path $w_\mathbf{i}$. 
  Since $|w_\mathbf{i}| = k$ by the construction of $w_\mathbf{i}$, it follows that
  \[
  |w_\mathbf{i}| \equiv \sum_{e \in E_\mathbf{i}} w_\mathbf{i}(e) = k.
  \]
\end{proof}
%---------%
%---------%
% Motivated by these conditions, we define $\mathcal{G}_k$:
%---------% 
%---------% 
\begin{definition}[Length $|w|$ : R-1-3 : def:length\_of\_w]
  \label{def:length_of_w}
  \uses{}
  % We use the definition of an arbitrary path $w$.
  Given any graph $G=(V,E)$ and a path $w$, we let $|w|$ denote the length of $w$. 
\end{definition}
%---------% 
%---------% 
\begin{definition}[$\mathcal{G}_k$ : R-1-4 : def:g\_k]
  \label{def:g_k}
  \uses{def:length_of_w}
  Let $\mathcal{G}_k$ denote the set of all ordered pairs $(G,w)$ where $G = (V,E)$ is a connected graph with at most $k$ vertices, and
  $w$ is a closed path covering $G$ satisfying $|w| = k$.
\end{definition}
%---------%
%---------%
% We can count the set of $k$-indexes in Equation \ref{* *}: 
% For any $(G,w) \in \mathcal{G}_k$, an index with that corresponding graph $G$ and walk $w$ is completely determined by assigning which distinct values of $[n]$ appear at the vertices of $G$:
%---------%
%---------%
% The definition below proposes an alternate definition of $\mathcal{G}_k$ along with the equivalence relation. 
%\begin{definition}
%  \label{def:g_k}
%  \notready
%  \uses{}
%  %We propose an alternate (and perhaps more explicit) construction of $\mathcal{G}_k$. 
%  We define $\mathcal{G}_k$ as the set of equivalence classes of the set $\{ (G_\mathbf{i},w_\mathbf{i}) : \mathbf{i} \in [n]^k \text{ and } n \in \mathbb{N} \}$, 
%  where the equivalence relation is defined as: $(G_\mathbf{i},w_\mathbf{i}) \sim (G_{\mathbf{i}^*},w_{\mathbf{i}^*})$ 
%  if and only if there exists a bijection $\varphi$ from the set of entries $\mathbf{i}$ onto the set of entries $\mathbf{i}^*$ such that
%  \[
%  \mathbf{i} = (i_1,...,i_k) \,\, \Longleftrightarrow \,\, \mathbf{i}^* = \bigl( \varphi(i_1),\varphi(i_2),...,\varphi(i_k) \bigl).
%  \]
%  We denote an element of $\mathcal{G}_k$ as $(G,w)$.
%\end{definition}
%---------%
%---------%
\begin{lemma}[$w$ can be uniquely expressed : R-1-5-0 : def:w\_unique]
  \label{lem:w_unique}
  \uses{def:g_k}
  Given $(G,w) \in \mathcal{G}_k$, let us denote the path $w$ by
  \[
  w = (\{i_1,i_2\},\{i_3,i_4\},...,\{i_{2k-3},i_{2k-2}\},\{i_{2k-1},i_{2k}\}).
  \] 
  Then we can choose the smallest integer appearing in $\{i_1,i_2\} \cap \{i_{2k-1},i_{2k}\}$ such that $w$ takes the form
  \begin{equation}\label{equation_w_unique}
  w = (\{j_1,j_2\},\{j_2,j_3\},...,\{j_{k-1},j_k\},\{j_k,j_1\}).
  \end{equation} 
\end{lemma}
\begin{proof}
  There is at least one way and at most two ways to express $w$ in the form of Equation \ref{equation_w_unique}.
  This follows from the fact that detemining the first entry $j_1$ (for which we have two choices of $i_1$ or $i_2$) of a path completely determines the remaining entries $j_2,j_3,...,j_k$. 
  Note that this also implies that we can not express express $w$ in two `distinct' forms of Equation \ref{equation_w_unique} by starting with the same choice of $j_1$.
% If there is only one way to express $w$ in the form of Equation \ref{equation_w_unique}, then there is nothing to prove.
% If there are two ways to express $w$ in the form of Equation \ref{equation_w_unique}, then we can simply choose the smaller choice of $j_1$.
\end{proof}
%---------%
%---------%
\begin{definition}[$k$-index generated by $(G,w)$ : R-1-5 : def:g\_k\_j]
  \label{def:g_k_j}
  \uses{def:g_k,lem:w_unique}
  % We also utilize the the well-ordering of natural numbers.
  % We might also utilize the notion of a directed path.
  Given $(G,w) \in \mathcal{G}_k$, we denote $\mathbf{j}$ as the $k$-index generated by $(G,w)$ in the following way. 
  The path $w$ can be uniquely expressed under the condition of Lemma \ref{lem:w_unique}:
  \[
  w = (\{j_1,j_2\},\{j_2,j_3\},...,\{j_{k-1},j_k\},\{j_k,j_1\}).
  \] 
  We define $\mathbf{j} = (j_1,j_2,...,j_{k-1},j_k)$.
\end{definition}
%---------%
%---------%
\begin{definition}[$(G_\mathbf{i},w_\mathbf{i}) = (G,w)$ : R-1-6 : def:g\_k\_equiv]
  \label{def:g_k_equiv}
  \uses{def:graph_multi_index,def:graph_path,def:g_k,def:g_k_j}
  Let $(G_\mathbf{i},w_\mathbf{i})$ be an ordered pair generated by some $k$-index $\mathbf{i}$ and $(G,w) \in \mathcal{G}_k$.
  We say $(G_\mathbf{i},w_\mathbf{i}) = (G,w)$ if and only if there exists a bijection $\varphi$ from the set of entries $\mathbf{i}$ onto the set of entries $\mathbf{j}$ such that
  \[
  \mathbf{i} = (i_1,...,i_k) \,\, \Longleftrightarrow \,\, \mathbf{j} = \bigl( \varphi(i_1),\varphi(i_2),...,\varphi(i_k) \bigl),
  \]
  where $\mathbf{j}$ is a $k$-index generated by $(G,w)$.
\end{definition}
%---------%
%---------%
\begin{lemma}[$\mathbf{i} \sim \mathbf{j} \Rightarrow \mathbb{E}(Y_\mathbf{i}) = \mathbb{E}(Y_\mathbf{j})$ : R-1-7 : lem:eq\_equiv\_eq\_expect]
  \label{lem:eq_equiv_eq_expect}
  \uses{def:matrix_multi_index,lem:expectation_matrix_multi_index}
  Given two $k$-indexes $\mathbf{i} = (i_1,...,i_k)$ and $\mathbf{j} = (j_1,...,j_k)$, 
  suppose there exists a bijection $\varphi$ from the set of entries of $\mathbf{i}$ onto the set of entries of $\mathbf{j}$ such that
  \[
  \mathbf{i} = (i_1,...,i_k) \,\, \Longleftrightarrow \,\, \mathbf{j} = \bigl( \varphi(i_1),\varphi(i_2),...,\varphi(i_k) \bigl).
  \]
  Then $\bE (Y_\mathbf{i}) = \bE (Y_{\mathbf{j}})$.
\end{lemma}
\begin{proof}
  Given $Y_\mathbf{i} = Y_{i_1 i_2}Y_{i_2 i_3} \cdots Y_{i_{k-1} i_k}Y_{i_k i_1}$, we have
  \[
  Y_{\mathbf{j}} = Y_{j_1 j_2}Y_{j_2 j_3} \cdots Y_{j_{k-1} j_k}Y_{j_k j_1} 
  = Y_{\varphi(i_1) \varphi(i_2)}Y_{\varphi(i_2) \varphi(i_3)} \cdots Y_{\varphi(i_{k-1}) \varphi(i_k)}Y_{\varphi(i_k) \varphi(i_1)}.
  \] 
  Observe that $\{ i_{\lambda_l},i_{\lambda_{l+1}} \}$ is a singleton and only if $\{ \varphi(i_{\lambda_l}),\varphi(i_{\lambda_{l+1}}) \}$ is a singleton.
  The fact that $\{ i_{\lambda_l},i_{\lambda_{l+1}} \} = \{ i_{\lambda_\mu},i_{\lambda_{\mu+1}} \}$ if and only if $\{ \varphi(i_{\lambda_l}),\varphi(i_{\lambda_{l+1}}) \} = \{ \varphi(i_{\lambda_\mu}),\varphi(i_{\lambda_{\mu+1}}) \}$ completes the proof.
\end{proof}
%---------%
%---------%
%\begin{lemma}
%  \label{lem:equal_equiv_class_equal_expectation}
%  \notready
%  \uses{}
%  %\uses:
%  %(1)Definition of $Y_\mathbf{i}$, 
%  Given two $k$-indexes $\mathbf{i} = (i_1,...,i_k)$ and $\mathbf{i}^* = (j_1,...,j_k)$, 
%  suppose there exists a bijection $\varphi$ from the set of entries $\mathbf{i}$ onto the set of entries $\mathbf{i}^*$ such that
%  \[
%  \mathbf{i} = (i_1,...,i_k) \,\, \Longleftrightarrow \,\, \mathbf{i}^* = \bigl( \varphi(i_1),\varphi(i_2),...,\varphi(i_k) \bigl).
%  \]
%  Then $\bE (Y_\mathbf{i}) = \bE (Y_{\mathbf{i}^*})$.
%\end{lemma}
%\begin{proof}
%  It suffices to show that $Y_\mathbf{i}$ and $Y_{\mathbf{i}^*}$ represent the same number of, respectively, self-edges and connecting edges.
%  Given $Y_\mathbf{i} = Y_{i_1 i_2}Y_{i_2 i_3} \cdots Y_{i_{k-1} i_k}Y_{i_k i_1}$,
%  \[
%  Y_{\mathbf{i}^*} = Y_{j_1 j_2}Y_{j_2 j_3} \cdots Y_{j_{k-1} j_k}Y_{j_k j_1} 
%  = Y_{\varphi(i_1) \varphi(i_2)}Y_{\varphi(i_2) \varphi(i_3)} \cdots Y_{\varphi(i_{k-1}) \varphi(i_k)}Y_{\varphi(i_k) \varphi(i_1)}.
%  \] 
%  The fact that $\{ i_{\lambda_l},i_{\lambda_{l+1}} \}$ is a self-edge 
%  if (i.e. a singleton) and only if $\{ \varphi(i_{\lambda_l}),\varphi(i_{\lambda_{l+1}}) \}$ is a self-edge completes the proof.
%\end{proof}
%---------%
%---------%
\begin{definition}[$|G|$ : R-1-8 : def:abs.G]
  \label{def:abs.G}
  \uses{def:g_k}
  Given an ordered pair $(G,w) \in \mathcal{G}_k$, we define $|G|$ to be the number of distinct vertices in the graph $G$.
\end{definition}
%---------%
%---------%
\begin{lemma}[Lemma 4.3 in \cite{Kemp2013RMTNotes} : R-1-9 : lem:lem\_4.3]
  \label{lem:lem_4.3}
  \uses{def:g_k,def:graph_multi_index,def:graph_path,def:g_k_equiv,lem:eq_equiv_eq_expect}
  Given $(G,w) \in \mathcal{G}_k$, we have
  \[
  \# \{ \mathbf{i} \in [n]^k : (G_\mathbf{i},w_\mathbf{i}) = (G,w) \} = n (n-1) \cdots (n - |G| + 1).
  \]
\end{lemma}
\begin{proof}
  By the way the equivalence relation is defined in Definition \ref{def:g_k}, 
  the fact that there are $n (n - 1) \cdots (n -|G| + 1)$ ways to assign $|G|$ distinct values from $[n]$ into the indices $i_1,...,i_{|G|}$ completes the proof.
\end{proof}
%---------%
% Using Equation \ref{* *}, we can re-index the sum of Equation \ref{* *} as
%---------%
\begin{lemma}[Partitioning into double summation : R-1-10 : lem:equation\_4.5\_1]
  \label{lem:equation_4.5_1}
  \uses{def:g_k,def:g_k_equiv,lem:trace_expectation_of_matrix}
  \[
  \bE \Tr (\mathbf{Y}_\mathbf{i}^k) = \sum_{(G,w) \in \mathcal{G}_k} \sum_{\substack{\mathbf{i} \in [n]^k \\ (G_\mathbf{i},w_\mathbf{i}) = (G,w)}} \bE (Y_\mathbf{i}).
  \]
\end{lemma}
\begin{proof}
  This follows from `partitioning' the summation appearing in Lemma \ref{lem:trace_expectation_of_matrix} using the equivalence relation defined in Definition \ref{def:g_k_equiv}.
\end{proof}
%---------%
%---------%
\begin{definition}[$\Pi (G,w)$: R-1-11 : def:Pi.G.w]
  \label{def:Pi.G.w}
  \uses{def:g_k,def:g_k_j}
  Given an ordered pair $(G,w) \in \mathcal{G}_k$, let $\mathbf{j}$ be the $k$-index generated by $(G,w)$. We define 
  \[
  \Pi (G,w) = \mathbb{E}(Y_\mathbf{j}).
  \]
\end{definition}
%---------%
%---------%
\begin{lemma}[Re-indexing the sum with counting argument : R-1-12 : lem:equation\_4.5\_2]
  \label{lem:equation_4.5_2}
  \uses{lem:equation_4.5_1,lem:eq_equiv_eq_expect,lem:lem_4.3,def:Pi.G.w}
  \[
  \bE \Tr (\mathbf{Y}_\mathbf{i}^k) = \sum_{(G,w) \in \mathcal{G}_k} \Pi (G,w) \cdot \# \{ \mathbf{i} \in [n]^k : (G_\mathbf{i},w_\mathbf{i}) = (G,w) \}.
  \]
\end{lemma}
\begin{proof}
  This follows from re-indexing the sum of Lemma \ref{lem:equation_4.5_1} by using Lemma \ref{lem:eq_equiv_eq_expect} and Lemma \ref{lem:lem_4.3}.
\end{proof}
%---------%
%---------%
\begin{lemma}[Re-introducing the renormalization factor : R-1-13 : lem:equation\_4.5\_3]
  \label{lem:equation_4.5_3}
  \uses{lem:equation_4.5_2}
  \[
  \frac{1}{n} \bE \Tr (\mathbf{X}_n^k) = \sum_{(G,w) \in \mathcal{G}_k} \Pi (G,w) \cdot \frac{n (n-1) \cdots (n - |G| + 1)}{n^{k/2+1}}.
  \]
\end{lemma}
\begin{proof}
  Combining with the renormalization factor $n^{-1}$ of Proposition \ref{???} gives
  \[
  \frac{1}{n} \bE \Tr (\mathbf{X}_n^k) = \frac{1}{n^{k/2+1}} \bE \Tr (\mathbf{Y}_\mathbf{i}^k).
  \]
  Substituting the term $\bE \Tr (\mathbf{Y}_\mathbf{i}^k)$ with the expression in Equation \ref{lem:equation_4.5_2} gives
  \[
  \frac{1}{n} \bE \Tr (\mathbf{X}_n^k) = \sum_{\substack{(G,w) \in \mathcal{G}_k}} \Pi (G,w) \cdot \frac{n (n-1) \cdots (n - |G| + 1)}{n^{k/2+1}}.
  \]
\end{proof}
% Note that the summation is finite, and thus we only need to determine the values of $\Pi (G,w)$ to evaluate the summation. 
%---------%
%---------%
\begin{definition}[$\mathcal{G}_{k, w \geq 2}$ : R-1-14 : def:g\_k\_ge\_2]
  \label{def:g_k_ge_2}
  \uses{def:g_k,def:graph_path}
  Let $\mathcal{G}_{k,w \geq 2}$ be a subset of $\mathcal{G}_k$ in which the walk $w$ traverses each edge at least twice.
\end{definition}
%---------%
%\begin{definition}
%  \label{def:g_k_ge_2}
%  \uses{def:g_k}
%  Given an equivalence class $(G,w) \in \mathcal{G}_k$, 
%  let $\substack{\mathcal{G}_k \\ w \geq 2}$ be a subset of $\mathcal{G}_k$ 
%  in which the walk $w_\mathbf{i}$ represetned by every $(G_\mathbf{i},w_\mathbf{i}) \in (G,w)$ 
%  crosses each edge at least twice.
%\end{definition}
%---------%
%\begin{definition}
%  \label{def:Pi.graph}
%  \uses{def:g_k}
%  %uses the definition of $\Pi (G_\mathbf{i},w_\mathbf{i})$.
%  Given an equivalence class $(G,w) \in \mathcal{G}_k$, let 
%  \[
%  \Pi (G,w) = \Pi (G_\mathbf{i},w_\mathbf{i})
%  \]
%  where $(G_\mathbf{i},w_\mathbf{i}) \in (G,w)$.
%\end{definition}
%---------%
%\begin{lemma}
%  \label{lem:g_k_ge_2_wd}
%  \uses{def:g_k_ge_2,lem:equal_equiv_class_equal_expectation}
%  The set $\substack{\mathcal{G}_k \\ w \geq 2}$ in Definition \ref{def:g_k_ge_2} is well-defined.
%  Furthermore, the common value $\Pi (G,w)$ in Definition \ref{def:Pi.graph} is well-defined.  
%\end{lemma}
%\begin{proof}
%  The proof for both statements follows an identical reasoning as in the proof of Lemma \ref{lem:equal_equiv_class_equal_expectation}.
%\end{proof}
%---------%
%---------%
\begin{lemma}[$\Pi (G,w) = 0$ : R-1-15 : lem:Pi.prod\_eq\_zero\_if\_w\_le\_two]
  \label{lem:Pi.prod_eq_zero_if_w_le_two}
  \uses{def:g_k,def:expectation_matrix_multi_index,def:length_of_w_i,def:g_k_j}
  Given an ordered pair $(G,w) \in \mathcal{G}_k$, suppose there exists an edge $e \in E$ in which it is traversed only once in the walk $w$. Then
  \[
  \Pi (G,w) = 0.
  \]
\end{lemma}
\begin{proof}
  Let $(G,w) \in \mathcal{G}_k$ and let $\mathbf{j}$ be the $k$-index generated by $(G,w)$. Suppose there exists an edge $e \in E_\mathbf{j}$ such that $w_\mathbf{j}(e) = 1$.
  This means, in Lemma \ref{lem:expectation_matrix_multi_index}, a singleton term $\bE (Y_{ij}^{w_\mathbf{j}(e)}) = \bE (Y_{ij})$ appears.
  The rest of the proof follows from the assumption of Proposition \ref{???} that $\bE (Y_{ij}) = 0$ for every $i$ and $j$. 
\end{proof}
%---------%
%\begin{lemma}
%  \label{lem:Pi.prod_eq_zero_if_w_le_two}
%  \uses{def:g_k,prop:matrix_moments_convergence}
%  %uses definition of $Y_\mathbf{i}$ AND $\Pi (G,w)$.
%  Given an ordered pair $(G_\mathbf{i},w_\mathbf{i})$, suppose there exists an edge $e \in E_\mathbf{i}$ in which it is traversed only once in the walk $w_\mathbf{i}$.
%  Then
%  \[
%  \Pi (G,w) = 0.
%  \]
%\end{lemma}
%\begin{proof}
%  This directly follows from the assumption of Proposition \ref{prop:matrix_moments_convergence} that $\bE (Y_{ij}) = 0$ for every $i$ and $j$. 
%\end{proof}
%---------%
%---------%
\begin{lemma}[Simplifying the summation with the fact $\Pi (G,w) = 0$ in certain cases: R-1-16 : lem:equation\_4.8]
  \label{lem:equation_4.8}
  \uses{def:g_k_ge_2,lem:equation_4.5_3,lem:Pi.prod_eq_zero_if_w_le_two}
  \[
  \frac{1}{n} \bE \Tr (\mathbf{X}_n^k) 
  = \sum_{(G,w) \in \mathcal{G}_{k, w \geq 2}} \Pi (G,w) \cdot \frac{n (n-1) \cdots (n - |G| + 1)}{n^{k/2+1}}.
  \]
\end{lemma}
\begin{proof}
  This follows from applying the result of Lemma \ref{lem:Pi.prod_eq_zero_if_w_le_two} to Lemma \ref{lem:equation_4.5_3}.
\end{proof}
%---------%
%---------%
\begin{lemma}[$\#E \leq k/2$ : R-1-17 : lem:edge\_set\_order\_leq\_k\_over\_two]
  \label{lem:edge_set_order_leq_k_over_two}
  \uses{def:g_k_ge_2}
  Given an ordered pair $(G,w) \in \mathcal{G}_{k,w \geq 2}$, we must have $\# E \leq k/2$.
\end{lemma}
\begin{proof}
  Since $|w| = k$, if each edge in $G$ is traversed at least twice, then by construction of $w$ the number of edges is at most $k/2$.
\end{proof}
%---------%
% This lemma might be merged with the following one(s).
% If using the equivalence definition of $\mathcal{G}_k$, it might be more convenient to introduce the Axiom of Choice to ease the notation on $(G,w)$.
%---------%
%---Richard's Part for Proposition 4.1 (Updated) (End)---% 






%---Richard's Part for Proposition 4.5 (Start)---% 
%---------%
%---------%
\begin{definition}[R-2-1 : def:common\_val\_prod\_of]
  \label{def:common_val_prod_of}
  \uses{def:matrix_multi_index,def:ordered_triple} % Uses: 1) Definition of $Y_\mathbf{i}$; 2) Definition of the ordered triple 
  Given an ordered triple $(G_{\mathbf{i}\#\mathbf{j}},w_\mathbf{i},w_\mathbf{j})$ generated by two $k$-indexes $\mathbf{i}$ and $\mathbf{j}$, we define
  \[
  \pi(G_{\mathbf{i}\#\mathbf{j}},w_\mathbf{i},w_\mathbf{j}) = \mathbb{E}(Y_\mathbf{i} Y_\mathbf{j}) - \mathbb{E}(Y_\mathbf{i}) \mathbb{E}(Y_\mathbf{j}).
  \]
\end{definition}
%---------%
%---------%
\begin{definition}[R-2-2 : def:graph\_walk\_triple\_set]
  \label{def:graph_walk_triple_set}
  \uses{} % n/a
  We define $\mathcal{G}_{k,k}$ to be the set of connected graphs $G$ with $\leq 2k$ vertices, 
  together with two paths each of length $k$ whose union covers $G$.
\end{definition}
%---------%
%---------%
\begin{definition}[R-2-3-1 : def:index\_pair]
  \label{def:index_pair}
  \uses{def:g_k_j} % Uses: 1) [R-1-5]
  Given $(G,w,w') \in \mathcal{G}_{k,k}$, we say $(\mathbf{i},\mathbf{j})$ is an ordered pair of $k$-indexes generated by $(G,w,w')$ 
  if $\mathbf{i}$ and $\mathbf{j}$ are, repsectively, $k$-indexes generated by $w$ and $w'$ as in Definition \ref{def:g_k_j}.
\end{definition}
%---------%
%---------%
\begin{definition}[R-2-3-2 : def:index\_pair\_rel]
  \label{def:index_pair_rel}
  \uses{}
  Let $\mathbf{i}_\lambda$ and $\mathbf{j}_\lambda$ be $k$-indexes for each $\lambda=1,2$. We say $(\mathbf{i}_1,\mathbf{j}_1) = (\mathbf{i}_2,\mathbf{j}_2)$ if and only if there exists a bijection $\varphi_1$ from the set of entries $\mathbf{i}_1$ onto the set of entries $\mathbf{i}_2$, 
  and a bijection $\varphi_2$ from the set of entries $\mathbf{j}_1$ onto the set of entries $\mathbf{j}_2$ such that
  \begin{equation}
  \begin{split}
    \mathbf{i}_1 = (i_1,...,i_k) & \,\, \Longleftrightarrow \,\, \mathbf{i}_2 = \bigl( \varphi_1(i_1),\varphi_1(i_2),...,\varphi_1(i_k) \bigl)
  \end{split}
  \end{equation}
  and
  \begin{equation}
  \begin{split}
    \phantom{.}\mathbf{j}_1 = (j_1,...,j_k) & \,\, \Longleftrightarrow \,\, \mathbf{j}_2 = \bigl( \varphi_2(j_1),\varphi_2(j_2),...,\varphi_2(j_k) \bigl).
  \end{split}
  \end{equation}
\end{definition}
%---------%
%---------%
\begin{definition}[R-2-3-3 : def:graph\_walk\_triple\_rel]
  \label{def:graph_walk_triple_rel}
  \uses{}% Uses: 1) [R-2-3-0]; 2) [R-2-3-1]
  Let $(G_{\mathbf{i} \# \mathbf{j}},w_\mathbf{i},w_\mathbf{j})$ be an ordered triple generated by two $k$-indexes $\mathbf{i}$ and $\mathbf{j}$, and let $(G,w,w') \in \mathcal{G}_{k,k}$.
  We say $(G_{\mathbf{i} \# \mathbf{j}},w_\mathbf{i},w_\mathbf{j}) = (G,w,w')$ if and only $(\mathbf{i},\mathbf{j}) = (\mathbf{i}^*,\mathbf{j}^*)$,
  where $(\mathbf{i}^*,\mathbf{j}^*)$ is an ordered pair of $k$-indexes generated by $(G,w,w')$.
\end{definition}
%---------%
%---------%
\begin{lemma}[R-2-3-4 : lem:common\_val\_eq\_of\_index\_pair\_rel]
  \label{lem:common_val_eq_of_index_pair_rel}
  \uses{}
  Let $\mathbf{i}_\lambda$ and $\mathbf{j}_\lambda$ be $k$-indexes for each $\lambda=1,2$ such that $(\mathbf{i}_1,\mathbf{j}_1) = (\mathbf{i}_2,\mathbf{j}_2)$.
  Then $\bE (Y_{\mathbf{i}_1}Y_{\mathbf{j}_1}) = \bE (Y_{\mathbf{i}_1}Y_{\mathbf{j}_2})$.
\end{lemma}
\begin{proof}
  This follows a similar reasoning as in Lemma \ref{lem:eq_equiv_eq_expect}.
\end{proof}
%---------%
%---------%
\begin{lemma}[R-2-3-5 : lem:common\_val\_prod\_of\_eq\_of\_graph\_walk\_triple\_rel]
  \label{lem:common_val_prod_eq_of_graph_walk_triple_rel}
  \uses{}% Uses: 1) Definition of the common value; 2) Definition of $Y_\mathbf{i}$; 3) Definition of the ordered triple 
  Let $(G_{\mathbf{i}_1 \# \mathbf{j}_1},w_{\mathbf{i}_1},w_{\mathbf{j}_1})$ and $(G_{\mathbf{i}_2 \# \mathbf{j}_2},w_{\mathbf{i}_2},w_{\mathbf{j}_2})$ be ordered triples generated by their respective ordered pair of $k$-indexes, and $(G,w,w') \in \mathcal{G}_{k,k}$,
  If $(G_{\mathbf{i}_1 \# \mathbf{j}_1},w_{\mathbf{i}_1},w_{\mathbf{j}_2}) = (G,w,w')$ and $(G_{\mathbf{i}_1 \# \mathbf{j}_2},w_{\mathbf{i}_2},w_{\mathbf{j}_2}) = (G,w,w')$, then
%  If the ordered triples $(G_{\mathbf{i}\#\mathbf{j}},w_\mathbf{i},w_\mathbf{j})$ and $(G_{\mathbf{i}^*\#\mathbf{j}^*},w_{\mathbf{i}^*},w_{\mathbf{j}^*})$ are same up to relabeling, then
  \[
  \pi(G_{\mathbf{i}_1\#\mathbf{j}_1},w_{\mathbf{i}_1},w_{\mathbf{j}_1}) = \pi(G_{\mathbf{i}_2 \# \mathbf{j}_2},w_{\mathbf{i}_2},w_{\mathbf{j}_2}).
  \]
\end{lemma}
\begin{proof}
This follows from Lemma \ref{lem:eq_equiv_eq_expect}.
\end{proof}
%---------%
%---------%
\begin{lemma}[R-2-4 : lem:sum\_eq\_sum\_over\_classes]
  \label{lem:sum_eq_sum_over_classes}
  \uses{}% Uses: 1) Common value lemma; 2) Previous variance equation (2nd eq of p.14) (Paul's); 3) equivalence relation
  \[
  \text{Var} \biggl(\frac{1}{n} \Tr (\mathbf{X}_n^k) \biggl) 
  = \frac{1}{n^{k+2}} \sum_{(G,w,w') \in \mathcal{G}_{k,k}} \sum_{\substack{\mathbf{i},\mathbf{j} \in [n]^k \\ (G_{\mathbf{i}\#\mathbf{j}},w_\mathbf{i},w_\mathbf{j}) = (G,w,w')}} [\mathbb{E}(Y_\mathbf{i} Y_\mathbf{j}) - \mathbb{E}(Y_\mathbf{i}) \mathbb{E}(Y_\mathbf{j})].
  \]
\end{lemma}
\begin{proof}
  This follows from `partitioning' the summation appearing in Lemma \ref{???} using the equivalence relation defined in Definition \ref{???} ([R-2-3-1]).
% Given an ordered triple $(G_{\mathbf{i}\#\mathbf{j}},w_\mathbf{i},w_\mathbf{j})$ generated by two $k$-indexes $\mathbf{i}$ and $\mathbf{j}$, and $(G,w,w') \in \mathcal{G}_{k,k}$, 
%  only one of $(G_{\mathbf{i}\#\mathbf{j}},w_\mathbf{i},w_\mathbf{j}) = (G,w,w')$ or $(G_{\mathbf{i}\#\mathbf{j}},w_\mathbf{i},w_\mathbf{j}) \neq (G,w,w')$ holds. 
%  Hence, the partition of the summation is well-defined.
\end{proof}
%---------%
%---------%
\begin{definition}[R-2-5-1 : def:common\_val\_prod]
  \label{def:common_val_prod}
  \uses{} % Uses: Definition of $Y_\mathbf{i}$;
  Given $(G,w,w')$, let $(\mathbf{i},\mathbf{j})$ be an ordered pair of $k$-indexes generated by $w$ and $w'$. 
  We define
  \[
  \pi(G,w,w') = \mathbb{E}(Y_\mathbf{i} Y_\mathbf{j}) - \mathbb{E}(Y_\mathbf{i}) \mathbb{E}(Y_\mathbf{j}).
  \]
\end{definition}
%---------%
%---------%
\begin{lemma}[R-2-5-2 : lem:inner\_sum\_eq\_common\_val\_prod\_mul\_card]
  \label{lem:inner_sum_eq_common_val_prod_mul_card}
  \uses{}% Uses: 1) Common value lemma; 2) Sum relabeling 1; 3) [R-2-5-0]; 4) [R-2-3-0]
  \[
  \text{Var} \biggl(\frac{1}{n} \Tr (\mathbf{X}_n^k) \biggl) 
  = \frac{1}{n^{k+2}} \sum_{(G,w,w') \in \mathcal{G}_{k,k}} \pi(G,w,w')
\cdot \# \bigl\{ (\mathbf{i},\mathbf{j}) \in [n]^{2k} : (G_{\mathbf{i} \# \mathbf{j}},w_\mathbf{i},w_\mathbf{j}) = (G,w,w') \bigl\}.
  \]
\end{lemma}
\begin{proof}
  This follows from re-indexing the sum of Lemma \ref{???} using Lemma \ref{???}.
  % Lemma \ref{R-2-4} & Lemma \ref{R-2-3}
\end{proof}
%---------%
%---------%
\begin{definition}[R-2-6-1 : def:def:graph\_walk\_triple\_single\_edges]
  \label{def:graph_walk_triple_single_edges}
  \uses{}% Uses: 1) Definition of a glued graph (Paul) / ordered triple
  Given a graph $G_{\mathbf{i} \# \mathbf{j}}$, let $E^s_{\mathbf{i} \# \mathbf{j}}$ denote the set of self-edges.
\end{definition}
%---------%
%---------%
\begin{definition}[R-2-6-2 : def:graph\_walk\_triple\_repeated\_edges]
  \label{def:graph_walk_triple_repeated_edges}
  \uses{}% Uses: 1) Definition of a glued graph (Paul) / ordered triple
  Given a graph $G_{\mathbf{i} \# \mathbf{j}}$, let $E^c_{\mathbf{i} \# \mathbf{j}}$ denote the set of connecting edges.
\end{definition}
%---------%
%---------%
\begin{definition}[R-2-7]
  \label{def:R-2-7}% Path counting function
  \uses{}% Uses: 1) Definition of a glued graph (Paul) / ordered triple / path
  Given a graph $G_{\mathbf{i} \# \mathbf{j}}$, 
  let $w_{\mathbf{i} \# \mathbf{j}}(e)$ denote the number of times the edge $e$ is traversed by either of the two paths $w_\mathbf{i}$ and $w_\mathbf{j}$.
\end{definition}
%---------%
%---------%
\begin{lemma}[R-2-8]
  \label{lem:R-2-8}% Prod Prod form 1
  \uses{}% Uses: Equation 4.4 (Paul's)
  \[
  \mathbb{E} (Y_\mathbf{i}Y_\mathbf{j}) 
  = \prod_{e_s \in E^s_{\mathbf{i} \# \mathbf{j}}} \mathbb{E} (Y_{11}^{w_{\mathbf{i} \# \mathbf{j}}(e_s)}) \cdot \prod_{e_c \in E^c_{\mathbf{i} \# \mathbf{j}}} \mathbb{E} (Y_{12}^{w_{\mathbf{i} \# \mathbf{j}}(e_c)}).
  \]
\end{lemma}
\begin{proof}
  This follows from Lemma \ref{???} and the independency of random variables.
  % Equation 4.4 (Paul's)
\end{proof}
%---------%
%---------%
\begin{lemma}[R-2-9]
  \label{lem:R-2-9}% Prod Prod form 2
  \uses{}% Uses: Equation 4.4 (Paul's)
  \[
  \mathbb{E}(Y_\mathbf{i}) \mathbb{E}(Y_\mathbf{j}) 
  = \prod_{e_s \in E^s_{\mathbf{i}}} \mathbb{E} (Y_{11}^{w_{\mathbf{i}}(e_s)}) \cdot \prod_{e_c \in E^c_{\mathbf{i}}} \mathbb{E} (Y_{12}^{w_{\mathbf{i}}(e_c)})
  \cdot \prod_{e_s \in E^s_{\mathbf{j}}} \mathbb{E} (Y_{11}^{w_{\mathbf{j}}(e_s)}) \cdot \prod_{e_c \in E^c_{\mathbf{j}}} \mathbb{E} (Y_{12}^{w_{\mathbf{j}}(e_c)}).
  \]
\end{lemma}
\begin{proof}
  This directly follows from Lemma \ref{???}.
  % Equation 4.4 (Paul's)
\end{proof}
%---------%
%---------%
\begin{lemma}[R-2-10]
  \label{lem:R-2-10}% Prod Prod form total
  \uses{}% Uses: 1) Equation 4.4 (Paul's); 2) Prod Prod form 1; 3) Prod Prod form 2
  \[
  \mathbb{E} (Y_\mathbf{i}Y_\mathbf{j}) -  \mathbb{E}(Y_\mathbf{i}) \mathbb{E}(Y_\mathbf{j}) 
  = \prod_{e_s \in E^s_{\mathbf{i} \# \mathbf{j}}} \mathbb{E} (Y_{11}^{w_{\mathbf{i} \# \mathbf{j}}(e_s)}) \cdot \prod_{e_c \in E^c_{\mathbf{i} \# \mathbf{j}}} \mathbb{E} (Y_{12}^{w_{\mathbf{i} \# \mathbf{j}}(e_c)})
  -  \prod_{e_s \in E^s_{\mathbf{i}}} \mathbb{E} (Y_{11}^{w_{\mathbf{i}}(e_s)}) \cdot \prod_{e_c \in E^c_{\mathbf{i}}} \mathbb{E} (Y_{12}^{w_{\mathbf{i}}(e_c)})
  \cdot \prod_{e_s \in E^s_{\mathbf{j}}} \mathbb{E} (Y_{11}^{w_{\mathbf{j}}(e_s)}) \cdot \prod_{e_c \in E^c_{\mathbf{j}}} \mathbb{E} (Y_{12}^{w_{\mathbf{j}}(e_c)}).
  \]
\end{lemma}
\begin{proof}
  This follows from substituting the expressions derived in Lemma \ref{???} and Lemma \ref{???}.
  % Prod prod form 1 & 2
\end{proof}
%---------%
%---------%
\begin{lemma}[R-2-11]
  \label{lem:R-2-11} % Couting path adds up to $2k$
  \uses{} % Uses: 1) Defintiion of the counting function; 2) Definition of the glued graph; 3) Definition of the graph Gi
  \[
  \sum_{e \in E_{\mathbf{i} \# \mathbf{j}}} w_{\mathbf{i} \# \mathbf{j}}(e) 
  = 2k 
  = \sum_{e \in E_\mathbf{i}} w_{\mathbf{i}}(e) + \sum_{e \in E_\mathbf{j}} w_{\mathbf{j}}(e).
  \]
\end{lemma}
\begin{proof}
  By construction of the paths $w_\mathbf{i}$ and $w_\mathbf{j}$,
  \[
  \sum_{e \in E_\mathbf{i}} w_{\mathbf{i}}(e) = k = \sum_{e \in E_\mathbf{j}} w_{\mathbf{j}}(e).
  \]
  By definition of the counting function $w_{\mathbf{i} \# \mathbf{j}}(\cdot)$,
  \[
  \sum_{e \in E_{\mathbf{i} \# \mathbf{j}}} w_{\mathbf{i} \# \mathbf{j}}(e)  
  = \sum_{e \in E_\mathbf{i}} w_{\mathbf{i}}(e) + \sum_{e \in E_\mathbf{j}} w_{\mathbf{j}}(e)
  = 2k.
  \]
\end{proof}
%---------%
%---------%
\begin{lemma}[R-2-12-1]
  \label{lem:R-2-12-1} % Individual term bound 1.1
  \uses{} % Uses: 1) Definition of moments; 2) Equation 4.4 (Paul's)
  For any $k$-indexes $\mathbf{i}$ and $\mathbf{j}$, there exists $n \in \mathbb{N}$ such that
  \[
  \mathbb{E} (Y_{11}^{w_{\mathbf{i} \# \mathbf{j} (e)}}) \leq r_n
  \]
  for every $e \in E_{\mathbf{i} \# \mathbf{j}}$.
\end{lemma}
\begin{proof}
  Let $(G_{\mathbf{i}\#\mathbf{j}},w_\mathbf{i},w_\mathbf{j})$ be the ordered triple generated by the two $k$-indexes $\mathbf{i}$ and $\mathbf{j}$. 
  By construction, there are finite number of edges $e \in E_{\mathbf{i} \# \mathbf{j}}$.
  Moreover, there are finite number of integer values the counting function $w_{\mathbf{i} \# \mathbf{j}}(\cdot)$ can take over $E_{\mathbf{i} \# \mathbf{j}}$.
  This implies we can choose the maximum element of the set
  \[
  \mathscr{E} := \{ \mathbb{E} (Y_{11}^\lambda) : \lambda = w_{\mathbf{i} \# \mathbf{j}}(e) \text{ for some } e \in E_{\mathbf{i} \# \mathbf{j}} \},
  \]
  and therefore $\lambda \in \mathbb{N}$ so that $\mathbb{E}(Y_{11}^\lambda) = \max \mathscr{E}$.
  Using the fact that $\mathbb{E}(Z) \leq \mathbb{E}(|Z|)$ for any random variable $Z$,
  \[
  \mathbb{E} (Y_{11}^{w_{\mathbf{i} \# \mathbf{j} (e)}}) \leq \max\{ \mathbb{E}(|Y_{11}^\lambda|), \mathbb{E}(|Y_{12}^\lambda|) \} = r_\lambda.
  \]
\end{proof}
%---------%
%---------%
\begin{lemma}[R-2-12-2]
  \label{lem:R-2-12-2} % Individual term bound 1.2
  \uses{} % Uses: 1) Definition of moments; 2) Equation 4.4 (Paul's)
  For any $k$-indexes $\mathbf{i}$ and $\mathbf{j}$, there exists $m \in \mathbb{N}$ such that
  \[
  \mathbb{E} (Y_{12}^{w_{\mathbf{i} \# \mathbf{j} (e)}}) \leq r_m
  \]
  for every $e \in E_{\mathbf{i} \# \mathbf{j}}$.
\end{lemma}
\begin{proof}
  This follows an identical reasoning as in Lemma \ref{???}.
\end{proof}
%---------%
%---------%
\begin{lemma}[R-2-13]
  \label{lem:R-2-13} % Bound 1
  \uses{} % Uses: 1) Definition of moments; 2) Equation 4.4 (Paul's)
  For any $k$-indexes $\mathbf{i}$ and $\mathbf{j}$, there exists $M_1 \in \mathbb{R}$ such that
  \[
  \mathbb{E} (Y_\mathbf{i}Y_\mathbf{j}) \leq M_1.
  \]
\end{lemma}
\begin{proof}
  By Lemma \ref{???}, we have
  % Prod prod form 1
  \[
  \mathbb{E} (Y_\mathbf{i}Y_\mathbf{j}) 
  = \prod_{e_s \in E^s_{\mathbf{i} \# \mathbf{j}}} \mathbb{E} (Y_{11}^{w_{\mathbf{i} \# \mathbf{j}}(e_s)}) \cdot \prod_{e_c \in E^c_{\mathbf{i} \# \mathbf{j}}} \mathbb{E} (Y_{12}^{w_{\mathbf{i} \# \mathbf{j}}(e_c)}).
  \]
  By Lemma \ref{???}, there exists $n \in \mathbb{N}$ so that $\mathbb{E} (Y_{11}^{w_{\mathbf{i} \# \mathbf{j} (e)}}) \leq r_n$ for every $e \in E_{\mathbf{i} \# \mathbf{j}}$. 
  On the other hand, by Lemma \ref{???}, there exists $m \in \mathbb{N}$ so that $\mathbb{E} (Y_{11}^{w_{\mathbf{i} \# \mathbf{j} (e)}}) \leq r_m$ for every $e \in E_{\mathbf{i} \# \mathbf{j}}$. 
  Then
  % R-2-12
  \[
  \mathbb{E} (Y_\mathbf{i}Y_\mathbf{j}) 
  = \prod_{e_s \in E^s_{\mathbf{i} \# \mathbf{j}}} \mathbb{E} (Y_{11}^{w_{\mathbf{i} \# \mathbf{j}}(e_s)}) \cdot \prod_{e_c \in E^c_{\mathbf{i} \# \mathbf{j}}} \mathbb{E} (Y_{12}^{w_{\mathbf{i} \# \mathbf{j}}(e_c)})
  \leq 2k \cdot \max\{r_n,r_m\}.
  \]
  Setting $M_1 := 2k \cdot \max\{r_n,r_m\}$ satisfies the problem.
\end{proof}
%---------%
%---------%
\begin{lemma}[R-2-14]
  \label{lem:R-2-14} % Individual term bound 2.1
  \uses{} % Uses: 1) Definition of moments; 2) Equation 4.4 (Paul's)
  For any $k$-index $\mathbf{i}$, there exists $n \in \mathbb{N}$ such that
  \[
  \mathbb{E} (Y_{11}^{w_{\mathbf{i}}(e)}) \leq r_n.
  \]
  for every $e \in E_{\mathbf{i} \# \mathbf{j}}$.
\end{lemma}
\begin{proof}
  This follows a similar reasoning as in Lemma \ref{???} with the counting function $w_\mathbf{i}(\cdot)$.
\end{proof}
%---------%
%---------%
\begin{lemma}[R-2-15]
  \label{lem:R-2-15} % Individual term bound 2.2
  \uses{} % Uses: 1) Definition of moments; 2) Equation 4.4 (Paul's)
  For any $k$-index $\mathbf{i}$, there exists $n \in \mathbb{N}$ such that
  \[
  \mathbb{E} (Y_{12}^{w_{\mathbf{i}}(e)}) \leq r_n.
  \]
  for every $e \in E_\mathbf{i}$.
\end{lemma}
\begin{proof}
  This follows a similar reasoning as in Lemma \ref{???} with the counting function $w_\mathbf{i}(\cdot)$.
\end{proof}
%---------%
%---------%
\begin{lemma}[R-2-16]
  \label{lem:R-2-16} % Individual term bound 2.2
  For any $k$-indexes $\mathbf{i}$, there exists $M_2 \in \mathbb{R}$ such that
  \[
  \mathbb{E} (Y_\mathbf{i}) \leq M_2.
  \]
\end{lemma}
\begin{proof}
  This follows an identical reasoning as in Lemma \ref{???}.
\end{proof}
%---------%
%---------%
\begin{lemma}[R-2-17]
  \label{lem:R-2-17} % Bound Total
  \uses{} % Uses: Bound 1 & 2
  For any $k$-indexes $\mathbf{i}$ and $\mathbf{j}$, there exists $M_{2k} \in \mathbb{R}_{\geq 0}$ such that
  \[
  | \mathbb{E} (Y_\mathbf{i}Y_\mathbf{j}) -  \mathbb{E}(Y_\mathbf{i}) \mathbb{E}(Y_\mathbf{j}) |
  \leq 2 M_{2k}.
  \] 
\end{lemma}
\begin{proof}
  Combining the result of Lemma \ref{???} and Lemma \ref{???} gives
  \[
  | \mathbb{E} (Y_\mathbf{i}Y_\mathbf{j}) -  \mathbb{E}(Y_\mathbf{i}) \mathbb{E}(Y_\mathbf{j}) |
  \leq | \mathbb{E} (Y_\mathbf{i}Y_\mathbf{j}) | + | \mathbb{E}(Y_\mathbf{i})| | \mathbb{E}(Y_\mathbf{j}) |
  \leq M_1 + M_2^{(\mathbf{i})} \cdot M_2^{(\mathbf{j})},
  \]
  where $M_2^{(\mathbf{i})}$ and $M_2^{(\mathbf{j})}$ are the upper bounds acquired by Lemma \ref{???} with repsect to $\mathbf{i}$ and $\mathbf{j}$, repsectively. 
  Setting $M_{2k} := \max\{ M_1, M_2^{(\mathbf{i})} \cdot M_2^{(\mathbf{j})} \}$ satisfies the problem.
\end{proof}
%---------%
%---------%
% \begin{lemma}
%  \label{}
%  \uses{}
%\end{lemma}
%\begin{proof}
%  % TBW
%\end{proof}
%---------%
%---------%
% We might not need the last commented-off lemma.
%---Richard's Part for Proposition 4.5 (End)---% 





\iffalse
%---------%
%---------%
\begin{definition}
  \label{}
  \uses{}%Uses: 1) 

\end{definition}
%---------%
%---------%
\begin{lemma}
  \label{}
  \uses{}
\end{lemma}
\begin{proof}
  % TBW
\end{proof}
%---------%
%---------%
\fi