\begin{proposition}%[Proposition 4.1 in \cite{Kemp2013RMTNotes}]
  \label{prop:matrix_moments_convergence}
  \notready
  %\uses{}
  Let $\{Y_{ij}\}_{1\le i\le j}$ be independent random variables, with $\{Y_{ii}\}_{i\ge 1}$ identically distributed and $\{Y_{ij}\}_{1\le i<j}$ identically distributed.  Suppose that $r_k = \max\{\bE(|Y_{11}|^k),\bE(|Y_{12}|^k)\} <\infty$ for each $k\in\bN$.  Suppose further than $\bE(Y_{ij})=0$ for all $i,j$ and set $t=\bE(Y_{12}^2)$.  If $i>j$, define $Y_{ij} \equiv Y_{ji}$, and let $\mathbf{Y}_n$ be the $n\times n$ matrix with $[\mathbf{Y}_n]_{ij} = Y_{ij}$ for $1\le i,j\le n$.  Let $\mathbf{X}_n = n^{-1/2}\mathbf{Y}_n$ be the corresponding Wigner matrix.  Then
\[
\lim_{n\to\infty} \frac{1}{n}\bE\Tr(\mathbf{X}_n^k) = \begin{cases}
  t^{k/2}C_{k/2}, & k\text{ even} \\
  0, & k\text{ odd}
\end{cases}.
\]
\end{proposition}

\begin{proof}
\notready
Proof
\end{proof}




\begin{proposition}%[Exercise 4.3.1 in \cite{Kemp2013RMTNotes}]
  \label{prop:vertex_edge_inequality}
  \notready
  %\uses{}
  Let $G=(V,E)$ be a connected finite graph. Then, $|G|=\#V\le \#E+1$.
\end{proposition}

\begin{proof}
  \notready
  $|G|= \#V\le \#E+1$: proof by induction on $\#V$. Base case $\#V = 1$ is obvious. For each additional vertex, the number of edges must increase by at least one for the graph to remain connected.
\end{proof}

\begin{proposition}
  \label{prop:vertex_edge_tree_equality}
  \notready
  Let $G=(V,E)$ be a connected finite graph. Then, $|G|=\#V=\#E+1$ if and only if $G$ is a plane tree.
  \uses{prop:vertex_edge_inequality}
\end{proposition}

\begin{proof}
  \notready
  $|G|=\#V=\#E+1$ if $G$ is a plane tree is already in Lean: SimpleGraph.IsTree.card\_edgeFinset.

  $G$ is a plane tree if $|G|=\#V=\#E+1$: proof by induction on $\#V$. Base case $\#V = 1$ has no edges. Assume $\#V = \#E + 1$ for $\#V = k$. Now, consider a tree with $\#V = k+1$ nodes. Removing a leaf node leaves us with a tree with $\#V = k$ nodes. By IH, there are $k - 1$ edges. So including the leaf node gives us $k$ edges.
\end{proof}


\begin{lemma}
  \label{lemma:vertex_bound}
  \notready
  \uses{prop:vertex_edge_inequality} %uses: also include that # edges < k/2
  For any graph $G = (V, E)$ appearing in the sum in Equation 4.8, $|G| \le k/2 + 1$.
\end{lemma}

\begin{proof}
  \notready
  Follows directly from earlier lemmas (replacing $\#E$ with $k/2$).
\end{proof}



\begin{lemma}
  \label{lemma:asc_factorial_product}
  \notready
  $n(n-1)\cdots (n-|G|+1) \le n^{|G|}$.

\end{lemma}

\begin{proof}
  \notready
  Use Nat.ascFactorial\_eq\_div. Or prove directly.
\end{proof}



\begin{lemma}
  \label{lemma:bounded_map}
  \notready
  \uses{lemma:asc_factorial_product, lemma:vertex_bound} %uses: big statement from richard
  The sequence $n\mapsto \frac1n\bE\Tr(X_n^k)$ is bounded.
\end{lemma}

\begin{proof}
  \notready
  The only part that depends on $n$ is the big fraction. Since we only care about $w \ge 2$, $|G| \le k/2 + 1$. Use the fact that the product $n(n-1)\cdots (n-|G|+1)$ is asymptotically equal to $n^{|G|}$ to conclude that the fraction is bounded and doesn't explode.
\end{proof}

\begin{lemma}
  \label{lemma:odd_vertex_bound}
  \notready
  \uses{lemma:vertex_bound}
  Suppose $k$ odd. Then, $|G| \le \frac{k}{2} + \frac{1}{2}$.
\end{lemma}

\begin{proof}
  \notready
  %prove
\end{proof}

\begin{lemma}
  \label{lemma:odd_ratio_bound}
  \notready
  \uses{lemma:odd_vertex_bound, lemma:asc_factorial_product}
  Suppose $k$ odd. Then, $\frac{n(n-1)\cdots(n-|G|+1)}{n^{k/2+1}} \le \frac{1}{\sqrt{n}}$.
\end{lemma}

\begin{proof}
  \notready
  %prove
\end{proof}




\begin{proposition}
  \label{prop:odd_case}
  \notready
  \uses{lemma:odd_ratio_bound} %uses: big statement from richard
  Suppose $k$ odd. Then, $\lim_{n\to\infty} \bE\Tr(X_n^k) = 0$.
\end{proposition}

\begin{proof}
  \notready
  Since $|G|\le \#E+1 \le k/2+1$ and $|G|$ is an integer, it follows that $|G|\le (k-1)/2+1 = k/2 + 1/2$.  Hence, in this case, all the terms in the (finite $n$-independent) sum in Equation 4.8 are $O(n^{-1/2})$
\end{proof}

\begin{lemma}
  \label{lemma:edge_bound_large_w}
  \notready
  %\uses{}
  If $k$ is even and there exists $e$ such that $w(e) \ge 3$, then $\#E \le \frac{k-1}{2}$.
\end{proposition}

\begin{proof}
  \notready
  %proof
\end{proof}



\begin{proposition}%[Proposition 4.4 in \cite{Kemp2013RMTNotes}]
  \label{prop:g_bound_self_edge}
  \notready
  \uses{prop:vertex_edge_inequality, prop:vertex_edge_tree_equality} %uses: richard's E <= k/2
  Let $(G,w)\in\mathcal{G}_k$ with $w\ge 2$, and suppose $k$ is even. If there exists a self-edge $e\in E_s$ in $G$, then $|G|\le k/2$.
\end{proposition}

\begin{proof}
  \notready
  Since the graph $G = (V,E)$ contains a loop, it is not a tree; it follows from Exercise \ref{ex V < E+1} that $\#V < \#E+1$.  But $w\ge 2$ implies that $\#E\le k/2$, and so $\#V < k/2+1$, and so $|G| = \#V \le k/2$.
\end{proof}

\begin{proposition}%[Proposition 4.4 in \cite{Kemp2013RMTNotes}]
  \label{prop:g_bound_large_w}
  \notready
  \uses{prop:vertex_edge_inequality, prop:vertex_edge_tree_equality, lemma:edge_bound_large_w}
  Let $(G,w)\in\mathcal{G}_k$ with $w\ge 2$, and suppose $k$ is even. If there exists an edge $e$ in $G$ with $w(e)\ge 3$, then $|G|\le k/2$.
\end{proposition}

\begin{proof}
  \notready
  The sum of $w$ over all edges $E$ in $G$ is $k$.  Hence, the sum of $w$ over $E\setminus\{e\}$ is $\le k-3$.  Since $w\ge 2$, this means that the number of edges excepting $e$ is $\le (k-3)/2$; hence, $\#E \le (k-3)/2+1 = (k-1)/2$.  By the result of a previous lemma, this means that $\#V \le (k-1)/2+1 = (k+1)/2$.  Since $k$ is even, it follows that $|G|=\#V \le k/2$.
\end{proof}




\begin{definition}
  \label{def:special_set_g}
  \notready
  %\uses{def:}
  Let $\mathcal{G}^{k/2+1}_k$ to be the set of pairs $(G,w)\in\mathcal{G}_k$ where $G$ has $k/2+1$ vertices, contains no self-edges, and the walk $w$ crosses every edge exactly $2$ times.
\end{definition}


\begin{lemma}
  \label{lemma:graph_set_finite}
  \notready
  %\uses{} %uses: definition of G_k
  $|G_k|$ is finite.
\end{lemma}

\begin{proof}
  \notready
  Follows from definition.
\end{proof}


\begin{lemma}
  \label{lemma:special_g_tree}
  \notready
  \uses{def:special_set_g}
  Elements of $G_k^{k/2+1}$ are trees.
\end{lemma}

\begin{proof}
  \notready
  %proof
\end{proof}

\begin{lemma}
  \label{lemma:special_g_edge_count}
  \notready
  \uses{def:special_set_g}
  Elements of $G_k^{k/2+1}$ have $|E| = k/2$.
\end{lemma}

\begin{proof}
  \notready
  %proof
\end{proof}

\begin{lemma}
  \label{lemma:special_g_vertex_count}
  \notready
  \uses{def:special_set_g, lemma:special_g_edge_count, lemma:special_set_g, prop:vertex_edge_tree_equality}
  Elements of $G_k^{k/2+1}$ have $|V| = k/2 + 1$.
\end{lemma}

\begin{proof}
  \notready
  Follows directly from dependency graph.
\end{proof}



\begin{proposition}
  \label{prop:g_difference_bound}
  \notready
  \uses{lemma:graph_set_finite, def:special_set_g, lemma:asc_factorial_product}
  $|\sum_{\mathcal{G}_{k}, w \ge 2} - \sum_{\mathcal{G}_k^{k/2+1}}| \le |\mathcal{G}_k|/n$.
\end{proposition}

\begin{proof}
  \notready
  %proof
\end{proof}



\begin{proposition}%[Equation 4.9 in \cite{Kemp2013RMTNotes}]
  \label{prop:trace_ev_special_g}
  \notready
  \uses{prop:g_difference_bound} %uses: richard big formula
  $\frac1n\bE\Tr(X_n^k) = \sum_{(G,w)\in\mathcal{G}^{k/2+1}_k} \Pi(G,w) \cdot \frac{n(n-1)\cdots(n-|G|+1)}{n^{k/2+1}} + O_k(n^{-1})$
  %edit: remove big O notation
\end{proposition}

\begin{proof}
  \notready
  If $|G| < k/2 + 1$, then there is at least one more $n$ in the denominator than the numerator.
\end{proof}

\begin{proposition}
  \label{lemma:fraction_limit_one}
  \notready
  \uses{lemma:asc_factorial_product} %uses some lower bound stuff?
  $\lim_{n\to\infty}\frac{n^{k/2}}{n(n-1)\cdots(n-k/2+1)} = 1$.
\end{proposition}

\begin{proof}
  \notready
  %proof
  some lower bound stuff + other stuff?
\end{proof}


\begin{proposition}
  \label{prop:trace_ev_limit_equals_sum}
  \notready
  \uses{def:special_set_g, prop:g_difference_bound, lemma:fraction_limit_one}
  $\lim_{n\to\infty}\bE\Tr(X_n^k) = \sum_{(G,w)\in\mathcal{G}^{k/2+1}_k} \Pi(G,w)$
\end{proposition}

\begin{proof}
  \notready
  Proof: use the fact that $|G|=k/2+1$ and $n(n-1)\cdots(n-k/2+1) \sim n^{k/2+1}$. Limit as n approaches infinity of $O(n^{-1/2})$ is 0.
\end{proof}




\begin{proposition}%[Equation 4.10 in \cite{Kemp2013RMTNotes}]
  \label{prop:product_g_w_to_exponential}
  \notready
  \uses{lemma:special_g_tree, lemma:special_g_edge_count, lemma:special_g_vertex_count, prop:vertex_edge_tree_equality, prop:g_bound_self_edge, prop:g_bound_large_w} %uses: defn of t, defn of pi(G,w)
  $\Pi(G,w) = \prod_{e_c\in E^c} \bE(Y_{12}^{w(e_c)}) = \prod_{e_c\in E^c} \bE(Y_{12}^2) = t^{\#E} = t^{k/2}$.
\end{proposition}

\begin{proof}
  \notready
  Let $(G,w)\in\mathcal{G}^{k/2+1}_k$.  Since $w$ traverses each edge exactly twice, the number of edges in $G$ is $k/2$.  Since the number of vertices is $k/2+1$, Exercise (Prop) 4.3.1 shows that $G$ is a tree.  In particular there are no self-edges (as we saw already in Proposition 4.4).

  1st equality: definition right after equation 4.4 in notes
  2nd equality: proposition 4.4 (w < 3) and previous lemma (w = 1 --> 0)
  3nd equality: definition (from main proposition)
  4th equality: number of edges is k/2.
\end{proof}




\begin{proposition}%[Proposition 4.11 in \cite{Kemp2013RMTNotes}]
  \label{prop:limit_stuff}
  \notready
  \uses{prop:trace_ev_limit_equals_sum, prop:product_g_w_to_exponential}
  $\lim_{n\to\infty} \bE\Tr(X_n^k) = t^{k/2}\cdot\#\mathcal{G}_k^{k/2+1}$
\end{proposition}

\begin{proof}
  \notready
  Follows directly from 4.7.5 and 4.8.
\end{proof}
