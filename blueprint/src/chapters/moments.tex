\begin{proposition}[Proposition 4.1 in \cite{Kemp2013RMTNotes}]
  \label{prop:matrix_moments_convergence}
  \notready
  \uses{def:}
  Let $\{Y_{ij}\}_{1\le i\le j}$ be independent random variables, with $\{Y_{ii}\}_{i\ge 1}$ identically distributed and $\{Y_{ij}\}_{1\le i<j}$ identically distributed.  Suppose that $r_k = \max\{\bE(|Y_{11}|^k),\bE(|Y_{12}|^k)\} <\infty$ for each $k\in\bN$.  Suppose further than $\bE(Y_{ij})=0$ for all $i,j$ and set $t=\bE(Y_{12}^2)$.  If $i>j$, define $Y_{ij} \equiv Y_{ji}$, and let $\mathbf{Y}_n$ be the $n\times n$ matrix with $[\mathbf{Y}_n]_{ij} = Y_{ij}$ for $1\le i,j\le n$.  Let $\mathbf{X}_n = n^{-1/2}\mathbf{Y}_n$ be the corresponding Wigner matrix.  Then
\[
\lim_{n\to\infty} \frac{1}{n}\bE\Tr(\mathbf{X}_n^k) = \begin{cases}
  t^{k/2}C_{k/2}, & k\text{ even} \\
  0, & k\text{ odd}
\end{cases}.
\]
\end{proposition}

\begin{proof}
\notready
Proof
\end{proof}



\begin{lemma}\label{lem:trace_smul}
  \mathlibok 
  \lean{Matrix.trace_smul}
  Let $R$ be a ring with a monoid action from $\alpha$ (i.e., $\alpha$ acts distributively on $R$).
  For any scalar $r \in \alpha$ and any square matrix $A$ over $R$, the trace of the scalar
  multiple $r \cdot A$ equals the scalar multiple of the trace of $A$, i.e.,
  \[ \text{tr}(r \cdot A) = r \cdot \text{tr}(A). \]
  \end{lemma}

  \begin{proof}\leanok 
  \end{proof}




%---Richard's Part for Proposition 4.1 (Start)---% 
%---------% 
\begin{lemma}
  \label{lem:graph_walk_le_k}
  \uses{def:graph_multi_index,def:graph_path,def:graph_edge_count}
  For any $k$-index $\mathbf{i}$, the connected graph $G_\mathbf{i}$ has at most $k$ vertices. Furthermore
  \[
  |w_\mathbf{i}| \equiv \sum_{e \in E_\mathbf{i}} w_\mathbf{i}(e) = k.
  \]
\end{lemma}
\begin{proof}
  Foremost, since the number of vertices of the graph $G_\mathbf{i}$ 
  are the number of distinct elements of the $k$-index $\mathbf{i}$, it clearly follows that $|G| \leq k$.
  On the other hand, given an edge $e \in E_\mathbf{i}$, by Definition \ref{def:graph_edge_count}, $w_i(e)$ as the number of times the walk $w_\mathbf{i}$ traversed $e$. 
  Since $|w_\mathbf{i}| = k$ by construction, it follows that
  \[
  |w_\mathbf{i}| \equiv \sum_{e \in E_\mathbf{i}} w_\mathbf{i}(e) = k.
  \]
\end{proof}
%---------%
% Motivated by these conditions, we define $\mathcal{G}_k$:
%---------%
\begin{definition}
  \label{def:g_k}
  \uses{def:graph_multi_index,def:graph_path,def:graph_edge_count}
  Let $\mathcal{G}_k$ denote the set of all ordered pairs $(G,w)$ where $G = (V,E)$ is a connected graph with at most $k$ vertices, and
  $w$ is a closed walk covering $G$ satisfying $|w| = k$.
\end{definition}
%---------%
% We can count the set of $k$-indexes in Equation \ref{* *}: 
% For any $(G,w) \in \mathcal{G}_k$, an index with that corresponding graph $G$ and walk $w$ is completely determined by assigning which distinct values of $[n]$ appear at the vertices of $G$:
%---------%
% The definition below proposes an alternate definition of $\mathcal{G}_k$ along with the equivalence relation. 
%\begin{definition}
%  \label{def:g_k}
%  \notready
%  \uses{}
%  %We propose an alternate (and perhaps more explicit) construction of $\mathcal{G}_k$. 
%  We define $\mathcal{G}_k$ as the set of equivalence classes of the set $\{ (G_\mathbf{i},w_\mathbf{i}) : \mathbf{i} \in [n]^k \text{ and } n \in \mathbb{N} \}$, 
%  where the equivalence relation is defined as: $(G_\mathbf{i},w_\mathbf{i}) \sim (G_{\mathbf{i}^*},w_{\mathbf{i}^*})$ 
%  if and only if there exists a bijection $\varphi$ from the set of entries $\mathbf{i}$ onto the set of entries $\mathbf{i}^*$ such that
%  \[
%  \mathbf{i} = (i_1,...,i_k) \,\, \Longleftrightarrow \,\, \mathbf{i}^* = \bigl( \varphi(i_1),\varphi(i_2),...,\varphi(i_k) \bigl).
%  \]
%  We denote an element of $\mathcal{G}_k$ as $(G,w)$.
%\end{definition}
%---------%
\begin{lemma}
  \label{lem:equal_equiv_class_equal_expectation}
  \notready
  \uses{def:graph_multi_index,def:graph_path,def:matrix_multi_index}
  Given two $k$-indexes $\mathbf{i} = (i_1,...,i_k)$ and $\mathbf{i}^* = (j_1,...,j_k)$, 
  suppose there exists a bijection $\varphi$ from the set of entries $\mathbf{i}$ onto the set of entries $\mathbf{i}^*$ such that
  \[
  \mathbf{i} = (i_1,...,i_k) \,\, \Longleftrightarrow \,\, \mathbf{i}^* = \bigl( \varphi(i_1),\varphi(i_2),...,\varphi(i_k) \bigl).
  \]
  Then $\bE (Y_\mathbf{i}) = \bE (Y_{\mathbf{i}^*})$.
\end{lemma}
\begin{proof}
  It suffices to show that $Y_\mathbf{i}$ and $Y_{\mathbf{i}^*}$ represent the same number of, respectively, self-edges and connecting edges.
  Given $Y_\mathbf{i} = Y_{i_1 i_2}Y_{i_2 i_3} \cdots Y_{i_{k-1} i_k}Y_{i_k i_1}$,
  \[
  Y_{\mathbf{i}^*} = Y_{j_1 j_2}Y_{j_2 j_3} \cdots Y_{j_{k-1} j_k}Y_{j_k j_1} 
  = Y_{\varphi(i_1) \varphi(i_2)}Y_{\varphi(i_2) \varphi(i_3)} \cdots Y_{\varphi(i_{k-1}) \varphi(i_k)}Y_{\varphi(i_k) \varphi(i_1)}.
  \] 
  The fact that $\{ i_{\lambda_l},i_{\lambda_{l+1}} \}$ is a self-edge 
  if (i.e. a singleton) and only if $\{ \varphi(i_{\lambda_l}),\varphi(i_{\lambda_{l+1}}) \}$ is a self-edge completes the proof.
\end{proof}
%---------%
%\begin{lemma}
%  \label{lem:equal_equiv_class_equal_expectation}
%  \notready
%  \uses{}
%  %\uses:
%  %(1)Definition of $Y_\mathbf{i}$, 
%  Given two $k$-indexes $\mathbf{i} = (i_1,...,i_k)$ and $\mathbf{i}^* = (j_1,...,j_k)$, 
%  suppose there exists a bijection $\varphi$ from the set of entries $\mathbf{i}$ onto the set of entries $\mathbf{i}^*$ such that
%  \[
%  \mathbf{i} = (i_1,...,i_k) \,\, \Longleftrightarrow \,\, \mathbf{i}^* = \bigl( \varphi(i_1),\varphi(i_2),...,\varphi(i_k) \bigl).
%  \]
%  Then $\bE (Y_\mathbf{i}) = \bE (Y_{\mathbf{i}^*})$.
%\end{lemma}
%\begin{proof}
%  It suffices to show that $Y_\mathbf{i}$ and $Y_{\mathbf{i}^*}$ represent the same number of, respectively, self-edges and connecting edges.
%  Given $Y_\mathbf{i} = Y_{i_1 i_2}Y_{i_2 i_3} \cdots Y_{i_{k-1} i_k}Y_{i_k i_1}$,
%  \[
%  Y_{\mathbf{i}^*} = Y_{j_1 j_2}Y_{j_2 j_3} \cdots Y_{j_{k-1} j_k}Y_{j_k j_1} 
%  = Y_{\varphi(i_1) \varphi(i_2)}Y_{\varphi(i_2) \varphi(i_3)} \cdots Y_{\varphi(i_{k-1}) \varphi(i_k)}Y_{\varphi(i_k) \varphi(i_1)}.
%  \] 
%  The fact that $\{ i_{\lambda_l},i_{\lambda_{l+1}} \}$ is a self-edge 
%  if (i.e. a singleton) and only if $\{ \varphi(i_{\lambda_l}),\varphi(i_{\lambda_{l+1}}) \}$ is a self-edge completes the proof.
%\end{proof}
%---------%
\begin{lemma}[Lemma 4.3 in \cite{Kemp2013RMTNotes}]
  \label{lem:lem_4.3}
  \uses{def:g_k}
  % Perhaps will need to explicitly lay out the equivalence relation.
  Given $(G,w) \in \mathcal{G}_k$, denote by $|G|$ the number of vertices in $G$. Then
  \[
  \# \{ \mathbf{i} \in [n]^k : (G_\mathbf{i},w_\mathbf{i}) = (G,w) \} = n (n-1) \cdots (n - |G| + 1).
  \]
\end{lemma}
\begin{proof}
  By the way the equivalence relation is defined in Definition \ref{def:g_k}, 
  the fact that there are $n (n - 1) \cdots (n -|G| + 1)$ ways to assign $|G|$ distinct values from $[n]$ into the indices $i_1,...,i_{|G|}$ completes the proof.
\end{proof}
%---------%
% Using Equation \ref{* *}, we can re-index the sum of Equation \ref{* *} as
%---------%
\begin{lemma}
  \label{lem:equation_4.5_1}
  \uses{def:g_k,lem:trace_expectation_of_matrix}
  \[
  \bE \Tr (\mathbf{Y}_\mathbf{i}^k) = \sum_{(G,w) \in \mathcal{G}_k} \sum_{\substack{\mathbf{i} \in [n]^k \\ (G_\mathbf{i},w_\mathbf{i}) = (G,w)}} \bE (Y_\mathbf{i}).
  \]
\end{lemma}
\begin{proof}
  Given an ordered pair $(G_\mathbf{i},w_\mathbf{i})$ generated by a $k$-index $\mathbf{i}$ and $(G,w) \in \mathcal{G}_k$, 
  only one of $(G_\mathbf{i},w_\mathbf{i}) = (G,w)$ or $(G_\mathbf{i},w_\mathbf{i}) \neq (G,w)$ holds. 
  Hence, the partition of the summation is well-defined.
\end{proof}
%---------%
\begin{lemma}
  \label{lem:equation_4.5_2}
  \uses{lem:lem_4.3,lem:equal_equiv_class_equal_expectation,lem:trace_expectation_of_matrix,lem:expectation_matrix_multi_index}
  \[
  \bE \Tr (\mathbf{Y}_\mathbf{i}^k) = \sum_{(G,w) \in \mathcal{G}_k} \Pi (G,w) \cdot \# \{ \mathbf{i} \in [n]^k : (G_\mathbf{i},w_\mathbf{i}) = (G,w) \}.
  \]
\end{lemma}
\begin{proof}
  This follows from re-indexing the sum of Lemma \ref{lem:trace_expectation_of_matrix} by using Lemma \ref{lem:expectation_matrix_multi_index}.
\end{proof}
%---------%
\begin{lemma}
  \label{lem:equation_4.5_3}
  \uses{lem:equation_4.5_1,lem:equation_4.5_2}
  \[
  \frac{1}{n} \bE \Tr (\mathbf{X}_n^k) = \sum_{(G,w) \in \mathcal{G}_k} \Pi (G,w) \cdot \frac{n (n-1) \cdots (n - |G| + 1)}{n^{k/2+1}}.
  \]
\end{lemma}
\begin{proof}
  Combining with the renormalization factor $n^{-1}$ of Proposition \ref{prop:matrix_moments_convergence}
  \[
  \frac{1}{n} \bE \Tr (\mathbf{X}_n^k) = \frac{1}{n^{k/2+1}} \bE \Tr (\mathbf{Y}_\mathbf{i}^k).
  \]
  Substituting the term $\bE \Tr (\mathbf{Y}_\mathbf{i}^k)$ with the expression in Equation \ref{lem:equation_4.5_2}
  \[
  \frac{1}{n} \bE \Tr (\mathbf{X}_n^k) = \sum_{\substack{(G,w) \in \mathcal{G}_k}} \Pi (G,w) \cdot \frac{n (n-1) \cdots (n - |G| + 1)}{n^{k/2+1}}.
  \]
\end{proof}
% Note that the summation is finite, and thus we only need to determine the values of $\Pi (G,w)$ to evaluate the summation. 
%---------%
\begin{definition}
  \label{def:g_k_ge_2}
  \uses{def:g_k}
  Let $\substack{\mathcal{G}_k \\ w \geq 2}$ be a subset of $\mathcal{G}_k$ 
  in which the walk $w$ traverses each edge at least twice.
\end{definition}
%---------%
%\begin{definition}
%  \label{def:g_k_ge_2}
%  \uses{def:g_k}
%  Given an equivalence class $(G,w) \in \mathcal{G}_k$, 
%  let $\substack{\mathcal{G}_k \\ w \geq 2}$ be a subset of $\mathcal{G}_k$ 
%  in which the walk $w_\mathbf{i}$ represetned by every $(G_\mathbf{i},w_\mathbf{i}) \in (G,w)$ 
%  crosses each edge at least twice.
%\end{definition}
%---------%
%\begin{definition}
%  \label{def:Pi.graph}
%  \uses{def:g_k}
%  %uses the definition of $\Pi (G_\mathbf{i},w_\mathbf{i})$.
%  Given an equivalence class $(G,w) \in \mathcal{G}_k$, let 
%  \[
%  \Pi (G,w) = \Pi (G_\mathbf{i},w_\mathbf{i})
%  \]
%  where $(G_\mathbf{i},w_\mathbf{i}) \in (G,w)$.
%\end{definition}
%---------%
%\begin{lemma}
%  \label{lem:g_k_ge_2_wd}
%  \uses{def:g_k_ge_2,lem:equal_equiv_class_equal_expectation}
%  The set $\substack{\mathcal{G}_k \\ w \geq 2}$ in Definition \ref{def:g_k_ge_2} is well-defined.
%  Furthermore, the common value $\Pi (G,w)$ in Definition \ref{def:Pi.graph} is well-defined.  
%\end{lemma}
%\begin{proof}
%  The proof for both statements follows an identical reasoning as in the proof of Lemma \ref{lem:equal_equiv_class_equal_expectation}.
%\end{proof}
%---------%
\begin{lemma}
  \label{lem:Pi.prod_eq_zero_if_w_le_two}
  \uses{def:g_k,prop:matrix_moments_convergence,def:matrix_multi_index,def:expectation_matrix_multi_index}
  Given an ordered pair $(G,w) \in \mathcal{G}_k$, suppose there exists an edge $e \in E_\mathbf{i}$ in which it is traversed only once in the walk $w$. Then
  \[
  \Pi (G,w) = 0.
  \]
\end{lemma}
\begin{proof}
  This directly follows from the assumption of Proposition \ref{prop:matrix_moments_convergence} that $\bE (Y_{ij}) = 0$ for every $i$ and $j$. 
\end{proof}
%---------%
%\begin{lemma}
%  \label{lem:Pi.prod_eq_zero_if_w_le_two}
%  \uses{def:g_k,prop:matrix_moments_convergence}
%  %uses definition of $Y_\mathbf{i}$ AND $\Pi (G,w)$.
%  Given an ordered pair $(G_\mathbf{i},w_\mathbf{i})$, suppose there exists an edge $e \in E_\mathbf{i}$ in which it is traversed only once in the walk $w_\mathbf{i}$.
%  Then
%  \[
%  \Pi (G,w) = 0.
%  \]
%\end{lemma}
%\begin{proof}
%  This directly follows from the assumption of Proposition \ref{prop:matrix_moments_convergence} that $\bE (Y_{ij}) = 0$ for every $i$ and $j$. 
%\end{proof}
%---------%
\begin{lemma}
  \label{lem:equation_4.8}
  \uses{def:g_k_ge_2,lem:equation_4.5_3,lem:Pi.prod_eq_zero_if_w_le_two,lem:expectation_matrix_multi_index,prop:matrix_moments_convergence}
  \[
  \frac{1}{n} \bE \Tr (\mathbf{X}_n^k) 
  = \sum_{(G,w) \in \substack{\mathcal{G}_k \\ w \geq 2}} \Pi (G,w) \cdot \frac{n (n-1) \cdots (n - |G| + 1)}{n^{k/2+1}}.
  \]
\end{lemma}
\begin{proof}
  Let $(G,w) \in \mathcal{G}_k$ and suppose there exists an edge $e = \{i,j\}$ such that $w(e) = 1$.
  This means, in LEmma \ref{lem:expectation_matrix_multi_index}, a singleton term $\bE (Y_{ij}^{w(e)}) = \bE (Y_{ij})$ appears.
  Following the condition of Proposition \ref{prop:matrix_moments_convergence}, the fact that the variables $Y_{ij}$ are all centered implies the product $\Pi (G,w) = 0$ for any such pair $(G,w)$.
  Thus, we only need to consider those $w$ that cross each edge at least twice: 
  \[
  \frac{1}{n} \bE \Tr (\mathbf{X}_n^k) = \sum_{\substack{(G,w) \in \mathcal{G}_k}} \Pi (G,w) \cdot \frac{n (n-1) \cdots (n - |G| + 1)}{n^{k/2+1}}  = \sum_{(G,w) \in \substack{\mathcal{G}_k \\ w \geq 2}} \Pi (G,w) \cdot \frac{n (n-1) \cdots (n - |G| + 1)}{n^{k/2+1}}.
  \]
\end{proof}
%---------%
\begin{lemma}
  \label{lem:edge_set_order_leq_k_over_two}
  \uses{def:g_k_ge_2,lem:graph_walk_le_k}
  Given an ordered pair $(G_\mathbf{i},w_\mathbf{i})$, if $w_\mathbf{i} \geq 2$, then $\# E_\mathbf{i} \leq k/2$.
\end{lemma}
\begin{proof}
  Since $|w_\mathbf{i}| = k$, if each edge in $G_\mathbf{i}$ is traversed at least twice, then by construction the number of edges is at most $k/2$.
\end{proof}
%---------%
% This lemma might be merged with the following one(s).
% If using the equivalence definition of $\mathcal{G}_k$, it might be more convenient to introduce the Axiom of Choice to ease the notation on $(G,w)$.
%---------%
%---Richard's Part for Proposition 4.1 (End)---% 





%---Richard's Part for Proposition 4.2 (Start)---% 
%---------%
%---------%
\begin{definition}[R-2-1]
  \label{}% Common value
  \uses{}% Uses: 1) Definition of $Y_\mathbf{i}$; 2) Definition of the ordered triple 
  Let us denote
  \[
  \pi(G_{\mathbf{i}\#\mathbf{j}},w_\mathbf{i},w_\mathbf{j}) = \mathbb{E}(Y_\mathbf{i} Y_\mathbf{j}) - \mathbb{E}(Y_\mathbf{i}) \mathbb{E}(Y_\mathbf{j}).
  \]
\end{definition}
%---------%
%---------%
\begin{definition}[R-2-2]
  \label{g_k.k}% $\mathcal{G}_{k,k}$
  \uses{}% Uses: 1) (Potentially) uses the definition of the ordered triple 
  We define $\mathcal{G}_{k,k}$ to be the set of connected graphs $G$ with $\leq 2k$ vertices, 
  together with two paths each of length $k$ whose union covers $G$.
\end{definition}
%---------%
%---------%
\begin{lemma}[R-2-3]
  \label{}% Common value lemma
  \uses{}% Uses: 1) Definition of the common value; 2) Definition of $Y_\mathbf{i}$; 3) Definition of the ordered triple 
  If the ordered triples $(G_{\mathbf{i}\#\mathbf{j}},w_\mathbf{i},w_\mathbf{j})$ and $(G_{\mathbf{i}^*\#\mathbf{j}^*},w_{\mathbf{i}^*},w_{\mathbf{j}^*})$ are `same up to relabeling', then
  \[
  \pi(G_{\mathbf{i}\#\mathbf{j}},w_\mathbf{i},w_\mathbf{j}) = \pi(G_{\mathbf{i}^*\#\mathbf{j}^*},w_{\mathbf{i}^*},w_{\mathbf{j}^*}).
  \]
\end{lemma}
\begin{proof}
  % TBW
\end{proof}
%---------%
%---------%
\begin{lemma}[R-2-4]
  \label{}% Sum relabeling 1
  \uses{}% Uses: 1) Common value lemma; 2) Previous variance equation (2nd eq of p.14) (Paul's) 
  \[
  \text{Var} \biggl(\frac{1}{n} \Tr (\mathbf{X}_n^k) \biggl) 
  = \frac{1}{n^{k+2}} \sum_{(G,w,w') \in \mathcal{G}_{k,k}} \sum_{\substack{\mathbf{i},\mathbf{j} \in [n]^k \\ (G_{\mathbf{i}\#\mathbf{j}},w_\mathbf{i},w_\mathbf{j}) = (G,w,w')}} [\mathbb{E}(Y_\mathbf{i} Y_\mathbf{j}) - \mathbb{E}(Y_\mathbf{i}) \mathbb{E}(Y_\mathbf{j})].
  \]
\end{lemma}
\begin{proof}
  Given an ordered triple $(G_{\mathbf{i}\#\mathbf{j}},w_\mathbf{i},w_\mathbf{j})$ generated by two $k$-indexes $\mathbf{i}$ and $\mathbf{j}$, and $(G,w,w') \in \mathcal{G}_{k,k}$, 
  only one of $(G_{\mathbf{i}\#\mathbf{j}},w_\mathbf{i},w_\mathbf{j}) = (G,w,w')$ or $(G_{\mathbf{i}\#\mathbf{j}},w_\mathbf{i},w_\mathbf{j}) \neq (G,w,w')$ holds. 
  Hence, the partition of the summation is well-defined.
\end{proof}
%---------%
%---------%
\begin{lemma}[R-2-5]
  \label{}% Sum relabeling 2
  \uses{}% Uses: 1) Common value lemma; 2) Sum relabeling 1; 3) Common value def; 
  \[
  \text{Var} \biggl(\frac{1}{n} \Tr (\mathbf{X}_n^k) \biggl) 
  = \frac{1}{n^{k+2}} \sum_{(G,w,w') \in \mathcal{G}_{k,k}} \pi(G,w,w')
\cdot \# \bigl\{ (\mathbf{i},\mathbf{j}) \in [n]^{2k} : (G_{\mathbf{i} \# \mathbf{j}},w_\mathbf{i},w_\mathbf{j}) = (G,w,w') \bigl\}.
  \]
\end{lemma}
\begin{proof}
  This follows from re-indexing the sum of Lemma \ref{} using Lemma \ref{}.
  % Lemma \ref{R-2-4} & Lemma \ref{R-2-3}
\end{proof}
%---------%
%---------%
\begin{definition}[R-2-6]
  \label{}% Self and connecting edge sets 
  \uses{}% Uses: 1) Definition of a glued graph (Paul) / ordered triple
  Given a graph $G_{\mathbf{i} \# \mathbf{j}}$, 
  let $E^s_{\mathbf{i} \# \mathbf{j}}$ denote the set of self-edges and $E^C_{\mathbf{i} \# \mathbf{j}}$ denote the set of connecting edges.
\end{definition}
%---------%
%---------%
\begin{definition}[R-2-7]
  \label{}% Path counting function
  \uses{}% Uses: 1) Definition of a glued graph (Paul) / ordered triple / path
  Given a graph $G_{\mathbf{i} \# \mathbf{j}}$, 
  let $w_{\mathbf{i} \# \mathbf{j}}(e)$ denote the number of times the edge $e$ is traversed by either of the two paths $w_\mathbf{i}$ and $w_\mathbf{j}$.
\end{definition}
%---------%
%---------%
\begin{lemma}[R-2-8]
  \label{}% Prod Prod form 1
  \uses{}% Uses: Equation 4.4 (Paul's)
  \[
  \mathbb{E} (Y_\mathbf{i}Y_\mathbf{j}) 
  = \prod_{e_s \in E^s_{\mathbf{i} \# \mathbf{j}}} \mathbb{E} (Y_{11}^{w_{\mathbf{i} \# \mathbf{j}}(e_s)}) \cdot \prod_{e_c \in E^c_{\mathbf{i} \# \mathbf{j}}} \mathbb{E} (Y_{12}^{w_{\mathbf{i} \# \mathbf{j}}(e_c)}).
  \]
\end{lemma}
\begin{proof}
  This follows from Lemma \ref{} and the independency of random variables.
  % Equation 4.4 (Paul's)
\end{proof}
%---------%
%---------%
\begin{lemma}[R-2-9]
  \label{}% Prod Prod form 2
  \uses{}% Uses: Equation 4.4 (Paul's)
  \[
  \mathbb{E}(Y_\mathbf{i}) \mathbb{E}(Y_\mathbf{j}) 
  = \prod_{e_s \in E^s_{\mathbf{i}}} \mathbb{E} (Y_{11}^{w_{\mathbf{i}}(e_s)}) \cdot \prod_{e_c \in E^c_{\mathbf{i}}} \mathbb{E} (Y_{12}^{w_{\mathbf{i}}(e_c)})
  \cdot \prod_{e_s \in E^s_{\mathbf{j}}} \mathbb{E} (Y_{11}^{w_{\mathbf{j}}(e_s)}) \cdot \prod_{e_c \in E^c_{\mathbf{j}}} \mathbb{E} (Y_{12}^{w_{\mathbf{j}}(e_c)}).
  \]
\end{lemma}
\begin{proof}
  This directly follows from Lemma \ref{}.
  % Equation 4.4 (Paul's)
\end{proof}
%---------%
%---------%
\begin{lemma}[R-2-10]
  \label{}% Prod Prod form total
  \uses{}% Uses: 1) Equation 4.4 (Paul's); 2) Prod Prod form 1; 3) Prod Prod form 2
  \[
  \mathbb{E} (Y_\mathbf{i}Y_\mathbf{j}) -  \mathbb{E}(Y_\mathbf{i}) \mathbb{E}(Y_\mathbf{j}) 
  = \prod_{e_s \in E^s_{\mathbf{i} \# \mathbf{j}}} \mathbb{E} (Y_{11}^{w_{\mathbf{i} \# \mathbf{j}}(e_s)}) \cdot \prod_{e_c \in E^c_{\mathbf{i} \# \mathbf{j}}} \mathbb{E} (Y_{12}^{w_{\mathbf{i} \# \mathbf{j}}(e_c)})
  -  \prod_{e_s \in E^s_{\mathbf{i}}} \mathbb{E} (Y_{11}^{w_{\mathbf{i}}(e_s)}) \cdot \prod_{e_c \in E^c_{\mathbf{i}}} \mathbb{E} (Y_{12}^{w_{\mathbf{i}}(e_c)})
  \cdot \prod_{e_s \in E^s_{\mathbf{j}}} \mathbb{E} (Y_{11}^{w_{\mathbf{j}}(e_s)}) \cdot \prod_{e_c \in E^c_{\mathbf{j}}} \mathbb{E} (Y_{12}^{w_{\mathbf{j}}(e_c)}).
  \]
\end{lemma}
\begin{proof}
  This follows from substituting the expressions derived in Lemma \ref{} and Lemma \ref{}.
  % Prod prod form 1 & 2
\end{proof}
%---------%
%---------%
\begin{lemma}[R-2-11]
  \label{} % Couting path adds up to $2k$
  \uses{} % Uses: 1) Defintiion of the counting function; 2) Definition of the glued graph; 3) Definition of the graph Gi
  \[
  \sum_{e \in E_{\mathbf{i} \# \mathbf{j}}} w_{\mathbf{i} \# \mathbf{j}}(e) 
  = 2k 
  = \sum_{e \in E_\mathbf{i}} w_{\mathbf{i}}(e) + \sum_{e \in E_\mathbf{j}} w_{\mathbf{j}}(e).
  \]
\end{lemma}
\begin{proof}
  By construction of the paths $w_\mathbf{i}$ and $w_\mathbf{j}$,
  \[
  \sum_{e \in E_\mathbf{i}} w_{\mathbf{i}}(e) = k = \sum_{e \in E_\mathbf{j}} w_{\mathbf{j}}(e).
  \]
  By definition of the counting function $w_{\mathbf{i} \# \mathbf{j}}(\cdot)$,
  \[
  \sum_{e \in E_{\mathbf{i} \# \mathbf{j}}} w_{\mathbf{i} \# \mathbf{j}}(e)  
  = \sum_{e \in E_\mathbf{i}} w_{\mathbf{i}}(e) + \sum_{e \in E_\mathbf{j}} w_{\mathbf{j}}(e)
  = 2k.
  \]
\end{proof}
%---------%
%---------%
\begin{lemma}[R-2-12]
  \label{} % Individual term bound 1.1
  \uses{} % Uses: 1) Definition of moments; 2) Equation 4.4 (Paul's)
  For any $k$-indexes $\mathbf{i}$ and $\mathbf{j}$, there exists $n \in \mathbb{N}$ such that
  \[
  \mathbb{E} (Y_{11}^{w_{\mathbf{i} \# \mathbf{j} (e)}}) \leq r_n
  \]
  for every $e \in E_{\mathbf{i} \# \mathbf{j}}$.
\end{lemma}
\begin{proof}
  % TBW
\end{proof}
%---------%
%---------%
\begin{lemma}[R-2-12]
  \label{} % Individual term bound 1.2
  \uses{} % Uses: 1) Definition of moments; 2) Equation 4.4 (Paul's)
  For any $k$-indexes $\mathbf{i}$ and $\mathbf{j}$, there exists $n \in \mathbb{N}$ such that
  \[
  \mathbb{E} (Y_{12}^{w_{\mathbf{i} \# \mathbf{j} (e)}}) \leq r_n
  \]
  for every $e \in E_{\mathbf{i} \# \mathbf{j}}$.
\end{lemma}
\begin{proof}
  % TBW
\end{proof}
%---------%
%---------%
\begin{lemma}[R-2-13]
  \label{} % Bound 1
  \uses{} % Uses: 1) Definition of moments; 2) Equation 4.4 (Paul's)
  For any $k$-indexes $\mathbf{i}$ and $\mathbf{j}$, there exists $M_1 \in \mathbb{R}$ such that
  \[
  \mathbb{E} (Y_\mathbf{i}Y_\mathbf{j}) \leq M_1.
  \]
\end{lemma}
\begin{proof}
  By Lemma \ref{}, we have
  % Prod prod form 1
  \[
  \mathbb{E} (Y_\mathbf{i}Y_\mathbf{j}) 
  = \prod_{e_s \in E^s_{\mathbf{i} \# \mathbf{j}}} \mathbb{E} (Y_{11}^{w_{\mathbf{i} \# \mathbf{j}}(e_s)}) \cdot \prod_{e_c \in E^c_{\mathbf{i} \# \mathbf{j}}} \mathbb{E} (Y_{12}^{w_{\mathbf{i} \# \mathbf{j}}(e_c)}).
  \]
  By Lemma \ref{}, we can construct the (crude) upper bound
  % R-2-12
  \[
  \mathbb{E} (Y_\mathbf{i}Y_\mathbf{j}) 
  = \prod_{e_s \in E^s_{\mathbf{i} \# \mathbf{j}}} \mathbb{E} (Y_{11}^{w_{\mathbf{i} \# \mathbf{j}}(e_s)}) \cdot \prod_{e_c \in E^c_{\mathbf{i} \# \mathbf{j}}} \mathbb{E} (Y_{12}^{w_{\mathbf{i} \# \mathbf{j}}(e_c)})
  \leq 2k \cdot M_1
  \]
\end{proof}
%---------%
%---------%
\begin{lemma}[R-2-14]
  \label{} % Individual term bound 2.1
  \uses{} % Uses: 1) Definition of moments; 2) Equation 4.4 (Paul's)
  For any $k$-index $\mathbf{i}$, there exists $n \in \mathbb{N}$ such that
  \[
  \mathbb{E} (Y_{11}^{w_{\mathbf{i} (e)}}) \leq r_n.
  \]
  for every $e \in E_{\mathbf{i} \# \mathbf{j}}$.
\end{lemma}
\begin{proof}
  % TBW
\end{proof}
%---------%
%---------%
\begin{lemma}[R-2-15]
  \label{} % Individual term bound 2.2
  \uses{} % Uses: 1) Definition of moments; 2) Equation 4.4 (Paul's)
  For any $k$-index $\mathbf{i}$, there exists $n \in \mathbb{N}$ such that
  \[
  \mathbb{E} (Y_{12}^{w_{\mathbf{i} (e)}}) \leq r_n.
  \]
  for every $e \in E_\mathbf{i}$.
\end{lemma}
\begin{proof}
  % TBW
\end{proof}
%---------%
%---------%
\begin{lemma}[R-2-16]
  \label{} % Individual term bound 2.2
  For any $k$-indexes $\mathbf{i}$, there exists $M_2 \in \mathbb{R}$ such that
  \[
  \mathbb{E} (Y_\mathbf{i}) \leq M_2.
  \]
\end{lemma}
\begin{proof}
  % TBW
\end{proof}
%---------%
%---------%
\begin{lemma}[R-2-17]
  \label{} % Bound Total
  \uses{} % Uses: Bound 1 & 2
  For any $k$-indexes $\mathbf{i}$ and $\mathbf{j}$, there exists $M_{2k} \in \mathbb{R}_{\geq 0}$ such that
  \[
  | \mathbb{E} (Y_\mathbf{i}Y_\mathbf{j}) -  \mathbb{E}(Y_\mathbf{i}) \mathbb{E}(Y_\mathbf{j}) |
  \leq 2 M_{2k}.
  \] 
\end{lemma}
\begin{proof}
  % TBW
\end{proof}
%---------%
%---------%
% \begin{lemma}
%  \label{}
%  \uses{}
%\end{lemma}
%\begin{proof}
%  % TBW
%\end{proof}
%---------%
%---------%
% We might not need the last commented-off lemma.
%---Richard's Part for Proposition 4.2 (End)---% 





\iffalse
%---------%
%---------%
\begin{definition}
  \label{}
  \uses{}%Uses: 1) 

\end{definition}
%---------%
%---------%
\begin{lemma}
  \label{}
  \uses{}
\end{lemma}
\begin{proof}
  % TBW
\end{proof}
%---------%
%---------%
\fi





%---Richard's Part for Semicircle Distribution (Start)---% 
%---------%
%---------%

%---------%
%---------%
%---Richard's Part for Semicircle Distribution (Start)---% 