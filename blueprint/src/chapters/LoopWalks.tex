\chapter{LoopWalks}


\begin{definition}
  \label{def:loop_walk}
  \lean{SimpleGraph.LoopWalk}
  \leanok
  Given a simple graph $G$, a LoopWalk is a walk on $G$ that is allowed to have consecutive visits
  to the same vertex (i.e. loops).
\end{definition}


Note the general definition of a LoopWalk does not require it to be a closed walk. However, we will
only need to work with closed LoopWalks in the proof of the semicircle law, so when we write
LoopWalk we mean closed LoopWalk below.


\begin{definition}
  \label{def:graph_walk_multi_index}
  \uses{def:loop_walk}
  Let $\mathbf{i} \in[n]^k$ be a $k$-index, $\mathbf{i}=\left(i_1, i_2, \ldots, i_k\right)$. The path
  $w_{\mathbf{i}}$ is the closed LoopWalk on the complete graph $K_n$ given by
  $$
  w_\mathbf{i}=((i_1, i_2),(i_2, i_3), \ldots,(i_{k-1}, i_k),(i_k, i_1)).
  $$
\end{definition}


\begin{definition}
  \label{def:graph_walk_edges}
  \uses{def:loop_walk}
  Given a LoopWalk $w_\mathbf{i}=((i_1, i_2),(i_2, i_3), \ldots,(i_{k-1}, i_k),(i_k, i_1))$ on
  $K_n$ , let
  $E_\mathbf{i}= \{\{i_1, i_2\},\{i_2, i_3\}, \ldots,\{i_{k-1}, i_k\}, \{i_k, i_1\}\}$ be the set
  of edges traversed by $w_{\mathbf{i}}$ (this includes loops).
\end{definition}


\begin{definition}
  \label{def:graph_walk_vertices}
  \uses{def:loop_walk}
  Given a LoopWalk $w_\mathbf{i}=((i_1, i_2),(i_2, i_3), \ldots,(i_{k-1}, i_k),(i_k, i_1))$ on
  $K_n$, let $V_\mathbf{i}= \{i_1, i_2, \dots, i_k\}$ be the set
  of vertices visited by $w_{\mathbf{i}}$.
\end{definition}



\begin{proposition}%[Exercise 4.3.1 in \cite{Kemp2013RMTNotes}]
  \label{prop:vertex_edge_inequality}
  \uses{def:graph_walk_vertices, def:graph_walk_edges}
  %\uses{}
  Let $w_{\mathbf{i}}$ be a LoopWalk. Then,
  $|V_{\mathbf{i}}|\le |E_{\mathbf{i}}|+1$.
\end{proposition}

\begin{proof}
  $|V|\le |E|+1$: proof by induction on $|V|$. Base case $|V| = 1$ is obvious.
  For each additional vertex, the number of edges must increase by at least one.
\end{proof}


\begin{definition}[Graph Edge Count]
    \label{def:graph_edge_count}
    \uses{def:graph_walk_multi_index}
    Let $\mathbf{i} \in[n]^k$ be a $k$-index, $\mathbf{i}=\left(i_1, i_2, \ldots, i_k\right)$.
    For any edge $e=\{i,j\}$ of $K_n$ traversed in $w_{\mathbf{i}}$, we define the edge count
    $w_{\mathbf{i}}(e)$ as the number of times edge $e$ is traversed, and if
    $\{i, j\} \notin w_{\mathbf{i}}$, then $w_{\mathbf{i}}(\{i, j\}) = 0$.
\end{definition}


\begin{definition}[Self Edges]
    \label{def:graph_self_edges}
    \uses{def:graph_walk_edges}
    Let $\mathbf{i} \in[n]^k$ be a $k$-index with walk $w_{\mathbf{i}}$ on $K_n$. Define the
    self-edges
    $E_{\mathbf{i}}^{s}$ as:
    $$
    \{\{i, i\} \in E_{\mathbf{i}}\},
    $$
    i.e the set of loops in $w_{\mathbf{i}}$.
\end{definition}


\begin{definition}[Connecting Edges]
    \label{def:graph_connecting_edges}
    \uses{def:graph_walk_edges}
    Let $\mathbf{i} \in[n]^k$ be a $k$-index with walk $w_{\mathbf{i}}$ on $K_n$. Define the
    connecting-edges $E_{\mathbf{i}}^{c}$ as:
    $$
    \{\{i, j\} \in E_{\mathbf{i}} : i \neq j\}
    $$
\end{definition}


\begin{definition}[Length $|w_\mathbf{i}|$ : R-1-1 : def:length\_of\_w\_i]
  \label{def:length_of_w_i}
  \uses{def:graph_walk_multi_index}
  % We use the definition of the walk $w_\mathbf{i}$.
  Given a LoopWalk $w_\mathbf{i}$ generated by some $k$-index $\mathbf{i}$, we let $|w_\mathbf{i}|$
  denote the length of $w_\mathbf{i}$.
\end{definition}


\begin{lemma}[$|w_\mathbf{i}| = k$ : R-1-2 : lem:abs\_w\_i\_eq\_k]
  \label{lem:abs_w_i_eq_k}
  \uses{def:length_of_w_i,def:graph_edge_count}
  For any $k$-index $\mathbf{i}$, $|V_\mathbf{i}| \leq k$ and
  \[
  |w_\mathbf{i}| \equiv \sum_{e \in E_\mathbf{i}} w_\mathbf{i}(e) = k.
  \]
\end{lemma}

\begin{proof}
  Foremost, since the number of vertices
  are the number of distinct elements of the $k$-index $\mathbf{i}$, it clearly follows that $\#V_\mathbf{i} \leq k$.
  On the other hand, recall that each $w_i(e)$ denotes the number of times the edge $e \in E_\mathbf{i}$
  is traversed by the path $w_\mathbf{i}$.Since $|w_\mathbf{i}| = k$ by the construction of $w_\mathbf{i}$, it follows that
  \[
  |w_\mathbf{i}| \equiv \sum_{e \in E_\mathbf{i}} w_\mathbf{i}(e) = k.
  \]
\end{proof}


\begin{definition}
  \label{def:graph_walk_equiv}
  \uses{def:loop_walk}
  Given two LoopWalks $w_{\mathbf{j}} = ((j_1, j_2), \dots (j_{k-1}, j_k), (j_k, j_1))$ and
  $w_{\mathbf{i}}=((i_1, i_2), \dots (i_{k-1}, i_k), (i_k, i_1))$
   on $K_n$, we say that $w \sim v$ if there exists a permutation $\sigma \in S_n$ such that
   $w_{\sigma(\mathbf{j})} \equiv ((\sigma(j_1), \sigma(j_2)), \dots (\sigma(j_{k-1}), \sigma(j_k)), (\sigma(j_{k}), \sigma(j_1)))
   = w_{\mathbf{i}}$.
\end{definition}


\begin{lemma}
  \label{lem:graph_walk_equiv}
  \uses{def:graph_walk_equiv}
  The relation in \ref{def:graph_walk_equiv} is indeed an equivalence relation.
\end{lemma}


\begin{lemma}
  \label{lem:walk_vertex_card_equiv}
  \uses{def:graph_walk_equiv}
  If $w_{\mathbf{i}} \sim w_{\mathbf{j}}$, then $|V_{\mathbf{i}}| = |V_{\mathbf{j}}|$.
\end{lemma}


\begin{lemma}
  \label{lem:walk_edge_card_equiv}
  \uses{def:graph_walk_equiv}
  If $w_{\mathbf{i}} \sim w_{\mathbf{j}}$, then $|E_{\mathbf{i}}| = |E_{\mathbf{j}}|$.
\end{lemma}


\begin{lemma}
  \label{lem:walk_edge_count_equiv}
  \uses{def:graph_walk_equiv, def:graph_edge_count}
  If $w_{\mathbf{i}} \sim w_{\mathbf{j}}$, then the set of edge counts for $w_{\mathbf{i}}$ is the
  same as $w_{\mathbf{j}}$. In other words,
  $\{w_{\mathbf{i}}(i_1, i_2) , \dots, w_{\mathbf{i}}(i_{k-1}, i_k)\} =
  \{w_{\mathbf{j}}(j_1, j_2) , \dots, w_{\mathbf{j}}(j_{k-1}, j_k)\}$ as sets.
\end{lemma}


\begin{definition}[$\mathcal{G}_k$ : R-1-4 : def:g\_k]
  \label{def:g_k}
  \uses{def:length_of_w, def:graph_walk_equiv}
  Let $\mathcal{G}_k$ denote the set of all equivalence classes under \ref{def:graph_walk_equiv}
  of walks of length $k$ on $K_n$.
\end{definition}

\begin{lemma}
  \label{lem:graph_set_finite}
  \uses{def:g_k}
  $|\mathcal{G}_k|$ is finite.
\end{lemma}

\begin{proof}
  Requires proof.
\end{proof}


We will denote the equivalence classes in $\mathcal{G}_k$ with $w$, and particular representatives
of the equivalence classes will be denoted $w_{\mathbf{i}}$, where $\mathbf{i}$ is the multi-index
that induces $w_\mathbf{i}$. We will use the notation that $w_{\mathbf{i}} = w$ to denote that
$w_{\mathbf{i}}$ is an element of the equivalence class $w$. We will similarly write $|V_w|$ and
$|E_w|$ for the cardinality of the vertex and edge sets of an equivalence class. We will also define
the multiset $EC_w$, which contains the edge counts of every edge in $w$.


\begin{definition}[$|V_w|$ : ef:abs.V\_w]
  \label{def:abs.V_w}
  \uses{def:g_k, lem:walk_vertex_card_equiv, def:graph_walk_vertices}
  Given a  $w \in \mathcal{G}_k$, we define $|V_w|$ to be $|V_{\mathbf{i}}|$, where $w_{\mathbf{i}}$
  is a LoopWalk in the equivalence class $w$.
\end{definition}

\begin{definition}[$|E_w|$ : def:abs.E\_w]
  \label{def:abs.E_w}
  \uses{def:g_k, lem:walk_edge_card_equiv, def:graph_walk_edges}
  Given a  $w \in \mathcal{G}_k$, we define $|E_w|$ to be $|E_{\mathbf{i}}|$, where $w_{\mathbf{i}}$
  is a LoopWalk in the equivalence class $w$.
\end{definition}

\begin{definition}
  \label{def:edge_count_w}
  \uses{def:g_k, lem:walk_edge_count_equiv}
  Given $w \in \mathcal{G}_k$, we define $EC_w$ to be the multiset of edge counts of $w$.
\end{definition}


\begin{lemma}[Lemma 4.3 in \cite{Kemp2013RMTNotes} : R-1-9 : lem:lem\_4.3]
  \label{lem:lem_4.3}
  \uses{def:g_k,def:graph_walk_multi_index, def:graph_walk_equiv}
  Given $w \in \mathcal{G}_k$, we have
  \[
   |\{ \mathbf{i} \in [n]^k : w_\mathbf{i} = w \}| = n (n-1) \cdots (n - |V_w| + 1).
  \]
\end{lemma}

\begin{proof}
  By the way the equivalence relation is defined in Definition \ref{def:graph_walk_equiv},
  the fact that there are $n (n - 1) \cdots (n -|V_w| + 1)$ ways to assign $|V_w|$ distinct values
  from $[n]$ into the indices $i_1,...,i_{|V_w|}$ completes the proof.
\end{proof}


\begin{definition}[$\mathcal{G}_{k, w \geq 2}$ : R-1-14 : def:g\_k\_ge\_2]
  \label{def:g_k_ge_2}
  \uses{def:g_k,def:edge_count_w}
  Let $\mathcal{G}_{k,w \geq 2}$ be a subset of $\mathcal{G}_k$ in which the walk $w$ traverses
  each edge at least twice, i.e. $EC_w$ contains only elements 2 or larger.
\end{definition}


\begin{lemma}[$\#E \leq k/2$ : R-1-17 : lem:edge\_set\_order\_leq\_k\_over\_two]
  \label{lem:edge_set_order_leq_k_over_two}
  \uses{def:g_k_ge_2, def:abs.E_w}
  Given a $w \in \mathcal{G}_{k,w \geq 2}$, we must have $|E_w| \leq k/2$.
\end{lemma}

\begin{proof}
  Since $|w| = k$, if each edge in $G$ is traversed at least twice, then by construction of $w$ the
   number of edges is at most $k/2$.
\end{proof}


\begin{proposition}
  \label{prop:vertex_edge_tree_equality}
  \notready
  Let $G=(V,E)$ be a connected finite graph. Then, $|G|=\#V=\#E+1$ if and only if $G$ is a plane tree.
  \uses{prop:vertex_edge_inequality}
\end{proposition}

\begin{proof}
  \notready
  $|G|=\#V=\#E+1$ if $G$ is a plane tree is already in Lean: SimpleGraph.IsTree.card\_edgeFinset.

  $G$ is a plane tree if $|G|=\#V=\#E+1$: proof by induction on $\#V$. Base case $\#V = 1$ has no edges.
  Assume $\#V = \#E + 1$ for $\#V = k$. Now, consider a tree with $\#V = k+1$ nodes. Removing a leaf
  node leaves us with a tree with $\#V = k$ nodes. By IH, there are $k - 1$ edges. So including the
  leaf node gives us $k$ edges.
\end{proof}


\begin{lemma}
  \label{lem:vertex_bound}
  \uses{prop:vertex_edge_inequality, lem:edge_set_order_leq_k_over_two, def:abs.V_w}
  Let $w \in \mathcal{G}_{k,w \geq 2}$. Then $|V_w| \le k/2 + 1$.
\end{lemma}

\begin{proof}
  Follows directly from earlier lemmas (replacing $\#E$ with $k/2$).
\end{proof}


\begin{lemma}
  \label{lem:odd_vertex_bound}
  \uses{lem:vertex_bound}
  Let $w \in \mathcal{G}_{k,w \geq 2}$.
  Suppose $k$ odd. Then, $|V_w| \le \frac{k}{2} + \frac{1}{2}$.
\end{lemma}

\begin{proof}
  %prove
\end{proof}


\begin{lemma}
  \label{lem:edge_bound_large_w}
  \uses{def:abs.E_w}
  If $w \in \mathcal{G}_{k,w \geq 2}$, $k$ is even, and there exists $e$ such that $w(e) \ge 3$,
  then $|E_w| \le \frac{k-1}{2}$.
\end{lemma}

\begin{proof}
  %proof
\end{proof}


\begin{proposition}%[Proposition 4.4 in \cite{Kemp2013RMTNotes}]
  \label{prop:g_bound_self_edge}
  \uses{prop:vertex_edge_inequality}
  Let $w\in\mathcal{G}_{k,w \geq 2}$, and suppose $k$ is even. If there exists a loop in $w$,
  then $|V_w|\le k/2$.
\end{proposition}

\begin{proof}

\end{proof}


\begin{proposition}%[Proposition 4.4 in \cite{Kemp2013RMTNotes}]
  \label{prop:g_bound_large_w}
  \uses{prop:vertex_edge_inequality, prop:vertex_edge_tree_equality, lem:edge_bound_large_w}
  Let $w \in\mathcal{G}_{k,w \geq 2}$, and suppose $k$ is even. If there exists $e\in EC_w$
  with $e \ge 3$, then $|V_w|\le k/2$.
\end{proposition}

\begin{proof}
  The sum of $w$ over all edges $E$ in $G$ is $k$.  Hence, the sum of $w$ over $E\setminus\{e\}$ is $\le k-3$.
  Since $w\ge 2$, this means that the number of edges excepting $e$ is $\le (k-3)/2$; hence, $\#E \le (k-3)/2+1 = (k-1)/2$.
  By the result of a previous lemma, this means that $\#V \le (k-1)/2+1 = (k+1)/2$.
  Since $k$ is even, it follows that $|V_w|=\#V \le k/2$.
\end{proof}


\begin{definition}
  \label{def:special_set_g}
  \uses{def:g_k}
  Let $\mathcal{G}^{k/2+1}_k$ to be the set of $w\in\mathcal{G}_k$ where $|V_w|=k/2+1$,
  contains no self-edges, and the walk $w$ crosses every edge exactly $2$ times, i.e.
  $EC_w =\{2, \dots, 2\}$.
\end{definition}


\begin{lemma}
  \label{lem:special_g_edge_count}
  \uses{def:special_set_g, prop:g_bound_large_w}
  Elements of $\mathcal{G}_k^{k/2+1}$ have $|E_w| = k/2$.
\end{lemma}

\begin{proof}
  %proof
\end{proof}


\begin{lemma}
  \label{lem:special_g_vertex_count}
  \uses{def:special_set_g}
  Elements of $\mathcal{G}_k^{k/2+1}$ have $|V_w| = k/2 + 1$.
\end{lemma}

\begin{proof}
  By definition.
\end{proof}


\begin{lemma}
  \label{lem:special_g_tree}
  \uses{def:special_set_g, lem:special_g_edge_count, prop:vertex_edge_tree_equality}
  Elements of $\mathcal{G}_k^{k/2+1}$ are trees.
\end{lemma}

\begin{proof}
  %proof
\end{proof}


\begin{definition}
  \notready
  \label{def:Dyck_paths}
  A Dyck path of length $k$ is a sequence $(d_1,...,d_k) \in \{\pm 1\}^k$ such that their partial sum $\sum_{i=1}^j d_i \geq 0$
  and total sum $\sum_{i = 1}^{k}d_i = 0$. More intuitively, consider a diagonal lattice path from $(0,0)$ to $(k, 0)$ consisting of
  $\frac{k}{2}$ ups and $\frac{k}{2}$ downs such that the path never goes below the $x$-axis.
\end{definition}


\begin{definition}
  \notready
  \label{def:graph_to_Dyck_map}
  \uses{def:special_set_g}
   Define a map $\phi$ whose input is $w \in \mathcal{G}^{k/2 + 1}_k$. Then for its output,
   define a sequence $\mathbf{d}=\mathbf{d}(w)\in\{+1,-1\}^k$ recursively as follows.
   Let $d_1=+1$.  For $1<j\le k$, if $w_j\notin\{w_1,\ldots,w_{j-1}\}$, set $d_j=+1$; otherwise, set $d_j=-1$; then
   $\mathbf{d}(w) = (d_1,\ldots,d_k)$. Set $\phi((w)) = \mathbf{d}(w)$.
\end{definition}


\begin{lemma}
  \notready
  \label{lem:graph_Dyck_correspondence}
  \uses{def:Dyck_paths, def:graph_to_Dyck_map}
  $\phi(w) = \mathbf{d}(w) \in \mathcal{D}_k$, where $\mathcal{D}_k$ denotes the set of Dyck path of order $k$.
\end{lemma}

\begin{proof}
  \notready
  set $P_0 = (0,0)$ and $P_j = (j,d_1+\cdots+d_j)$ for $1\le j\le k$; then the piecewise linear path
  connecting $P_0,P_1,\ldots,P_k$ is a lattice path.  Since $(G,w)\in\mathcal{G}_k^2$, each edge appears exactly two times in $w$,
  meaning that the $\pm1$s come in pairs in $\mathbf{d}(G,w)$.  Hence $d_1+\cdots+d_k=0$.  What's more, for any edge $e$,
  the $-1$ assigned to its second appearance in $w$ comes {\em after} the $+1$ corresponding to its first appearance;
  this means that the partial sums $d_1+\cdots+d_j$ are all $\ge 0$.  That is: $\mathbf{d}(G,w)$ is a Dyck path
\end{proof}


\begin{definition}
  \notready
  \label{def:Dyck_to_graph_map}
  Define a map $\psi$ whose input is a Dyck path of order $k$: $\mathbf{d}_k \in \{\pm1\}^k$. Then the output is viewing
  this Dyck path as a contour reversal of a tree where an up ($d_i = 1$) corresponds to visiting a child node
  and a down ($d_i = -1$) corresponds to returning to parent node.
\end{definition}


\begin{lemma}
  \notready
  \label{lem:Dyck_graph_correspondence}
  \uses{def:Dyck_paths, def:Dyck_to_graph_map}
  $\psi(\mathbf{d}_k) \in \mathcal{G}^{k/2 + 1}_k$
\end{lemma}

\begin{proof}
  \notready
  Use induction on the order of Dyck path $k$ which is an even number. Assume $\phi(\mathbf{d}_{k-2})
  \subseteq \mathcal{G}_{k-2}^{k/2 -1}$. In the case of $k$, the last two steps appended to $\mathbf{d}_{k-2}$
  has to be $1$ followed by $-1$ in order for $\mathbf{d}_k$ to be a Dyck path. By induction hypothesis,
  this generates a graph with one extra vertex from the parent node, whose walk traversed at the last two steps of the walk.
\end{proof}


\begin{lemma}
  \notready
  \label{lem:composition1}
  \uses{def:Dyck_to_graph_map, def:graph_to_Dyck_map, lem:Dyck_graph_correspondence}
  $$\phi \circ \psi = id_{\mathcal{D}_k}$$
\end{lemma}

\begin{proof}
  \notready
  % \leanok
  Apply $\psi$ to a given Dyck path $\mathbf{d}$ by definition, then apply $\phi$ to get a new sequence
  $\mathbf{d}'$ such that $d_j' = 1$ for new vertex, $-1$ otherwise. For the original Dyck path, the new
  vertex is the up step corresponding to $1$. This implies $\phi$ recovers the original Dyck path.
\end{proof}


\begin{lemma}
  \notready
  \label{lem:composition2}
  \uses{def:Dyck_to_graph_map, def:graph_to_Dyck_map, lem:graph_Dyck_correspondence}
  $$\psi \circ \phi = id_{\mathcal{G}^{k/2 + 1}_k}$$
\end{lemma}

\begin{proof}
  \notready
  % \leanok
  The map $\psi$ recovers the graph walk structure of the input from its Dyck path by the definition.
\end{proof}


\begin{lemma}
  \notready
  \label{lem:walk_to_Dyck_paths_bijection}
  \uses{lem:composition1, lem:composition2}
  Let $k$ be even and let $\mathcal{D}_k$ denote the set of Dyck paths of length $k$
  \[ \mathcal{D}_k = \{(d_1,\ldots,d_k)\in\{\pm 1\}\colon \sum_{i=1}^k d_i\ge 0\text{ for }1\le j\le j\text{, and}\sum_{i=1}^kd_i=0\}. \]
  Then $w\mapsto {d}(w)$ is a bijection $\mathcal{G}_k^{k/2+1}\to\mathcal{D}_k$.
\end{lemma}

\begin{proof}
  \notready
  obvious from the previous lemmas.
\end{proof}


\begin{definition}
  \mathlibok
  \label{def:Catalan_number}
  %\lean{Combinatorics.Enumerative.Catalan}
  \[C_0 = 1, \quad \text{and for } n \geq 1, \quad C_n = \sum_{k=0}^{n-1} C_k C_{n-1-k}.\]
\end{definition}


\begin{lemma}
  \notready
  \label{lem:binary_tree_Catalan_number}
  \uses{def:Catalan_number}
  \#\{binary trees with $\frac{k}{2}$ vertices\} is given by Catalan number $C_{k / 2}$
\end{lemma}

\begin{proof}
  \notready
\end{proof}


\begin{proposition}
  \notready
  \label{prop:Catalan_Dyck_samecardinality}
  \uses{def:Catalan_number, def:Dyck_paths, lem:binary_tree_Catalan_number}
  \[|\mathcal{D}_k| = C_{k/2} \] where $|\mathcal{D}_k|$ denotes the number of Dyck paths of length $k$ while
  $C_k$ is the $k$th Catalan number.
\end{proposition}

\begin{proof}
  \notready
  Given a binary tree with $k$ nodes, perform preorder traversal: for each internal node visited, write
  an up-step \(U = (1,1)\). For each time return from a child, write a down-step \(D = (1,-1)\).
  Since every internal node has exactly two children, there are \(k/2\) \(U\)'s and \(k/2\) \(D\)'s,
  giving a Dyke path of length \(k\). Conversely, given a Dyck path, $U$ is interpreted as adding new node
  while $D$ is returning to the parent node.
\end{proof}


\begin{proposition}
    \notready
    \label{prop:graph_Catalan_number}
  \uses{prop:Catalan_Dyck_samecardinality, lem:walk_to_Dyck_paths_bijection}
  \[|\mathcal{G}^{k/2 + 1}_k| = C_{k/2}\]
\end{proposition}


\begin{definition}[Graph Union]
  \label{def:graph_union}
  \notready
  \uses{def:graph_walk_multi_index}
  Given $k$-indices $\mathbf{i} = (i_1, i_2, \cdots , i_{k}), \mathbf{j} = (j_1, j_2, \cdots , j_{k}) \in [n]^{k}$,
  and Graphs $G_{\mathbf{i}}, G_{\mathbf{j}}$, we define the graph union $G_{\mathbf{i} \# \mathbf{j}}$ as follows:
  $$
  G_{\mathbf{i} \# \mathbf{j}}  = G_{\mathbf{i}} \cup G_{\mathbf{j}} = (V_{\mathbf{i}} \cup V_{\mathbf{j}}, E_{\mathbf{i}} \cup E_{\mathbf{j}})
  $$
\end{definition}


\begin{definition}[Ordered Triple]
  \label{def:ordered_triple}
  \notready
  \uses{def:graph_union, def:graph_walk_multi_index}
  Given $k$-indices $\mathbf{i} = (i_1, i_2, \cdots , i_{k}), \mathbf{j} = (j_1, j_2, \cdots , j_{k}) \in [n]^{k}$,
  we define the ordered triple $(G_{\mathbf{i} \# \mathbf{j}}, w_{\mathbf{i}}, w_{\mathbf{j}})$ as follows:
  \begin{itemize}
      \item $G_{\mathbf{i} \# \mathbf{j}}$ is the graph union of the graphs defined by $\mathbf{i}$ and $\mathbf{j}$ from definition \ref{def:graph_union}
      \item $w_{\mathbf{i}}$ is the closed walk defined by $\mathbf{i}$ from definition \ref{def:graph_walk_multi_index}
      \item $w_{\mathbf{i}}$ is the closed walk defined by $\mathbf{j}$ from definition \ref{def:graph_walk_multi_index}
  \end{itemize}
\end{definition}


\begin{definition}[R-2-2 : def:graph\_walk\_triple\_set]
  \notready
  \label{def:graph_walk_triple_set}
  \uses{} % n/a
  We define $\mathcal{G}_{k,k}$ to be the set of connected graphs $G$ with $\leq 2k$ vertices,
  together with two paths each of length $k$ whose union covers $G$.
\end{definition}


\begin{definition}[R-2-3-1 : def:index\_pair]
  \notready
  \label{def:index_pair}
  \uses{def:g_k_j,def:graph_walk_triple_set}
  Given $(G,w,w') \in \mathcal{G}_{k,k}$, we say $(\mathbf{i},\mathbf{j})$ is an ordered pair of $k$-indexes generated by $(G,w,w')$
  if $\mathbf{i}$ and $\mathbf{j}$ are, repsectively, $k$-indexes generated by $w$ and $w'$ as in Definition \ref{def:g_k_j}.
\end{definition}


\begin{definition}[R-2-3-2 : def:index\_pair\_rel]
  \notready
  \label{def:index_pair_rel}
  \uses{} % n/a
  Let $\mathbf{i}_\lambda$ and $\mathbf{j}_\lambda$ be $k$-indexes for each $\lambda=1,2$.
  We say $(\mathbf{i}_1,\mathbf{j}_1) = (\mathbf{i}_2,\mathbf{j}_2)$ if and only if there exists a
  bijection $\varphi_1$ from the set of entries $\mathbf{i}_1$ onto the set of entries $\mathbf{i}_2$,
  and a bijection $\varphi_2$ from the set of entries $\mathbf{j}_1$ onto the set of entries $\mathbf{j}_2$ such that
  \begin{equation}
  \begin{split}
    \mathbf{i}_1 = (i_1,...,i_k) & \,\, \Longleftrightarrow \,\, \mathbf{i}_2 = \bigl( \varphi_1(i_1),\varphi_1(i_2),...,\varphi_1(i_k) \bigl)
  \end{split}
  \end{equation}
  and
  \begin{equation}
  \begin{split}
    \phantom{.}\mathbf{j}_1 = (j_1,...,j_k) & \,\, \Longleftrightarrow \,\, \mathbf{j}_2 = \bigl( \varphi_2(j_1),\varphi_2(j_2),...,\varphi_2(j_k) \bigl).
  \end{split}
  \end{equation}
\end{definition}


\begin{definition}[R-2-3-3 : def:graph\_walk\_triple\_rel]
  \notready
  \label{def:graph_walk_triple_rel}
  \uses{def:graph_walk_triple_set,def:ordered_triple,def:index_pair,def:index_pair_rel}
  Let $(G_{\mathbf{i} \# \mathbf{j}},w_\mathbf{i},w_\mathbf{j})$ be an ordered triple generated by
  two $k$-indexes $\mathbf{i}$ and $\mathbf{j}$, and let $(G,w,w') \in \mathcal{G}_{k,k}$.
  We say $(G_{\mathbf{i} \# \mathbf{j}},w_\mathbf{i},w_\mathbf{j}) = (G,w,w')$ if and only
  $(\mathbf{i},\mathbf{j}) = (\mathbf{i}^*,\mathbf{j}^*)$,
  where $(\mathbf{i}^*,\mathbf{j}^*)$ is an ordered pair of $k$-indexes generated by $(G,w,w')$.
\end{definition}


\begin{lemma}[R-2-3-5 : lem:common\_val\_prod\_of\_eq\_of\_graph\_walk\_triple\_rel]
  \notready
  \label{lem:common_val_prod_eq_of_graph_walk_triple_rel}
  \uses{def:common_val_prod_of,def:ordered_triple,def:graph_walk_triple_rel,lem:eq_equiv_eq_expect,lem:common_val_eq_of_index_pair_rel}
  Let $(G_{\mathbf{i}_1 \# \mathbf{j}_1},w_{\mathbf{i}_1},w_{\mathbf{j}_1})$ and
  $(G_{\mathbf{i}_2 \# \mathbf{j}_2},w_{\mathbf{i}_2},w_{\mathbf{j}_2})$ be ordered triples
  generated by their respective ordered pair of $k$-indexes, and $(G,w,w') \in \mathcal{G}_{k,k}$,
  If $(G_{\mathbf{i}_1 \# \mathbf{j}_1},w_{\mathbf{i}_1},w_{\mathbf{j}_2}) = (G,w,w')$ and
  $(G_{\mathbf{i}_1 \# \mathbf{j}_2},w_{\mathbf{i}_2},w_{\mathbf{j}_2}) = (G,w,w')$, then
%  If the ordered triples $(G_{\mathbf{i}\#\mathbf{j}},w_\mathbf{i},w_\mathbf{j})$ and $(G_{\mathbf{i}^*\#\mathbf{j}^*},w_{\mathbf{i}^*},w_{\mathbf{j}^*})$ are same up to relabeling, then
  \[
  \pi(G_{\mathbf{i}_1\#\mathbf{j}_1},w_{\mathbf{i}_1},w_{\mathbf{j}_1}) = \pi(G_{\mathbf{i}_2 \# \mathbf{j}_2},w_{\mathbf{i}_2},w_{\mathbf{j}_2}).
  \]
\end{lemma}

\begin{proof}
This follows from Lemma \ref{lem:eq_equiv_eq_expect}.
\end{proof}


\begin{definition}[R-2-6-1 : def:def:graph\_walk\_triple\_single\_edges]
  \notready
  \label{def:graph_walk_triple_single_edges}
  \uses{def:graph_union}
  Given a graph $G_{\mathbf{i} \# \mathbf{j}}$, let $E^s_{\mathbf{i} \# \mathbf{j}}$ denote the set of self-edges.
\end{definition}


\begin{definition}[R-2-6-2 : def:graph\_walk\_triple\_connected\_edges]
  \notready
  \label{def:graph_walk_triple_connected_edges}
  \uses{def:graph_union}
  Given a graph $G_{\mathbf{i} \# \mathbf{j}}$, let $E^c_{\mathbf{i} \# \mathbf{j}}$ denote the set of connecting edges.
\end{definition}


\begin{definition}[R-2-7 : def:edgeCountPair]
  \notready
  \label{def:edgeCountPair}
  \uses{def:ordered_triple}
  Given a graph $G_{\mathbf{i} \# \mathbf{j}}$,
  let $w_{\mathbf{i} \# \mathbf{j}}(e)$ denote the number of times the edge $e$ is traversed by either of the two paths $w_\mathbf{i}$ and $w_\mathbf{j}$.
\end{definition}


\begin{lemma}[R-2-11 : lem:sum\_count\_edge\_pair\_eq\_length\_add\_length]
  \notready
  \label{lem:sum_count_edge_pair_eq_length_add_length}
  \uses{def:edgeCountPair,def:graph_union,def:graph_walk_multi_index,def:graph_edge_count}
  \[
  \sum_{e \in E_{\mathbf{i} \# \mathbf{j}}} w_{\mathbf{i} \# \mathbf{j}}(e)
  = 2k
  = \sum_{e \in E_\mathbf{i}} w_{\mathbf{i}}(e) + \sum_{e \in E_\mathbf{j}} w_{\mathbf{j}}(e).
  \]
\end{lemma}

\begin{proof}
  By construction of the paths $w_\mathbf{i}$ and $w_\mathbf{j}$,
  \[
  \sum_{e \in E_\mathbf{i}} w_{\mathbf{i}}(e) = k = \sum_{e \in E_\mathbf{j}} w_{\mathbf{j}}(e).
  \]
  By definition of the counting function $w_{\mathbf{i} \# \mathbf{j}}(\cdot)$,
  \[
  \sum_{e \in E_{\mathbf{i} \# \mathbf{j}}} w_{\mathbf{i} \# \mathbf{j}}(e)
  = \sum_{e \in E_\mathbf{i}} w_{\mathbf{i}}(e) + \sum_{e \in E_\mathbf{j}} w_{\mathbf{j}}(e)
  = 2k.
  \]
\end{proof}


\begin{definition}[R-2-2 : def:graph\_walk\_triple\_set]
  \notready
  \label{def:graph_walk_triple_set_w_ge_two}
  \uses{def:graph_walk_triple_set} % n/a
  $(G,w,w')\in \mathcal{G}_{k,k,w+w'\ge 2}$ if $(G,w,w') \in \mathcal{G}_{k,k}$ and $w + w' \ge 2$.
\end{definition}


\begin{proposition}
  \label{lem:g_w_w_count}
  \notready
  \uses{def:index_pair_rel, def:graph_walk_triple_set_w_ge_two, def:graph_walk_triple_rel}
  For $(G, w, w') \in \mathcal{G}_{k,k,w+w'\ge 2}$, $\#\left\{(i,{j})\in [n]^{2k}\colon (G_{i\#j},w_{i},w_{j}) = (G,w,w')\right\} = n(n-1)\cdots (n-|G|+1)$.
\end{proposition}

\begin{proof}
  \notready
  The enumeration of the number of $2k$-tuples yielding a certain graph with two walks is the same
  as in the previous proposition: the structure $(G,w,w')$ specifies the $2k$-tuple precisely once
  we select the $|G|$ distinct indices for the vertices.  So, as before, this ratio becomes \[ \frac{n(n-1)\cdots (n-|G|+1)}{n^{k+2}}. \]
\end{proof}


\begin{lemma}
  \label{lem:g_k_k_edge_count_maximum}
  \uses{def:graph_walk_triple_set_w_ge_two}
  \notready

  In the set $\mathcal{G}_{k,k,w+w'\ge 2}$, edge count is at most $k$.
\end{lemma}

\begin{proof}
  \notready
  Now, we have the condition $w+w'\ge 2$, meaning every edge is traversed at least twice.
  Since there are $k$ steps in each of the two paths, this means there are at most $k$ edges.
\end{proof}


\begin{lemma}
  \label{lem:exactly_k_edges}
  \uses{prop:vertex_edge_inequality}
  \notready
  $G \in \mathcal{G}_{k, k}$. If $|G| = k + 1$, then $G$ has exactly $k$ edges.
\end{lemma}


\begin{lemma}
  \label{lem:G_is_tree}
  \uses{prop:vertex_edge_tree_equality}
  \notready
  $G \in \mathcal{G}_{k, k}$. If $|G| = k + 1$, then $G \in \mathcal{G}_{k, k}$ is a tree.
\end{lemma}


\begin{lemma}
  \label{lem:subgraph_of_tree}
  \uses{lem:G_is_tree, def:graph_union, lem:special_g_tree}
  \notready
  $G_{\mathbf{i} \# \mathbf{j}}$ is a tree, then its subgraph $G_{\mathbf{i}}$ and $G_{\mathbf{j}}$ are also trees.
\end{lemma}


\begin{lemma}
  \label{lem:traverse_exactly_twice}
  \notready
  \uses{lem:exactly_k_edges}     % also depends on  total length of walk i, j is exactly 2k
  $(G, w, w') \in \mathcal{G}_{k, k}$. If $|G| = k + 1$, then for a common edge $e$ between the two
  walks, $w(e) + w'(e) =2$. In other words, every edge is traversed exactly twice.
\end{lemma}

\begin{proof}
  Assume for contradiction there is one edge $l$ such that $w(l) + w'(l) > 2$, then $\sum_{e \in G} w(e) + w'(e) > 2k$.
\end{proof}


%maybe require the statement that they are closed walks
\begin{lemma}
  \label{lem:i_j_traverse_once}
  \notready
  \uses{lem:traverse_exactly_twice, lem:subgraph_of_tree}
    $(G, w, w') \in \mathcal{G}_{k, k}$ and $|G| = k + 1$, then for a common edge it is impossible that $w(e) = w'(e) = 1$.
\end{lemma}

\begin{proof}
  Each walk is a closed walk, which implies their edge needs to be traversed in even number.
\end{proof}


\begin{lemma}
  \label{lem:one_walk_traverse_twice}
  \notready
  \uses{lem:i_j_traverse_once}
    $(G, w, w') \in \mathcal{G}_{k, k}$ and $|G| = k + 1$, then it can only be the case that
    $w(e) = 2, w'(e) = 0$, or $w(e) = 0, w'(e) = 2$ .
\end{lemma}

\begin{proof}
As a common edge, it is impossible that $w'$ does not traverse it.
\end{proof}


\begin{lemma}
  \label{lem:no_shared_edges}
  \uses{lem:one_walk_traverse_twice}
  $(G, w, w') \in \mathcal{G}_{k, k}$ and $|G| = k + 1$. Then two walks have no shared edges $e$ which they have traversed.
\end{lemma}


\begin{lemma}
  \label{lem:disjoint_edge_set}
  \notready
  \uses{lem:no_shared_edges}
  The only graph walks $(G,w,w')\in \mathcal{G}_{k,k}$ with $|G|=k+1$ must have the edge sets covered by $w$ and $w'$ distinct.
  In other words, if $(G_{\mathbf{i}\#\mathbf{j}},w_\mathbf{i},w_\mathbf{j}) = (G,w,w')$, then the edge sets do not intersect:
  $\{\{i_1,i_2\},\ldots,\{i_k,i_1\}\}\cap\{\{j_1,j_2\},\ldots,\{j_k,j_1\}\}=\emptyset$.
\end{lemma}


%depends on the definition of \mathcal{G}_{k,k} and \pi(G, w, w'    %R-2-5-1)
\begin{lemma}
  \label{lem:G_leq_k}
  \notready
  \uses{lem:disjoint_edge_set, def:common_val_prod}
  For any $(G_{\mathbf{i}\#\mathbf{j}},w_\mathbf{i},w_\mathbf{j}) \in \mathcal{G}_{k,k}$, with
  $k + 1$ vertices, $\pi(G_{\mathbf{i}\#\mathbf{j}},w_\mathbf{i},w_\mathbf{j})  = 0$
\end{lemma}

\begin{proof}
  $\pi(G_{\mathbf{i}\#\mathbf{j}},w_\mathbf{i},w_\mathbf{j}) = \mathbb{E}(X_{\mathbf{i}}X_{\mathbf{j}}) - \mathbb{E}(X_{\mathbf{i}})\mathbb{E}(X_{\mathbf{j}})  = 0$
\end{proof}
