\chapter{Random Matrices}

\section{Random Matrices}

Below, we let $M_n(\bR)$ denote the space of $n \times n$ real matrices. We only consider matrices with real entries here, so we drop the
dependence on $\bR$ in the notation and just write $M_n$ to denote the space of $n \times n$ real matrices. $M_n$ comes with a natural inner product structure, namely the Hilbert-Schmidt inner product.
Explicitly, $\langle A, B\rangle = \Tr(B^{T}A) = \sum_{i, j = 1}^n A_{ij}B_{ij}$. This inner product generates a metric on $M_n$ making it into a Polish space.
We then equip $M_n$ with the Borel $\sigma$-algebra generated by this metric.

Let $E_n$ be a subset of $M_n$.

Let $K$ be a field. Let

\begin{definition}
  \label{def:realValuedSquareRandomMatrix}
  \notready
  A real-valued square random matrix is given by a tuple $(\Omega, \phi_n)$, where $\Omega$ is a probability space $(\Omega, \mathcal{F}, \mu)$, and $\phi_n: \Omega \to M_n$ is a measurable function.
\end{definition}

\begin{definition}
  \label{def:realValuedSquareRandomMatrix_pushforward_measure}
  \uses{def:realValuedSquareRandomMatrix}
  \notready
  The map $\phi_n:\Omega \to M_n$ induces a measure on $M_n$ via the pushforward.
\end{definition}

\begin{definition}
  \label{def:WignerNbyNProbabilitySpace}
  \notready
  Fix two probability spaces $(\Omega_D, \mathcal{F}_D, \mu_D)$ and $(\Omega_O, \mathcal{F}_O, \mu_O)$.
  $(\Omega_D, \mathcal{F}_D, \mu_D)$ is the probability space that determines the distribution of the diagonal entries, and $(\Omega_O,\mathcal{F}_O, \mu_O)$ is the probability space that determines the distribution of the off-diagonal entries.
  The \textit{Wigner probability space} is defined as
  \[
  \Omega_n \coloneqq \Omega_D^{\otimes n} \otimes \Omega_O^{\otimes n \choose 2},
  \]
  equipped with the product $\sigma$-algebra. The measure on $\Omega$ is the product measure $\mu = \mu_D^{\otimes n} \otimes \mu_O^{\otimes n \choose 2}$.
\end{definition}

\begin{definition}
  \label{def:symmetricNbyNIndependentEntriesRandomMatrixUnnormalized}
  \uses{def:WignerNbyNProbabilitySpace, def:realValuedSquareRandomMatrix}
  \notready
  Let $M_n$ denote the space of $n \times n$ real-valued matrices. An unormalized Wigner random matrix is a measurable function $W_n :\Omega_n \to M_n$ such that $(W_n(\omega))_{i, j} = (W_n(\omega))_{j, i}$ for $i \neq j$, $i, j \in [n]$.
  Further, we require the following identically distributed condition. For any Borel set $A \subseteq \bR$, and $i, j \in [n]$
  \[
  \mu(\{\omega \in \Omega \vert (W_n(\omega))_{ii} \in A\}) = mu(\{\omega \in \Omega \vert (W_n(\omega))_{jj} \in A\}).
  \]
  For the off diagonal entries, we require, for any $1 \leq i < j \leq n$ and $1 \leq k < l \leq n$,
  \[
  \mu(\{\omega \in \Omega \vert (W_n(\omega))_{ij} \in A\}) = mu(\{\omega \in \Omega \vert (W_n(\omega))_{kl} \in A\}).
  \]
  Finally, we require that the collection of random variables $\{W_{i, j}\}_{1 \leq i \leq j \leq n}$ is an independent collection.
\end{definition}

\begin{definition}
  \label{def:symmetricNbyNIndependentEntriesRandomMatrix}
  \uses{def:symmetricNbyNIndependentEntriesRandomMatrixUnnormalized, def:WignerNbyNProbabilitySpace}
  \notready
  An $n \times n$ Wigner matrix is defined by the map $X_n(\omega) = \frac{1}{\sqrt{n}}W_n(\omega) :\Omega \to M_n$.
\end{definition}

\begin{lemma}
  \label{lem:symmetricNbyNIndependentEntriesRandomMatrix_map_measurable}
  \uses{def:symmetricNbyNIndependentEntriesRandomMatrix}
  \notready
  The map $X_n$ in the definition of the Wigner matrix is a measurable map from $\Omega \to M_n$.
\end{lemma}

\section{Empirical Spectral Distributions}


\begin{definition}[Empirical Spectral Distribution]
  \label{def:empiricalSpectralDistribution}
  \uses{def:empiricalProbabilityDistribution}
  \notready
  Let $X_n$ be an $n \times n$ random matrix with eigenvalues $\lambda_1, \dots, \lambda_n$. Then the \textit{empirical spectral distribution} associated to $X_n$ is the random probability measure
  \[
  \mu_{X_n}(\omega) = \sum_{i=1}^n \delta_{\lambda_i(\omega)}.
  \]
\end{definition}

\begin{lemma}
  \label{lem:pth_moment_ESD_trace_pth_power}
  \uses{def:empiricalSpectralDistribution, def:realValuedSquareRandomMatrix}
  \notready
  The $p$th moment of the ESD of a random matrix $X_n$ is equal to the trace of the $p$th power of $X_n$, i.e.
  \[
  \int_{bR} x^p \, d\mu_{X_n}(\omega) = \frac{1}{n}\Tr[X_n^p]
  \]
\end{lemma}
